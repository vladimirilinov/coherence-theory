% Options for packages loaded elsewhere
\PassOptionsToPackage{unicode}{hyperref}
\PassOptionsToPackage{hyphens}{url}
%
\documentclass[
]{article}
\usepackage{amsmath,amssymb}
\usepackage{lmodern}
\usepackage{iftex}
\ifPDFTeX
  \usepackage[T1]{fontenc}
  \usepackage[utf8]{inputenc}
  \usepackage{textcomp} % provide euro and other symbols
\else % if luatex or xetex
  \usepackage{unicode-math}
  \defaultfontfeatures{Scale=MatchLowercase}
  \defaultfontfeatures[\rmfamily]{Ligatures=TeX,Scale=1}
\fi
% Use upquote if available, for straight quotes in verbatim environments
\IfFileExists{upquote.sty}{\usepackage{upquote}}{}
\IfFileExists{microtype.sty}{% use microtype if available
  \usepackage[]{microtype}
  \UseMicrotypeSet[protrusion]{basicmath} % disable protrusion for tt fonts
}{}
\makeatletter
\@ifundefined{KOMAClassName}{% if non-KOMA class
  \IfFileExists{parskip.sty}{%
    \usepackage{parskip}
  }{% else
    \setlength{\parindent}{0pt}
    \setlength{\parskip}{6pt plus 2pt minus 1pt}}
}{% if KOMA class
  \KOMAoptions{parskip=half}}
\makeatother
\usepackage{xcolor}
\IfFileExists{xurl.sty}{\usepackage{xurl}}{} % add URL line breaks if available
\IfFileExists{bookmark.sty}{\usepackage{bookmark}}{\usepackage{hyperref}}
\hypersetup{
  hidelinks,
  pdfcreator={LaTeX via pandoc}}
\urlstyle{same} % disable monospaced font for URLs
\usepackage{longtable,booktabs,array}
\usepackage{calc} % for calculating minipage widths
% Correct order of tables after \paragraph or \subparagraph
\usepackage{etoolbox}
\makeatletter
\patchcmd\longtable{\par}{\if@noskipsec\mbox{}\fi\par}{}{}
\makeatother
% Allow footnotes in longtable head/foot
\IfFileExists{footnotehyper.sty}{\usepackage{footnotehyper}}{\usepackage{footnote}}
\makesavenoteenv{longtable}
\setlength{\emergencystretch}{3em} % prevent overfull lines
\providecommand{\tightlist}{%
  \setlength{\itemsep}{0pt}\setlength{\parskip}{0pt}}
\setcounter{secnumdepth}{-\maxdimen} % remove section numbering

% --- Minimal header for physics-style paper ---
\usepackage[utf8]{inputenc}
\usepackage[T1]{fontenc}
\usepackage{lmodern}
\usepackage{microtype}
\usepackage{geometry}
\geometry{margin=1in}
\usepackage{hyperref}
\hypersetup{colorlinks=true,linkcolor=blue,citecolor=blue,urlcolor=blue}
\usepackage{amsmath,amssymb,mathtools,amsthm}
\usepackage{bm}
\usepackage{physics}
\usepackage{siunitx}
\usepackage{graphicx}
\usepackage{booktabs}
\usepackage{titlesec}
\titleformat{\section}{\Large\bfseries}{\thesection}{1em}{}
\titleformat{\subsection}{\large\bfseries}{\thesubsection}{1em}{}
\titleformat{\subsubsection}{\normalsize\bfseries}{\thesubsubsection}{1em}{}
% Equation numbering
\numberwithin{equation}{section}
\ifLuaTeX
  \usepackage{selnolig}  % disable illegal ligatures
\fi

\author{}
\date{}

\begin{document}

{
\setcounter{tocdepth}{2}
\tableofcontents
}
\hypertarget{front-matter-abstract-global-hypotheses-h0h5}{%
\section{Front Matter --- Abstract \& Global Hypotheses
(H0--H5)}\label{front-matter-abstract-global-hypotheses-h0h5}}

\hypertarget{abstract}{%
\subsection{Abstract}\label{abstract}}

We formalize coherence as the staying‑power of a pattern under
admissible pokes, priced by three convex budgets---throughput,
complexity, and leakage---selected by symmetry and locality. We prove
budget minimal completeness (no fourth budget), poke‑ensemble
robustness, and a non‑teleological variational principle via
risk‑sensitive large deviations. Fast‑sector KKT on a C*‑quadratic fixes
\(\hbar\); slow‑sector Γ‑compactness recovers Einstein--Hilbert. Pointer
alignment follows from a unitary‑orbit minimizer. Multi‑cone geometries
pay a strict L¹ coherence penalty.

\hypertarget{global-hypotheses}{%
\subsection{Global hypotheses}\label{global-hypotheses}}

\textbf{H0 (Spaces).} Fast: separable Hilbert space \(\mathcal H\);
states \(\rho\in\mathfrak T_1(\mathcal H)\) (\(\|\cdot\|_1\)). Pokes:
CPTP maps (\(\|\cdot\|_\diamond\)). Slow: fields in \(H^1_{\rm loc}\)
modulo gauge/diffeo; Γ‑convergence frames locality. \textbf{H1
(Budgets).} Convex, l.s.c., coercive, invariance‑compatible, additive,
monotone under coarse‑graining; slow Γ‑limit is second‑order local.
\textbf{H2 (Operational measurability).} Budgets and \(\mathrm{CL}\)
estimable from finite experiments; probabilities continuous in
\(\|\cdot\|_\diamond\). \textbf{H3 (Coherence functional).} l.s.c. in
\(\Phi\); u.s.c. and concave in \(A\) on budget sublevels. \textbf{H4
(Causality/Locality for pokes).} Single light‑cone; Γ‑locality (no
super‑quadratic derivatives at selection scale). \textbf{H5 (Leakage
regularity).} Transfer‑kernel lower hull continuous; strict hull
convexity ⇒ unimodality; else a weak U‑shape suffices.

\hypertarget{chapter-1-coherence-theory-idea-law-roadmap}{%
\section{Chapter 1 --- Coherence Theory: Idea, Law, \&
Roadmap}\label{chapter-1-coherence-theory-idea-law-roadmap}}

\hypertarget{plain-idea}{%
\subsection{1.1 Plain idea}\label{plain-idea}}

Patterns that \textbf{keep working when poked} are selected. Call this
staying‑power \textbf{coherence}. Selection pays three prices:
\textbf{throughput} (fuel/time/compute), \textbf{complexity} (moving
parts/coordination), and \textbf{leakage} (unwanted
emissions/crosstalk). Environments poke within causal limits. The
winning scaffold maximizes predictive staying‑power minus these prices.

\hypertarget{assumptions-ledger-what-we-assume-where-it-is-used}{%
\subsubsection{1.1.1 Assumptions Ledger (what we assume, where it is
used)}\label{assumptions-ledger-what-we-assume-where-it-is-used}}

We assume exactly three hygiene items; everything else is derived.

\begin{itemize}
\tightlist
\item
  \textbf{(A) Poke cone}: a causal, Γ-local class
  \texttt{\textbackslash{}(\textbackslash{}mathcal\ P\textbackslash{})}
  of disturbances, closed under composition and mixing; product-topology
  continuity on finite windows. \textbf{Used in}: §1.2 (envelope), Ch. 2
  (operational l.s.c.), Ch. 5 (directional envelope), Ch. 9
  (microcausality).
\item
  \textbf{(B) Budgets}: three convex, l.s.c., coercive
  quadratics---\textbf{throughput} (B\_\{\rm th\}) (derivation-priced),
  \textbf{complexity} (B\_\{\rm cx\}) (Ad-invariant Hilbertian),
  \textbf{leakage} (B\_\{\rm leak\}) (Dirichlet-type on channels)---with
  norm-equivalent representatives and calibration stability.
  \textbf{Used in}: Ch. 2 (irreducible basis), Ch. 3 (fast sector/GKSL),
  Ch. 5 (multipliers), Ch. 7 (horizon).
\item
  \textbf{(C) Spaces/topologies}: quasi-local C* algebra for fast
  variables; cone-preserving Γ-compact slow sector (bounded geometry +
  gauge fixing). \textbf{Used in}: §1.2 (existence), Ch. 4 (Γ-limit ⇒
  EH), Ch. 5 (first-variation convergence).
\end{itemize}

\hypertarget{what-we-assume-vs.-what-we-derive-vs.-how-to-falsify}{%
\paragraph{1.1.2 What we assume vs.~what we derive vs.~how to
falsify}\label{what-we-assume-vs.-what-we-derive-vs.-how-to-falsify}}

\begin{longtable}[]{@{}
  >{\raggedright\arraybackslash}p{(\columnwidth - 6\tabcolsep) * \real{0.2500}}
  >{\raggedright\arraybackslash}p{(\columnwidth - 6\tabcolsep) * \real{0.2500}}
  >{\raggedright\arraybackslash}p{(\columnwidth - 6\tabcolsep) * \real{0.2500}}
  >{\raggedright\arraybackslash}p{(\columnwidth - 6\tabcolsep) * \real{0.2500}}@{}}
\toprule
\begin{minipage}[b]{\linewidth}\raggedright
Item
\end{minipage} & \begin{minipage}[b]{\linewidth}\raggedright
Assumed (minimal)
\end{minipage} & \begin{minipage}[b]{\linewidth}\raggedright
Derived (selection outputs)
\end{minipage} & \begin{minipage}[b]{\linewidth}\raggedright
Falsify if (single predictive bit)
\end{minipage} \\
\midrule
\endhead
Poke cone
\texttt{\textbackslash{}(\textbackslash{}mathcal\ P\textbackslash{})} &
Causal, Γ-local, closed under mixing/composition & Worst-case envelope
well-posed; directional minimizers exist & Empirically observe
\textbf{super-cone signaling} or non-local poke effects that violate
cone bounds \\
Budgets (B\_\{\rm th\},B\_\{\rm cx\},B\_\{\rm leak\}) & Convex, l.s.c.,
coercive; symmetries (Ad-invariance etc.) & \textbf{No fourth
independent budget} (irredundant 3-D span on feasible quotient);
constants = \textbf{multipliers} & A \textbf{fourth quadratic direction}
separates under the same symmetries/calibration \\
Fast sector & HS geometry on blocks; derivation price &
(\hbar=\lambda\_\{\rm th\}\^{}\{-1\}); \textbf{Heisenberg/GKSL} with
\textbf{pointer basis} (W-diagonalization) & Lab interferometer shows
\textbf{basis-invariant} decoherence contrary to W-alignment \\
Slow sector & Γ-compact, cone-preserving class & \textbf{Γ-limit ⇒
Einstein--Hilbert} scaffold; coupled EH--YM + GKSL at stationarity & GW
phasing residual slope departs from predicted envelope-multiplier law \\
Horizons & Same budgets; near-horizon cone & \textbf{Amplitude
suppression} of Hawking flux at \textbf{fixed temperature} & Detect a
\textbf{temperature shift} (leading order) instead of pure amplitude
suppression \\
\bottomrule
\end{longtable}

\emph{(Pointers to full proofs remain where those theorems live.)}

\hypertarget{the-coherence-law-auditable-form}{%
\subsection{1.2 The Coherence Law (auditable
form)}\label{the-coherence-law-auditable-form}}

Let
\(\mathcal A:= \mathcal A_{\rm fast}\times\mathcal A_{\rm slow}\times \mathcal D\)
be the product of the fast, slow, and discrete choices, equipped with
the product topology (trace/diamond‑side for fast; cone‑preserving weak
\(H^2\) for slow after gauge‑fixing; and the discrete topology on
\(\mathcal D\)). Let \(\mathcal P\) be the admissible poke cone (causal,
Γ‑local; closed under composition/mixing) and \(\overline{\mathcal P}\)
its diamond‑norm closure. Budgets \(B_{\rm th},B_{\rm cx},B_{\rm leak}\)
are convex, l.s.c., \textbf{coercive and invariance‑compatible}
(Appendix A); binders \(B_{\rm bind}^{(j)}\) are l.s.c. and either
indicator‑type or convex quadratics on submanifolds. The coherence
functional \(\mathrm{CL}\) is operational and l.s.c. in pokes (Lemma
II.2).

\[
A^* \in \underset{A\in\mathcal A}{\arg\max}\Big\{\underbrace{\inf_{\Phi\in\overline{\mathcal P}} \,\mathrm{CL}(A,\Phi)}_{:=\,\mathcal C(A)}\ -\ \lambda_{\rm th} B_{\rm th}(A)\ -\ \lambda_{\rm cx} B_{\rm cx}(A)\ -\ \lambda_{\rm leak} B_{\rm leak}(A)\ -\ \textstyle\sum_j \mu_j B_{\rm bind}^{(j)}(A)\Big\}.
\]

\hypertarget{operationalization-finite-protocols-and-risk-sensitive-limit}{%
\paragraph{Operationalization (finite protocols) and risk-sensitive
limit}\label{operationalization-finite-protocols-and-risk-sensitive-limit}}

\textbf{Finite-protocol representation.} There exists a family of
experimentally finite protocols (T\in\mathscr T) (finite POVMs plus
bounded continuous post-processings) such that {[}
\mathrm{CL}(A,\Phi)=\sup\_\{T\in\mathscr T\} F\_T(A,\Phi),\qquad
F\_T(A,\Phi):=g\_T!\big(p\_T(A,\Phi)\big), {]} with (T\mapsto F\_T)
continuous in the diamond norm. Hence (\Phi\mapsto \mathrm{CL}(A,\Phi))
is l.s.c. and \textbf{auditable}.

\textbf{Risk-sensitive aggregator.} For poke law (\Pi) and
(\beta\textgreater0), {[}
\mathrm{CL}\emph{\beta(A):=\frac{1}{\beta}\log\mathbb E}\{\Phi\sim\Pi\}\exp\big(\beta,\mathrm{CL}(A,\Phi)\big)
\searrow \inf\_\{\Phi\in\overline{\mathcal P}\}\mathrm{CL}(A,\Phi)\quad(\beta\to\infty)
{]} (epi-convergence). Thus the \textbf{worst-case envelope} is the
(\beta\to\infty) risk limit---no teleology is assumed.

\textbf{Global cross-domain KPI.} We track the \textbf{coherence number}
{[} \chi:=\frac{\tau_{\rm dec}}{\tau_{\rm mess}}, {]} the ratio of
\textbf{decoherence time} (fast, pointer-aligned) to \textbf{messenger
time} (slow, cone-propagating). (\chi) appears in both lab
interferometers and near-horizon tiles and is linked to multipliers by
the envelope identities (App. E.4).

\hypertarget{concrete-coherence-functionals-two-exemplars}{%
\paragraph{Concrete coherence functionals (two
exemplars)}\label{concrete-coherence-functionals-two-exemplars}}

We exhibit two \textbf{computable} coherence functionals ( \mathrm{CL} )
within the admissible class defined above. Both respect finite protocols
and the poke cone, and both yield the same selection outputs up to a
monotone transform (Appendix A1).

\textbf{(A) Classical toy-world (cellular-automaton) CL.}\\
State space: a finite grid ( \mathbb Z\_n\^{}2 ) with cell states
(S=\{0,1,2\}) for empty/scaffold/messenger. A pattern (A) is a seed
(a\in S\^{}\{n\times n\}) plus a local update rule (R\_\theta)
(finite-radius). A poke ( \Phi ) is a Markovian disturbance with
parameters ((p\_\{\rm noise\},q\_\{\rm adv\})) acting at each step for
(T) steps. A \textbf{finite protocol} (T\_\{\rm CA\}) fixes thresholds
((s\_\{\min\},\tau\emph{\{\rm hold\},\theta}\{\rm msg\})) and an
evaluation schedule (\mathcal S\subset\{1,\dots,T\}). Let (X) be the
trajectory under (A,\Phi).

Define three finite, observable functionals under (T\_\{\rm CA\}): -
(S\_\{A,\Phi\}\^{}\{T\_\{\rm CA\}\}:=\mathbb P{[}\text{there exists a 4-connected component of state \(1\) of size}\ge s\_\{\min\}\text{ that persists }\ge \tau\_\{\rm hold\}{]}).
-
(M\_\{A,\Phi\}\^{}\{T\_\{\rm CA\}\}:=\mathbb P\big[ \#\{t\in\mathcal S:\text{messenger mass in core} \ge m_{\min}\}\ge \theta_{\rm msg}|\mathcal S|\big]).
-
(L\_\{A,\Phi\}\^{}\{T\_\{\rm CA\}\}:=\mathbb E{[}\text{leakage events in }X{]}/L\_\{\rm cap\})
(dimensionless).

For weights (u=(\alpha,\beta,\gamma)\in\mathcal U\subset\Delta\emph{2)
(finite grid on the simplex), define the \textbf{CA protocol score} {[}
F}\{T\_\{\rm CA\},u\}(A,\Phi):=\alpha,S\_\{A,\Phi\}\textsuperscript{\{T\_\{\rm CA\}\}+\beta,M\_\{A,\Phi\}}\{T\_\{\rm CA\}\}-\gamma,L\_\{A,\Phi\}\^{}\{T\_\{\rm CA\}\}\in[-1,1].
{]} Then the \textbf{CA coherence functional} is {[}
\boxed{\ \mathrm{CL}_{\rm CA}(A,\Phi):=\max_{T_{\rm CA}\in\mathscr T_{\rm CA}}\ \max_{u\in\mathcal U}\ F_{T_{\rm CA},u}(A,\Phi)\ }.
{]} This is \textbf{finite-protocol}, \textbf{measurable}, and
\textbf{l.s.c.} in the diamond/trace product topology (Appendix A1,
Lemma A1.1). Under randomized mixtures of patterns (Section 1.1.1),
(A\mapsto \mathrm{CL}\_\{\rm CA\}(A,\Phi)) is \textbf{concave} (supremum
of linear expectations over a finite family composed with an affine
mixing).

\textbf{(B) Quantum toy (binary channel reliability) CL.}\\
Fix two finite-energy input states (\rho\_0,\rho\emph{1) encoding
``useful thing holds / fails.'' For (T) steps, a fast-sector pattern (A)
composed with poke (\Phi) induces a CPTP map (\mathcal N}\{A,\Phi\}).
Denote the outputs (\sigma\emph{i:=\mathcal N}\{A,\Phi\}(\rho\_i)). The
\textbf{optimal binary decision} error for distinguishing (\sigma\_0)
vs.~(\sigma\_1) is Helstrom's {[}
P\_e\^{}*(\sigma\_0,\sigma\_1)=\tfrac12\Big(1-\tfrac12\textbar{}\sigma\_0-\sigma\_1\textbar{}\emph{1\Big).
{]} Define the \textbf{quantum reliability score} {[}
F}\{\rm Q\}(A,\Phi):=1-P\_e\^{}*(\mathcal N\_\{A,\Phi\}(\rho\emph{0),\mathcal N}\{A,\Phi\}(\rho\emph{1))=\tfrac12\Big(1+\tfrac12\textbar{}\mathcal N}\{A,\Phi\}(\Delta)\textbar\_1\Big),
{]} with (\Delta:=\rho\_0-\rho\_1). Then {[}
\boxed{\ \mathrm{CL}_{\rm Q}(A,\Phi):=\inf_{\Phi'\in\overline{\mathcal P}}\ F_{\rm Q}(A,\Phi')\ }.
{]} Continuity of
(\mathcal N\mapsto \textbar{}\mathcal N(\Delta)\textbar\_1) in diamond
norm yields \textbf{l.s.c.} in ((A,\Phi)); mixing (A) gives
\textbf{concavity} in the mixture (Appendix A1, Lemma A1.2). Choosing
(\rho\_i) aligned with the environmental weight (W) connects this CL to
the pointer-basis selection in Chapter 3.

\begin{quote}
\textbf{Box 1.B --- Robustness (no fine-tuning).}\\
On any bounded window and admissible poke class, every CL in the family
{[}
\Big\{~\mathrm{CL}=\sup\emph{\{T\in\mathscr T\}~\mathbb E\big[s_T(Z_{A,\Phi})\big]~:~s\_T
\text{ is a bounded, concave proper score on a finite observable }Z~\Big\}
{]} is equivalent up to an \textbf{increasing bi-Lipschitz transform}.
Thus their maximizers under the same budgets coincide in the
(\Gamma)-limit, and multipliers
(e.g.~(\hbar=\lambda}\{\rm th\}\^{}\{-1\})) are invariant. \emph{(Proof:
Appendix A1, Prop. A1.3.)}
\end{quote}

\hypertarget{wellposedness-existence-direct-method-full-proofs}{%
\subsubsection{Well‑posedness \& existence (direct method --- full
proofs)}\label{wellposedness-existence-direct-method-full-proofs}}

\textbf{Notation.} Let C(A):= inf\_\{Phi in Pbar\} CL(A,Phi) and J(A):=
C(A) − lambda\_th B\_th(A) − lambda\_cx B\_cx(A) − lambda\_leak
B\_leak(A) − sum\_j mu\_j B\_bind\^{}\{(j)\}(A). Fix c\textless∞ and
denote S\_c := \{ A in A : lambda\_th B\_th(A)+lambda\_cx
B\_cx(A)+lambda\_leak B\_leak(A)+sum\_j mu\_j B\_bind\^{}\{(j)\}(A) ≤ c
\}.

\textbf{Prop. 1.2.1 (compact sublevels).} Under H1, H4 and the
cone‑preserving gauge‑fixed topology for the slow sector (Ch. 4), S\_c
is compact in the product topology fast × slow × discrete. \emph{Proof.}
By H1 each budget is lower semicontinuous and coercive in the stated
product topology, so its sublevel sets are precompact. Finite
intersections of precompact sets are precompact. Because all budgets and
binders are l.s.c., S\_c is closed; hence S\_c is compact. The
projection of S\_c to the discrete factor is precompact; in a discrete
space that implies finiteness, so only finitely many discrete labels
occur on S\_c.~□

\textbf{Prop. 1.2.2 (upper semicontinuity of A ↦ C(A)).} Under H3, C is
u.s.c. on A. \emph{Proof.} Fix A and ε\textgreater0. Choose Phi\_ε with
C(A) ≥ CL(A,Phi\_ε) − ε. For any net A\_α→A, u.s.c. of A ↦ CL(A,Phi\_ε)
on budget sublevels (H3) gives limsup\_α CL(A\_α,Phi\_ε) ≤ CL(A,Phi\_ε).
Hence limsup\_α C(A\_α) ≤ C(A)+ε. Let ε↓0. □

\textbf{Thm. 1.2.3 (existence of maximizers).} The set Argmax\_\{A in
A\} J(A) is nonempty. \emph{Proof.} By construction CL is bounded above
(built from probabilities; H2), so C(A) ≤ M for some finite M (normalize
M=1 w.l.o.g.). Let M*:=sup\_A J(A) and pick a maximizing net A\_α with
J(A\_α)→M*. Then for some finite c the budget sum at A\_α is ≤ c for all
large α, so A\_α∈S\_c.~By Prop. 1.2.1, S\_c is compact; pass to a
convergent subnet A\_\{α\_k\}→A*. By Prop. 1.2.2 and l.s.c. of budgets,
J is u.s.c. on S\_c, so M* = limsup\_k J(A\_\{α\_k\}) ≤ J(A*) ≤ M*. Thus
A* attains the supremum. □

\textbf{Cor. 1.2.4 (tightness of discrete choices).} Any maximizing net
is eventually supported on a finite subset of the discrete menu; the
discrete factor does not spoil compactness. □

\textbf{Lemma 1.2.5 (inner attainment or minimizing net).} For fixed A,
the map Phi ↦ CL(A,Phi) is l.s.c. on the closed set Pbar (Lemma II.2).
Hence: (i) if some sublevel \{Phi : CL(A,Phi) ≤ t\} is
diamond‑precompact for t\textgreater C(A), then there exists Phi*(A) ∈
Pbar with CL(A,Phi*)=C(A); (ii) in general, there exists a minimizing
net Phi\_β(A) with CL(A,Phi\_β) ↓ C(A). \emph{Proof.} (i) l.s.c. +
compact sublevel ⇒ minimum attained. (ii) pick a directed family of
ε‑minimizers with ε ↓ 0. □

\textbf{Thm. 1.2.6 (Danskin--Valadier envelope).} Assume for each Phi
the directional derivative D\^{}+\_A CL(A,Phi;h) exists on S\_c.~Then
for every A∈S\_c and direction h, D\^{}+ C(A;h) = inf\{ D\^{}+\_A
CL(A,Phi;h) : Phi ∈ Argmin(A) \}, with the right side understood as the
infimum over cluster points of minimizing nets when Argmin(A)=∅. If
Argmin(A) is nonempty and compact, the infimum is attained.
\emph{Proof.} Standard marginal‑function formula (Rockafellar--Wets,
Variational Analysis, Thm. 10.31) using u.s.c. in A and l.s.c. in Phi;
extend to noncompact Phi by epigraphical limits and minimizing nets
(Valadier 1973). □

\textbf{Remark (measurable selections).} On any Polish slice of S\_c the
multifunction A↦Argmin(A) admits Borel ε‑selections
(Kuratowski--Ryll‑Nardzewski) when nonempty/closed; when empty use the
minimizing‑net construction to define ε‑selectors, which suffices for
Appendix J's estimation schemes. \textbf{Lemma 1.2.A (precompact
sublevels \& u.s.c.).} For each \(c<\infty\), the sublevel set
\(\mathsf S_c:=\Big\{A\in\mathcal A:\ \lambda_{\rm th} B_{\rm th}(A)+\lambda_{\rm cx} B_{\rm cx}(A)+\lambda_{\rm leak} B_{\rm leak}(A)+\textstyle\sum_j\mu_j B_{\rm bind}^{(j)}(A)\le c\Big\}\)
is \textbf{precompact} in the product topology, and the map
\(A\mapsto \mathcal C(A)=\inf_{\Phi\in\overline{\mathcal P}}\mathrm{CL}(A,\Phi)\)
is \textbf{upper semicontinuous} on \(\mathsf S_c\). \emph{Sketch.}
Precompactness: fast‑sector coercivity (Appendix A) gives compactness of
generator/state sublevels in the trace/diamond product; slow‑sector
Γ‑compactness (Ch. 4) gives weak \(H^2\) precompactness under
cone‑preserving bounds; \(\mathcal D\) is either finite or handled by
the discrete clause below. Upper semicontinuity:
\(\Phi\mapsto\mathrm{CL}(A,\Phi)\) is l.s.c. in \(\|\cdot\|_\diamond\)
(Lemma II.2), and \(A\mapsto\mathrm{CL}(A,\Phi)\) is u.s.c. on budget
sublevels (H3); infima of such families preserve u.s.c. on
\(\mathsf S_c\).

\textbf{Lemma 1.2.B (outer maximizer exists).} The objective is u.s.c.
on each \(\mathsf S_c\) and \(\mathsf S_c\) is precompact; hence the
outer \(\arg\max\) is non‑empty. Any maximizing sequence has a
convergent subsequence with a maximizer in the limit.

\textbf{Discrete choices.} Either (a) work with a compactified
\(\overline{\mathcal D}\) (e.g., finite menu or one‑point
compactification that carries no advantage by coercivity), or (b) impose
a \textbf{finite‑improvement property}: on each \(\mathsf S_c\) only
finitely many discrete flips can strictly improve the objective (holds
when each flip increases at least one active budget by a fixed
\(\varepsilon>0\)). This ensures attainment over \(\mathcal D\).

\hypertarget{inner-attainment-envelope}{%
\subsubsection{Inner attainment \&
envelope}\label{inner-attainment-envelope}}

\textbf{Lemma 1.2.C (attainment/minimizing net for inner infimum).} For
fixed \(A\), the map \(\Phi\mapsto \mathrm{CL}(A,\Phi)\) is l.s.c. on
the closed set \(\overline{\mathcal P}\). Thus the infimum is attained
whenever the relevant slice of \(\overline{\mathcal P}\) is
diamond‑precompact; otherwise there exists a \textbf{minimizing net}
\(\Phi_\alpha(A)\) with
\(\mathrm{CL}(A,\Phi_\alpha)\downarrow \mathcal C(A)\). In either case,
the \textbf{Danskin--Valadier envelope} applies: for any direction
\(\delta A\),
\(D^+\mathcal C(A;\delta A)=\inf\{\partial_A\mathrm{CL}(A,\Phi)[\delta A]:\ \Phi\in \operatorname{Argmin}(A)\}\)
with the right derivative taken on \(\mathsf S_c\). This is the version
used later for KKT and constants‑as‑multipliers.

\textbf{Notes.} (i) \emph{No teleology} (risk‑sensitive
\(\beta\to\infty\) limit) is shown in Ch. 2. (ii) \emph{Canonical
budgets only} is Theorem I.1. They are referenced here but \textbf{not}
assumed---existence follows from Lemmas 1.2.A--C plus H0--H5.

\begin{quote}
\textbf{Box 1.A --- Three-budget sufficiency (no fourth direction).}\\
Under the admissible symmetries and calibration stability, the
admissible budgets span an \textbf{irreducible 3-dimensional cone} on
the feasible quotient. A \textbf{fourth independent quadratic} is
excluded by separation (Hahn--Banach) at fixed calibration.
\emph{(Proofs in Ch. 2: Lemma I.5, Theorem I.1.)}
\end{quote}

\hypertarget{fast-and-slow-sectors-what-is-being-selected}{%
\subsection{1.3 Fast and slow sectors (what is being
selected)}\label{fast-and-slow-sectors-what-is-being-selected}}

\begin{itemize}
\tightlist
\item
  \textbf{Fast:} quantum‑like sector on a separable Hilbert space;
  states evolve by GKSL; in the \textbf{zero‑leakage limit} the
  evolution is unitary. The throughput budget is a C*‑compatible
  quadratic of the derivation \(\delta_H(A)=i[H,A]\). KKT/Riesz in
  Chapter 3 fixes \(\hbar=\lambda_{\rm th}^{-1}\).
\item
  \textbf{Slow:} geometry and gauge scaffold selected by Γ‑limits under
  cone‑preserving, gauge‑fixed topologies; equi‑coercivity and
  Γ‑compactness proved in Chapter 4.
\end{itemize}

\hypertarget{what-follows-map}{%
\subsection{1.4 What follows (map)}\label{what-follows-map}}

\begin{enumerate}
\def\labelenumi{\arabic{enumi}.}
\tightlist
\item
  \textbf{Budgets \& pokes} minimal completeness and robustness (Ch. 2).
  2) \textbf{Fast sector normalization} and pointer alignment (Ch. 3).
  3) \textbf{Γ‑compactness} ⇒ Einstein--Hilbert scaffold (Ch. 4). 4)
  \textbf{Coupled laws} (Ch. 5). 5) \textbf{Horizons \& area law} with
  Hawking suppression (Ch. 7). 6) \textbf{RG selection} (Ch. 8). 7)
  \textbf{Quantum geometry completion} (Ch. 9). 8) \textbf{Predictions
  \& tests} (Ch. 10). 9) \textbf{Appendices}: technical proofs,
  data/estimation kits.
\end{enumerate}

\hypertarget{readers-checklist-operational-sufficiency}{%
\subsection{1.5 Reader's checklist (operational
sufficiency)}\label{readers-checklist-operational-sufficiency}}

\begin{itemize}
\tightlist
\item
  Spaces/topologies named (trace, diamond, Sobolev/Γ).
\item
  Budgets are convex, l.s.c., coercive; only three independent.
\item
  Poke cone causal/Γ‑local; envelopes are closure‑invariant.
\item
  KKT/Riesz matches gradient metric to quadratic; \(\hbar\) calibrated.
\item
  Γ‑compactness after gauge ensures slow‑sector existence.
\item
  Predictions stated with single primary KPI and low‑burden
  measurements. \# Chapter 2 --- Selection Functional, Budgets \& Pokes
  (airtight, Blocks I--III)
\end{itemize}

\begin{quote}
\textbf{Global convention.} Throughout this chapter \textbf{all}
quadratic forms, operator adjoints, and norms on blocks are taken with
respect to the \textbf{unweighted normalized Hilbert--Schmidt} (HS)
geometry

\[
\langle X,Y\rangle_{2;\Lambda}:=\frac{1}{d_\Lambda}\,\mathrm{tr}_\Lambda(X^\dagger Y),\qquad 
\|X\|_{2;\Lambda}^2=\langle X,X\rangle_{2;\Lambda}.
\]

Superoperator norms \$\textbar{}\cdot\textbar\_\{HS\to HS\}\$, cb-norms,
and all adjoints \$\{\}\^{}\{!\}\$ are computed in this geometry. This
fixes inner-product consistency across Lemmas I.1--I.3 and the budgets.
\end{quote}

\begin{center}\rule{0.5\linewidth}{0.5pt}\end{center}

\hypertarget{admissible-budgets-minimal-completeness-block-i}{%
\subsection{2.1 --- Admissible budgets \& minimal completeness (Block
I)}\label{admissible-budgets-minimal-completeness-block-i}}

\textbf{Why only these three?} On each finite block, admissible budget
functionals are support functions of convex, symmetry-invariant sets.
Ad-invariance and additivity force the \textbf{derivation-priced
throughput}, an \textbf{Ad-invariant Hilbertian complexity}, and a
\textbf{Dirichlet-type leakage} as a complete basis. Passing to the
feasible quotient and using Hahn--Banach separation at fixed calibration
excludes any \textbf{fourth independent quadratic}. Null-budget
directions are modded out; norm-equivalent representatives preserve the
multipliers.

\hypertarget{definition-i.0-admissible-budget-class-finalized}{%
\subsubsection{Definition I.0 (Admissible budget class ---
finalized)}\label{definition-i.0-admissible-budget-class-finalized}}

Let \$\omega\$ be a faithful normal state with GNS triple
\$(\pi\_\omega,\mathcal H\_\omega,\Omega\_\omega)\$ and quasi-local
C*-algebra
\$\mathcal A=\overline{\bigcup\_\Lambda \mathcal A\_\Lambda}\$ built
from finite blocks \$\Lambda\$ with matrix algebras
\$M\_\{d\_\Lambda\}\$. Equip each \$M\_\{d\_\Lambda\}\$ and
\$\mathrm{CB}(M\_\{d\_\Lambda\})\$ with their canonical operator-space
structures and the normalized HS geometry above.

A \textbf{budget} is a map \$\mathcal B:\mathcal A\to{[}0,\infty{]}\$
acting on scaffolds
\$A=(A\_\{\rm fast\},A\_\{\rm med\},A\_\{\rm leak\},A\_\{\rm slow\},A\_\{\rm disc\})\$
and satisfying:

\begin{enumerate}
\def\labelenumi{\arabic{enumi}.}
\item
  \textbf{(H1) Convexity, l.s.c., coercivity.} \$\mathcal B\$ is convex,
  lower semicontinuous for the product of blockwise HS topologies, and
  its sublevel sets intersected with the feasible class are relatively
  compact in the cone-preserving topology (Ch. 4).
\item
  \textbf{Invariance.}

  \begin{itemize}
  \item
    \emph{(Fast/med)} \$\mathcal B\$ is invariant under block-local
    unitaries \$U\$ (Ad-invariance) and relabeling symmetries.
  \item
    \emph{(Leak)} It is invariant under Kraus-index mixing by
    \$V\in U(m)\$ and \textbf{equivariant} under system unitaries:

    \[
    \mathcal B_{\rm leak}^\Lambda(\{UL_jU^\dagger\};UWU^\dagger)=\mathcal B_{\rm leak}^\Lambda(\{L_j\};W).
    \]
  \end{itemize}
\item
  \textbf{Second-order locality.} On each block \$\Lambda\$, the
  fast/mediator/leakage parts restrict to \textbf{quadratic forms}
  computed from the HS pairing above; the global budget is the
  inductive-limit supremum of block quadratics.
\item
  \textbf{Functorial ampliation.} Whenever a cb-norm (or a completely
  Hilbertian norm) appears, it is \textbf{complete}:
  \$\textbar\{\rm id\}\_k\otimes \mathcal J\textbar=\textbar{}\mathcal J\textbar\$
  for all \$k\in\mathbb N\$.
\item
  \textbf{(H0.loc) Bounded local dimension.} There exists
  \$d\_\{\rm site\}\textless{}\infty\$ and a locality radius
  \$r\in\mathbb N\$ such that any
  \$\mathcal J\in\mathsf{Loc}\emph{r(\Lambda)\$ acts nontrivially on at
  most \$C(r)\$ sites, whence its support algebra is
  \$\cong M}\{d\_\{\rm loc\}\}\$ with
  \$d\_\{\rm loc\}:=d\_\{\rm site\}\^{}\{C(r)\}\$ independent of
  \$\Lambda\$.
\item
  \textbf{(H5) Complete CP-monotonicity for leakage (sub-Markov in
  \$W\$).} For any admissible block-local CP maps
  \$\Phi\_\{\rm pre\},\Phi\_\{\rm post\}\$ whose \textbf{HS-adjoints}
  satisfy

  \[
  \Phi_{\rm pre}^{\!}(W)\le W,\qquad \Phi_{\rm post}^{\!}(W)\le W,
  \]

  one has

  \[
  \mathcal B_{\rm leak}^\Lambda(\{\Phi_{\rm post}\!\circ L_j\!\circ \Phi_{\rm pre}\};W)\ \le\ \mathcal B_{\rm leak}^\Lambda(\{L_j\};W).
  \]
\item
  \textbf{(H5.lin) Weight linearity (Dirichlet linearity postulate).}
  For all \$\alpha\ge 0\$ and positive \$W\_1,W\_2\$ with
  \$W\_1W\_2=0\$,

  \[
  \mathcal B_{\rm leak}^\Lambda(\{L_j\};\alpha W)=\alpha\,\mathcal B_{\rm leak}^\Lambda(\{L_j\};W),\qquad
  \mathcal B_{\rm leak}^\Lambda(\{L_j\};W_1+W_2)=\mathcal B_{\rm leak}^\Lambda(\{L_j\};W_1)+\mathcal B_{\rm leak}^\Lambda(\{L_j\};W_2).
  \]

  Equivalently, for the spectral resolution \$W=\sum\_a w\_a P\_a\$,

  \[
  \mathcal B_{\rm leak}^\Lambda(\{L_j\};W)=\sum_a w_a\,\mathfrak q_\Lambda\!\big(\{P_a L_j\}\big)
  \]

  for a fixed positive quadratic form \$\mathfrak q\_\Lambda\$
  independent of \$W\$ (calibration fixes its scale on rank-one
  \$P\_a\$).
\end{enumerate}

\begin{quote}
\textbf{Remark (Dirichlet linearity).} Axiom (H5.lin) is the
\textbf{Markov/Dirichlet linearity postulate} for leakage: the leakage
quadratic depends \textbf{linearly} on the pointer weight \$W\$ and is
\textbf{orthogonally additive} along its spectral decomposition. It is
satisfied by the canonical GKSL-weighted form and is equivalent, in
finite dimension, to requiring that the leakage quadratic be a
\textbf{left} Dirichlet form in \$W\$ with no weight-independent (bare)
component.
\end{quote}

Denote by \$\mathfrak B\$ the cone of budgets satisfying 1--7, and by
\$\overline{\mathcal F}\$ the feasible closure (Ch. 1.2).

\begin{center}\rule{0.5\linewidth}{0.5pt}\end{center}

\hypertarget{lemma-i.1-fastthroughput-classification-airtight}{%
\subsubsection{Lemma I.1 (Fast/throughput classification ---
airtight)}\label{lemma-i.1-fastthroughput-classification-airtight}}

On a block \$\Lambda\$, any Ad-invariant convex quadratic \$Q\_\Lambda\$
of the inner derivation \$\delta\_H\^{}\Lambda={[}H,\cdot{]}\$ (viewed
as a superoperator on
\$(M\_\{d\_\Lambda\},\langle\cdot,\cdot\rangle\_\{2;\Lambda\})\$) is a
constant multiple of

\[
\mathcal B^{\Lambda}_{\rm th}(H)=\tfrac12\,\|\delta_H^\Lambda\|_{HS\to HS}^2
=\tfrac12\,\frac1{d_\Lambda}\operatorname{Tr}_{HS}\!\big((\delta_H^\Lambda)^\dagger\delta_H^\Lambda\big).
\]

Calibrating on a two-site reference fixes the constants uniformly,
yielding the inductive-limit budget

\[
\boxed{\ \mathcal B_{\rm th}(H)=\sup_\Lambda \mathcal B^{\Lambda}_{\rm th}(H)\ }.
\]

\emph{Proof.} The adjoint representation of \$U(d\_\Lambda)\$ on
\$\mathfrak{su}(d\_\Lambda)\$ is irreducible; by Schur, any Ad-invariant
bilinear form is a scalar multiple of the Killing form. Passing to
superoperators \$\delta\_H\$ preserves Ad-covariance; the unique (up to
scale) Ad-invariant quadratic is the HS operator-norm square shown.
\$\square\$

\begin{center}\rule{0.5\linewidth}{0.5pt}\end{center}

\hypertarget{lemma-i.2-mediatorcomplexity-via-a-completely-bounded-hilbertian-norm-airtight}{%
\subsubsection{Lemma I.2 (Mediator/complexity via a completely bounded
Hilbertian norm ---
airtight)}\label{lemma-i.2-mediatorcomplexity-via-a-completely-bounded-hilbertian-norm-airtight}}

Let
\$\mathsf{Loc}\emph{r(\Lambda)\subset \mathrm{CB}(M}\{d\_\Lambda\})\$
denote block-local maps with radius \$\le r\$ (Def. I.0(5)). Suppose
\$Q\_\Lambda\$ is a convex quadratic on \$\mathsf{Loc}\emph{r(\Lambda)\$
that is (i) unitarily covariant, (ii) functorially ampliation-stable,
(iii) second-order local and cb-continuous. Then there exist constants
\$c\_1,c\_2\in(0,\infty)\$ depending \textbf{only} on
\$(r,d}\{\rm site\})\$ such that for every \$\Lambda\$ and
\$\mathcal J\in\mathsf{Loc}\_r(\Lambda)\$,

\[
\boxed{\ c_1\,\|\mathcal J\|_{cb}^2\ \le\ Q_\Lambda(\mathcal J)\ \le\ c_2\,\|\mathcal J\|_{cb}^2\ }.
\]

Equivalently,

\[
Q_\Lambda(\mathcal J)\ \simeq\ \|\mathcal J\|_{h,2}^2
:=\inf\Big\{\sum_{k}\|a_k\|_{2;\Lambda}^2\,\|b_k\|_{2;\Lambda}^2:\ \mathcal J(X)=\sum_k a_k X b_k\Big\},
\]

with equivalence constants depending only on \$(r,d\_\{\rm site\})\$.
Consequently, a block complexity budget can be taken---up to those
constants---as

\[
\boxed{\ \mathcal B^{\Lambda}_{\rm cx}:=\inf_{\mathcal L=\sum_\alpha \mathcal J_\alpha}\ \sum_\alpha \kappa_\alpha\,\|\mathcal J_\alpha\|_{cb}^2\ },
\]

where the weights \$\kappa\_\alpha\$ encode overlap counts from
locality. The value is \textbf{decomposition-independent}; the
inductive-limit budget is
\$\mathcal B\_\{\rm cx\}=\sup\_\Lambda \mathcal B\^{}\{\Lambda\}\_\{\rm cx\}\$.

\emph{Proof.} (A) \textbf{Dimension-free reduction.} By (H0.loc), each
\$\mathcal J\in\mathsf{Loc}\emph{r(\Lambda)\$ factors through
\$M}\{d\_\{\rm loc\}\}\$ with
\$d\_\{\rm loc\}=d\_\{\rm site\}\^{}\{C(r)\}\$ independent of
\$\Lambda\$. (B) \textbf{Haagerup--Hilbertian control.} On
\$M\_\{d\_\{\rm loc\}\}\$, functorial ampliation and unitary covariance
imply \$Q\_\Lambda\$ is completely Hilbertian; operator-space duality
identifies it (up to constants depending only on \$d\_\{\rm loc\}\$)
with the Haagerup-Hilbertian seminorm
\$\textbar{}\cdot\textbar{}\emph{\{h,2\}\$, cb-equivalent in fixed
finite dimension (standard Haagerup--Pisier cb≃Hilbertian equivalence on
\$M}\{d\_\{\rm loc\}\}\$). (C) \textbf{Decomposition independence.} In
finite dimension the Haagerup projective cone is closed; strong duality
equates the projective infimum with \$Q\_\Lambda\$. Pull back along the
locality factorization. \$\square\$

\begin{center}\rule{0.5\linewidth}{0.5pt}\end{center}

\hypertarget{lemma-i.3-leakage-factorization-as-a-completely-dirichlet-form-airtight}{%
\subsubsection{Lemma I.3 (Leakage factorization as a completely
Dirichlet form ---
airtight)}\label{lemma-i.3-leakage-factorization-as-a-completely-dirichlet-form-airtight}}

Let \$\mathcal L\_\{\rm leak\}\^{}\Lambda\$ assign to a Kraus list
\$\{L\_j\}\emph{\{j=1\}\^{}m\subset M}\{d\_\Lambda\}\$ a convex
quadratic satisfying: system-unitary \textbf{equivariance} in \$W\$
(Def. I.0(2)), \textbf{Kraus-mixing invariance}, (H5) \textbf{complete
CP-monotonicity} with \$\Phi\^{}\{!\}(W)\le W\$, \textbf{and} (H5.lin)
\textbf{weight linearity}. Then there exists a positive affiliated
weight \$W\succ0\$ (unique up to conjugation and equivalence on
\$\overline{\mathcal F}\$) such that

\[
\boxed{\ \mathcal B^{\Lambda}_{\rm leak}(\{L_j\};W)=\sum_{j=1}^m \|W^{1/2}L_j\|_{2;\Lambda}^2
=\sum_{j=1}^m\frac{1}{d_\Lambda}\operatorname{tr}\!\big(L_j^\dagger W L_j\big)\ .}
\]

\emph{Proof.} Work on a fixed block, drop \$\Lambda\$.

\textbf{(1) Kernel reduction.} Kraus-mixing invariance gives
\$Q(\{L\_j\})=\sum\_j \langle L\_j,,T\_W L\_j\rangle\_\{2\}\$ for some
positive linear \$T\_W\$ on \$M\_d\$. System-unitary covariance implies
\$T\_\{U W
U\textsuperscript{\dagger\}=\mathrm{Ad}\emph{U\circ T\_W\circ \mathrm{Ad}}\{U}\dagger\}\$.

\textbf{(2) Spectral diagonalization.} Let \$W=\sum\_a w\_a P\_a\$.
Using \textbf{pre/post pinching} that fix \$W\$ (so
\$(\Pi\textsuperscript{\{\rm L\})}\{!\}(W)=(\Pi\textsuperscript{\{\rm R\})}\{!\}(W)=W\$
and (H5) applies), one gets

\[
Q(L)=\sum_{a,b}Q(P_a L P_b),
\]

hence \$T\_W\$ commutes with \$L\_\{P\_a\}\$ and \$R\_\{P\_b\}\$ and
takes the form

\[
T_W=\sum_{a,b} t_{ab}(W)\,L_{P_a}R_{P_b}.
\]

\textbf{(3) Weight linearity forces left Dirichlet form.} By (H5.lin)
and orthogonal additivity, for all eigen-weights \$\{w\_a\}\$,

\[
Q(L;W)=\sum_a w_a\,\mathfrak q\!\big(\{P_a L_j\}\big).
\]

Comparing with the decomposition above and varying \$\{w\_a\}\$ yields
that \$t\_\{ab\}(W)\$ depends \textbf{only} on the left eigenvalue and
\textbf{linearly}: \$t\_\{ab\}(W)=\alpha(w\_a)\$ with \$\alpha\$
positive linear and \textbf{independent of \$b\$}. Consequently,

\[
T_W=\sum_a \alpha(w_a)\,L_{P_a}=\ L_{A(W)}\quad\text{with}\quad A(W):=\sum_a \alpha(w_a)P_a.
\]

No right term and no bare (weight-independent) term are admissible under
(H5.lin).

\textbf{(4) Identify \$A(W)=\kappa W\$.} Covariance within multiplicity
spaces and homogeneity force \$\alpha(w)=\kappa w\$; hence
\$A(W)=\kappa W\$ and

\[
Q(\{L_j\};W)=\kappa \sum_j \|W^{1/2}L_j\|_2^2.
\]

Calibrate on a two-site reference to fix \$\kappa=1\$. Uniqueness up to
conjugation/equivalence follows from faithfulness and polarization.
\$\square\$

\begin{center}\rule{0.5\linewidth}{0.5pt}\end{center}

\hypertarget{proposition-i.4-block-support-function-representation-admissible-observables-airtight}{%
\subsubsection{Proposition I.4 (Block support-function representation \&
admissible observables ---
airtight)}\label{proposition-i.4-block-support-function-representation-admissible-observables-airtight}}

For a fixed block \$\Lambda\$, set

\[
\mathcal V_\Lambda:=\mathfrak{su}(d_\Lambda)\ \oplus\ \mathsf{Loc}_r(\Lambda)\ \oplus\ (M_{d_\Lambda})^{\oplus m}
\]

with the product HS topology. Let \$\mathcal B\_\Lambda\$ be the
restriction of \$\mathcal B\in\mathfrak B\$ to \$\mathcal V\_\Lambda\$.
Then:

\begin{enumerate}
\def\labelenumi{\arabic{enumi}.}
\item
  \$\operatorname{epi}(\mathcal B\_\Lambda)\$ is a closed convex cone;
  by Fenchel--Moreau in finite dimension,

  \[
  \mathcal B_\Lambda(X)=\sup_{S\in\mathscr S_\Lambda}\ \langle S,X\rangle.
  \]
\item
  Averaging \$S\$ over the compact symmetry groups and admissible
  coarse-grains is continuous and value-preserving; the averaged
  \textbf{admissible observables} form a compact set.
\item
  By Lemmas I.1--I.3, the invariant quadratic subspace on
  \$\mathcal V\_\Lambda\$ is \textbf{three-dimensional}, generated by
  \$(\mathcal B\_\{\rm th\}\textsuperscript{\Lambda,\mathcal B\_\{\rm cx\}}\Lambda,\mathcal B\_\{\rm leak\}\^{}\Lambda)\$.
  Thus each block support functional reduces to a triple

  \[
  s=(s_{\rm th},s_{\rm cx},s_{\rm leak})\in\mathbb R_{\ge0}^3,
  \]

  modulo nulls. \$\square\$
\end{enumerate}

\begin{center}\rule{0.5\linewidth}{0.5pt}\end{center}

\hypertarget{lemma-i.5-hahnbanach-separation-three-dimensionality-on-the-feasible-quotient-airtight}{%
\subsubsection{Lemma I.5 (Hahn--Banach separation ⇒ three-dimensionality
on the feasible quotient ---
airtight)}\label{lemma-i.5-hahnbanach-separation-three-dimensionality-on-the-feasible-quotient-airtight}}

Let \$\mathcal Q\$ be the quotient of the linear span of
\$\{\mathcal B\_\Lambda:\Lambda\}\$ by the subspace of \textbf{null
budgets}
\$\{\mathcal N:\mathcal N\textbar*\{\overline{\mathcal F}\}=0\}\$. The
observable cone identified in Prop. I.4 is closed and equals
\$\operatorname{cone}{s*{\rm th},s\_{\rm cx},s\_{\rm leak}}\$. A
putative fourth independent admissible budget would be separated by a
continuous observable, contradicting Prop. I.4(3). Hence

\[
\boxed{\ \dim \mathcal Q=3\ }.
\]

\$\square\$

\begin{center}\rule{0.5\linewidth}{0.5pt}\end{center}

\hypertarget{consistency-lemma-block-constant-alignment-explicit-two-sided-bounds}{%
\subsubsection{Consistency Lemma (block-constant alignment; explicit
two-sided
bounds)}\label{consistency-lemma-block-constant-alignment-explicit-two-sided-bounds}}

If \$\Lambda\subset\Lambda'\$, locality and ampliation give, for each of
the three budgets,

\[
\mathcal B^\Lambda_{\bullet}(A|_\Lambda)\ \le\ \mathcal B^{\Lambda'}_{\bullet}(A|_{\Lambda'})\ \le\ C(r)\ \mathcal B^\Lambda_{\bullet}(A|_\Lambda),
\]

with \$C(r)\$ counting finite overlaps. Two-site calibration fixes a
common scale; the proportionality constants coincide across blocks.
Therefore the inductive-limit budgets

\[
\mathcal B_{\rm th}=\sup_\Lambda \mathcal B_{\rm th}^\Lambda,\qquad
\mathcal B_{\rm cx}=\sup_\Lambda \mathcal B_{\rm cx}^\Lambda,\qquad
\mathcal B_{\rm leak}=\sup_\Lambda \mathcal B_{\rm leak}^\Lambda
\]

are well-defined, l.s.c., and coercive with block-independent constants.
\$\square\$

\begin{center}\rule{0.5\linewidth}{0.5pt}\end{center}

\hypertarget{lemma-i.6-null-budgets-form-a-closed-subspace}{%
\subsubsection{Lemma I.6 (Null budgets form a closed
subspace)}\label{lemma-i.6-null-budgets-form-a-closed-subspace}}

Let
\$\mathsf N:=\{\mathcal N:\mathcal N\textbar\_\{\overline{\mathcal F}\}=0\}\$.
Then \$\mathsf N\$ is a \textbf{closed} linear subspace for the
inductive-limit topology.

\emph{Proof.} If \$\mathcal N\_k\to\mathcal N\$ and each vanishes on
\$\overline{\mathcal F}\$, then for any \$A\in\overline{\mathcal F}\$
and large enough \$\Lambda\$, \$A\textbar\_\Lambda\$ is feasible and
\$\mathcal N\_k(A)\to \mathcal N(A)\$ by l.s.c.; hence
\$\mathcal N(A)=0\$. \$\square\$

\begin{center}\rule{0.5\linewidth}{0.5pt}\end{center}

\hypertarget{theorem-i.1-irreducible-basis-of-admissible-budgets-airtight}{%
\subsubsection{Theorem I.1 (Irreducible basis of admissible budgets ---
airtight)}\label{theorem-i.1-irreducible-basis-of-admissible-budgets-airtight}}

Under Definition I.0 (including (H5.lin)), the Consistency Lemma, and
Lemma I.6, any \$\mathcal B\in\mathfrak B\$ admits the decomposition

\[
\boxed{\ \mathcal B=\alpha_{\rm th}\,\mathcal B_{\rm th}\;+\;\alpha_{\rm cx}\,\mathcal B_{\rm cx}\;+\;\alpha_{\rm leak}\,\mathcal B_{\rm leak}\;+\;\mathcal N,\qquad \alpha_\bullet\ge0,\ \ \mathcal N\in\mathsf N\ .}
\]

No fourth independent budget satisfying 1--7 exists.

\emph{Proof.} (i) \textbf{Blockwise representation.} Prop. I.4 expresses
\$\mathcal B\_\Lambda\$ as a support function over
\$\operatorname{cone}{s\_{\rm th},s\_{\rm cx},s\_{\rm leak}}\$, hence a
conic combination of
\$(\mathcal B\_\{\rm th\}\textsuperscript{\Lambda,\mathcal B\_\{\rm cx\}}\Lambda,\mathcal B\_\{\rm leak\}\^{}\Lambda)\$
modulo nulls. (ii) \textbf{Inductive limit.} Two-sided bounds and
equi-coercivity allow a diagonal selection so that the supremum passes
to the limit. (iii) \textbf{Exclusion of a fourth direction.} Lemma I.5
rules it out on the feasible quotient. (iv) \textbf{Null removal.} Lemma
I.6 removes null ambiguity. \$\square\$

\begin{center}\rule{0.5\linewidth}{0.5pt}\end{center}

\hypertarget{corollary-i.2-calibration-multiplier-stability-airtight}{%
\subsubsection{Corollary I.2 (Calibration \& multiplier stability ---
airtight)}\label{corollary-i.2-calibration-multiplier-stability-airtight}}

\begin{enumerate}
\def\labelenumi{(\alph{enumi})}
\item
  \textbf{Calibration.} After fixing a canonical two-site calibration,
  replacing any canonical quadratic by a blockwise
  \textbf{norm-equivalent} representative (constants depending only on
  \$(r,d\_\{\rm site\})\$) rescales the associated \$\alpha\$ by the
  fixed calibration factor.
\item
  \textbf{KKT multiplier equality under regularity.} If the optimization
  on \$\overline{\mathcal F}\$ enjoys Slater interior and \textbf{strong
  convexity} on feasible slices (so KKT multipliers are unique and
  stable), then any \textbf{calibrated} norm-equivalent replacement of a
  canonical quadratic leaves the Lagrange multipliers
  \textbf{identical}. Without strong convexity, multipliers are
  preserved \textbf{up to} the fixed calibration constants in (a).
  \$\square\$
\end{enumerate}

\begin{center}\rule{0.5\linewidth}{0.5pt}\end{center}

\hypertarget{poke-ensemble-robustness-block-ii}{%
\subsection{2.2 --- Poke ensemble robustness (Block
II)}\label{poke-ensemble-robustness-block-ii}}

\hypertarget{definition-ii.1-admissible-poke-cone}{%
\subsubsection{Definition II.1 (Admissible poke
cone)}\label{definition-ii.1-admissible-poke-cone}}

\$\mathcal P\$ is the smallest set of CPTP maps on
\$\mathfrak T\_1(\mathcal H)\$ that is: (i) \textbf{causal} (single
cone), (ii) \textbf{\$\Gamma\$-local}, (iii) closed under convex mixing
and composition, and (iv) contains neighborhoods of \$\mathrm{id}\$ and
of mixing channels at all allowed scales. Let \$\overline{\mathcal P}\$
be its \textbf{diamond-norm closure}.

\hypertarget{lemma-ii.2-operational-l.s.c.}{%
\subsubsection{Lemma II.2 (Operational
l.s.c.)}\label{lemma-ii.2-operational-l.s.c.}}

Under H2--H3, for fixed \$A\$ the map \$\Phi\mapsto\mathrm{CL}(A,\Phi)\$
is \textbf{lower semicontinuous} in the diamond norm
\$\textbar{}\cdot\textbar\_\diamond\$.

\emph{Proof (operational representation ⇒ l.s.c.).} By \textbf{H2} there
exists a directed family \$\mathscr T\$ of \textbf{finite experimental
protocols} \$T\$ (finitely many channel uses, interleaved with fixed
CPTP pre/post-processing and POVMs) and bounded continuous
post-processings \$g\_T\$ such that

\[
\mathrm{CL}(A,\Phi)=\sup_{T\in\mathscr T} F_T(A,\Phi),\qquad
F_T(A,\Phi):=g_T\!\big(p_T(A,\Phi)\big),
\]

where \$p\_T(A,\Phi)\$ is the finite outcome-probability vector
generated by \$T\$.

Fix \$A\$ and \$T\$ using at most \$N\$ calls to \$\Phi\$. A
telescoping/adaptivity bound gives

\[
\big\|\rho^{\rm out}_T(A,\Phi)-\rho^{\rm out}_T(A,\Psi)\big\|_1\ \le\ N\,\|\Phi-\Psi\|_\diamond,
\]

hence (POVM contractivity)
\$\textbar p\_T(A,\Phi)-p\_T(A,\Psi)\textbar{}\emph{1\le N\textbar{}\Phi-\Psi\textbar{}}\diamond\$.
With \$g\_T\$ continuous on the simplex, \$F\_T(A,\cdot)\$ is
\textbf{continuous} in \$\textbar{}\cdot\textbar\_\diamond\$. A
pointwise supremum of continuous functions is l.s.c.; therefore
\$\Phi\mapsto\mathrm{CL}(A,\Phi)\$ is l.s.c. \$\square\$

\hypertarget{theorem-ii.3-equivalence-of-ensembles-robustness}{%
\subsubsection{Theorem II.3 (Equivalence of ensembles /
robustness)}\label{theorem-ii.3-equivalence-of-ensembles-robustness}}

For every \$A\in\mathcal A\$,

\[
\inf_{\Phi\in \mathcal P}\mathrm{CL}(A,\Phi)=\inf_{\Phi\in \overline{\mathcal P}}\mathrm{CL}(A,\Phi).
\]

\emph{Proof.} Lower semicontinuity (Lemma II.2) plus ``infimum over a
set equals infimum over its closure.'' \$\square\$

\hypertarget{corollary-ii.4-leakage-envelope-attainment}{%
\subsubsection{Corollary II.4 (Leakage envelope
attainment)}\label{corollary-ii.4-leakage-envelope-attainment}}

With (H5), the spectral transfer envelope \$w\^{}*(\nu)\$
\textbf{attains} a minimum on \$\overline{\mathcal P}\$ (not necessarily
unique). \$\square\$

\begin{center}\rule{0.5\linewidth}{0.5pt}\end{center}

\hypertarget{selection-mechanics-large-deviations-block-iii}{%
\subsection{2.3 --- Selection mechanics \& large deviations (Block
III)}\label{selection-mechanics-large-deviations-block-iii}}

Let pokes \$\Phi\_t\stackrel{\rm i.i.d.}{\sim}\Pi\$ with support dense
in \$\overline{\mathcal P}\$. Define the risk-sensitive score
(\$\beta\textgreater0\$):

\[
\mathrm{CL}_\beta(A):=-\frac{1}{\beta}\log \mathbb E_\Pi\!\left[e^{-\beta\,\mathrm{CL}(A,\Phi)}\right].
\]

\textbf{LD hypotheses.} (LD1) Finite log-MGF on budget sublevels,
uniformly on compacta. (LD2) Coercivity (H1) ⇒ exponential tightness.
(LD3) \$A\mapsto\mathrm{CL}\_\beta(A)\$ u.s.c. on sublevels.

\hypertarget{lemma-iii.1-risk-sensitive-worst-case}{%
\subsubsection{Lemma III.1 (Risk-sensitive ⇒
worst-case)}\label{lemma-iii.1-risk-sensitive-worst-case}}

\[
\lim_{\beta\to\infty}\mathrm{CL}_\beta(A)=\inf_{\Phi\in\overline{\mathcal P}}\mathrm{CL}(A,\Phi).
\]

\emph{Proof.} Varadhan's lemma for
\$\log\mathbb E,e\^{}\{-\beta,\mathrm{CL}\}\$ yields the convex
conjugate; as \$\beta\to\infty\$, entropy regularization vanishes and
the essential infimum remains. Density plus l.s.c. (Lemma II.2) give
equality on \$\overline{\mathcal P}\$. \$\square\$

\hypertarget{theorem-iii.2-variational-growth-rate-non-teleological}{%
\subsubsection{Theorem III.2 (Variational growth rate;
non-teleological)}\label{theorem-iii.2-variational-growth-rate-non-teleological}}

Almost surely,

\[
\lim_{\beta\to\infty}\lim_{t\to\infty}\frac1t\log Z_t^{(\beta)}
= \sup_{A\in\mathcal A}\Big[\inf_{\Phi\in\overline{\mathcal P}}\mathrm{CL}(A,\Phi)\ -\ \alpha_{\rm th}B_{\rm th}(A)\ -\ \alpha_{\rm cx}B_{\rm cx}(A)\ -\ \alpha_{\rm leak}B_{\rm leak}(A)\Big].
\]

\emph{Proof.} Laplace principle under (LD2--LD3) gives
\$\lim\_\{t\to\infty\}t\textsuperscript{\{-1\}\log Z\_t}\{(\beta)\}=\sup\_A{[}\mathrm{CL}*\beta-\sum\alpha*\bullet B\_\bullet{]}\$.
Apply Lemma III.1 and epi-convergence (monotone in \$\beta\$).
\$\square\$

\begin{center}\rule{0.5\linewidth}{0.5pt}\end{center}

\hypertarget{envelope-identities-constants-as-multipliers}{%
\subsection{2.4 --- Envelope identities (constants as
multipliers)}\label{envelope-identities-constants-as-multipliers}}

Let \$\Phi(\tau)\$ be the optimal value with budget allowances
\$\tau=(\tau\_\{\rm th\},\tau\_\{\rm cx\},\tau\_\{\rm leak\})\$. Under
H1 and Slater interior, \$\Phi\$ is convex in \$\tau\$ and for a.e.
\$\tau\$,

\[
\lambda_\bullet(\tau)=\frac{\partial\Phi}{\partial\tau_\bullet}
\]

are the \textbf{Lagrange multipliers} (couplings/constants). Strict
convexity along active directions ⇒ uniqueness; else an epi-small
strictly convex regularizer from (H5) removes ties without new
parameters.

\begin{center}\rule{0.5\linewidth}{0.5pt}\end{center}

\textbf{Outcome of Blocks I--III.} With (H5.lin) added, the
admissible-budget cone is \textbf{exactly} three-dimensional up to
nulls, generated by \$(B\_\{\rm th\},B\_\{\rm cx\},B\_\{\rm leak\})\$ in
the normalized HS geometry; leakage is a \textbf{completely Dirichlet}
quadratic \$\sum\_j
\textbar W\textsuperscript{\{1/2\}L\_j\textbar\_2}2\$ (calibrated); the
poke-ensemble choice is robust to taking closures; and the selection
principle is a \textbf{worst-case} limit of risk-sensitive growth with
constants given by \textbf{envelope multipliers}. \textbf{Not
fine-tuned.} Any CL chosen from the bounded-concave proper-score family
(Appendix A1) is equivalent up to an increasing transform; budget
selection and multipliers are invariant. \textbf{Ablation note
(empirical refuter).} If we \textbf{drop} linear-response regularity
(H5.lin) or relax Ad-invariance, \textbf{mixed bimodule} terms re-enter
the admissible class and a fourth quadratic \textbf{does} separate on
the observable cone, violating Lemma I.5. This yields a concrete
falsifier: observe persistence of a fourth direction under the same
calibration and the three-budget claim fails.

\hypertarget{chapter-3-fast-sector-zero-leakage-unitary-as-throughput-dual-pointer-basis}{%
\section{\texorpdfstring{Chapter 3 --- Fast Sector: Zero-Leakage ⇒
Unitary; \$\hbar\$ as Throughput Dual; Pointer
Basis}{Chapter 3 --- Fast Sector: Zero-Leakage ⇒ Unitary; \$\$ as Throughput Dual; Pointer Basis}}\label{chapter-3-fast-sector-zero-leakage-unitary-as-throughput-dual-pointer-basis}}

\begin{quote}
\textbf{Scope.} We make the budgets C*-compatible, correct the
stationarity/dynamics bridge by \textbf{pricing motion along the orbit}
(not the superoperator norm), derive
\$\dot A=\tfrac{i}{\hbar}{[}H\^{}*,A{]}\$ from a unitary-path
variational principle (with a fully explicit variation calculus) and an
equivalent \textbf{pointwise}/Pontryagin control derivation, prove
multiplier stability under blockwise epi/Mosco limits and Ad-invariant
rescalings, and give a unitary-orbit minimizer for leakage (pointer
basis).
\end{quote}

\textbf{Bridge (what Chapter 3 actually fixes).} At KKT stationarity on
the unitary manifold with the normalized HS metric, the throughput
multiplier identifies {[}
\boxed{\ \hbar=\lambda_{\rm th}^{-1}\ }\quad\text{and}\quad \dot A=\tfrac{i}{\hbar}{[}H\^{}*,A{]}.
{]} With leakage re-enabled and the Dirichlet budget (B\_\{\rm leak\}),
GKSL generators minimize leakage by \textbf{W-alignment}, selecting the
\textbf{pointer basis} via co-diagonalization with the environmental
weight (W).

**How (\mathrm{CL}\_\{\rm Q\}) selects the pointer basis.**\\
For fixed (\rho\_0,\rho\emph{1) and environmental weight (W), minimizing
leakage (Dirichlet budget) at fixed
(\textbar{}\mathcal N}\{A,\Phi\}(\Delta)\textbar{}\emph{1) forces
co-diagonalization with (W); hence GKSL generators that \textbf{align}
with the (W)-eigenbasis are optimal. This is the \textbf{pointer basis}
selection seen in §3.4; the argument is entirely in terms of the
concrete (\mathrm{CL}}\{\rm Q\}).

\begin{center}\rule{0.5\linewidth}{0.5pt}\end{center}

\hypertarget{h0-spaces-norms-and-budgets-c-compatible}{%
\subsection{3.1 --- H0′: Spaces, norms, and budgets
(C*-compatible)}\label{h0-spaces-norms-and-budgets-c-compatible}}

\textbf{Quasi-local algebra \& GNS.} Let
\$\mathcal A=\overline{\bigcup\_\Lambda\mathcal A\_\Lambda}\$ be a
quasi-local C*-algebra with faithful state \$\omega\$ and GNS triple
\$(\pi\_\omega,\mathcal H\_\omega,\Omega\_\omega)\$. Identify
\$A\in\mathcal A\$ with
\$\pi\_\omega(A)\subset\mathcal B(\mathcal H\_\omega)\$.

\textbf{Common core \& derivation.} Fix a dense invariant core
\$\mathcal D\subset\mathcal H\_\omega\$ common to all unbounded
generators considered. For (essentially) self-adjoint \$H\$ with
\$\mathcal D\subset\mathrm{Dom}(H)\$, define on the local \$*\$-algebra
\$\mathcal A\_\{\rm loc\}\$:

\[
\delta_H(A):=i[H,A],\qquad A\in\mathcal A_{\rm loc}.
\]

Assume \$\delta\_H\$ is closable there; write its closure again as
\$\delta\_H\$.

\textbf{Normalized HS geometry on finite blocks.} On each finite
\$\Lambda\$ (matrix algebra \$M\_\{d\_\Lambda\}\$),

\[
\langle X,Y\rangle_{\rm HS;\Lambda}:=\frac{1}{d_\Lambda}\operatorname{Tr}(X^\dagger Y),\qquad 
\|X\|_{{\rm HS};\Lambda}^2=\langle X,X\rangle_{{\rm HS};\Lambda}.
\]

Adjoints of superoperators and norms are taken in this geometry.

\textbf{Throughput budget (derivation-quadratic, blockwise).} On
\$\Lambda\$, set

\[
\mathcal B_{\rm th}^\Lambda(H):=\tfrac12\,\|\delta_H\|_{{\rm der};\Lambda}^{2}
=\tfrac12\,\frac{1}{d_\Lambda}\operatorname{Tr}_{\rm HS}\!\big((\delta_H^\Lambda)^{\dagger}\delta_H^\Lambda\big),
\quad
\boxed{\ \mathcal B_{\rm th}(H):=\sup_\Lambda \mathcal B_{\rm th}^\Lambda(H)\ }.
\]

\textbf{Leakage budget.} For a Kraus list \$\{L\_j\}\$ and a positive
affiliated weight \$W\succ0\$ in the GNS von Neumann algebra,

\[
\boxed{\ \mathcal B_{\rm leak}(\{L_j\},W):=\sum_j \|W^{1/2}L_j\|_{2,\omega}^2\ =\ \sum_j\,\omega(L_j^{\dagger} W L_j)\ . }
\]

\textbf{Complexity budget (local CP decompositions).} For local CP
pieces \$\{\mathcal J\_\alpha\}\$ with cb-norms
\$\textbar{}\cdot\textbar*\{cb\}\$ and scale weights
\$\kappa*\alpha\textgreater0\$,

\[
\boxed{\ \mathcal B_{\rm cx}(\mathcal L):=\inf\Big\{\sum_\alpha \kappa_\alpha\,\|\mathcal J_\alpha\|_{cb}^2:\ \mathcal L=\sum_\alpha\mathcal J_\alpha\ \text{on}\ \mathcal A_{\rm loc}\Big\}.}
\]

\textbf{Coercivity/compactness.} Each budget is convex and l.s.c.;
sublevel sets are equi-coercive after gauge-fixing (Ch. 4). Finite-block
estimates lift by monotone convergence.

\begin{center}\rule{0.5\linewidth}{0.5pt}\end{center}

\hypertarget{exact-normalization-via-a-unitary-path-variational-principle}{%
\subsection{3.2 --- Exact normalization via a unitary-path variational
principle}\label{exact-normalization-via-a-unitary-path-variational-principle}}

We correct the stationarity→dynamics bridge by \textbf{pricing the
actual orbit speed} \$\textbar{[}H,A{]}\textbar{}\emph{\{\rm HS\}\$ (not
the Ad-invariant superoperator norm
\$\textbar{}\delta\_H\textbar{}}\{\rm der\}\$, whose directional
derivative vanishes along conjugations).

\hypertarget{a-setup}{%
\subsubsection{3.2.A --- Setup}\label{a-setup}}

\textbf{Unitary paths and kinematics.} Let \$U\_t\$ be a strongly
continuous unitary path with \$U\_0=\mathbf 1\$ and generator
\$H\_t=H\_t\^{}\dagger\$ on \$\mathcal D\$:

\[
\dot U_t=-\,iH_tU_t,\qquad A_t:=U_t^\dagger A\,U_t,\qquad \dot A_t=i[H_t,A_t].
\]

\textbf{Variations.} Admissible variations are \$\delta U\_t=-,iK\_t
U\_t\$ with \$K\_t\^{}\dagger=K\_t\$ and \$K\in C\_c\^{}1((0,T))\$; then

\[
\delta A_t=i[K_t,A_t],\qquad \delta H_t=\dot K_t+i[K_t,H_t].
\]

\textbf{Predictive score and gradient (HS metric).} Let
\$\mathcal S:\mathcal A\_\{\rm loc\}\to\mathbb R\$ be
Gateaux-differentiable along commutators in the \textbf{same} normalized
HS geometry:

\[
D\mathcal S(A)[\,i[K,A]\,]=\langle G(A),\,K\rangle_{\rm HS},\quad G(A)=G(A)^\dagger\ \text{unique}.
\]

\textbf{Action functional with orbit-throughput price.} On \${[}0,T{]}\$
consider

\[
\boxed{\ \mathfrak J[U_\cdot]:=\int_0^T\!\Big(\,\mathcal S(A_t)\ -\ \tfrac{\lambda_{\rm th}}{2}\,\|[H_t,A_t]\|_{\rm HS}^2\ \Big)\,dt\ }.
\]

Here \$\lambda\_\{\rm th\}\textgreater0\$ is the throughput multiplier;
using the \textbf{same} HS geometry will give
\$\hbar=\lambda\_\{\rm th\}\^{}\{-1\}\$.

\textbf{Orbit Laplacian.} For Hermitian \$A\$ on a finite block,

\[
\boxed{\ \mathcal A_A:=\mathrm{ad}_A^{\!*}\mathrm{ad}_A=[A,[A,\cdot]]\ \ge 0,\qquad \mathrm{ad}_A(\cdot):=[A,\cdot].\ }
\]

In an eigenbasis \$A=\sum\_k\alpha\_k P\_k\$,

\[
(\mathcal A_A X)_{ij}=(\alpha_i-\alpha_j)^2 X_{ij},\quad
\ker \mathcal A_A=\{X:[A,X]=0\},\quad \mathcal A_A^+\text{ is the Moore–Penrose pseudoinverse.}
\]

\textbf{No-go for superoperator-norm pricing.}
\$H\mapsto \textbar{}\delta\_H\textbar{}\emph{\{\{\rm der\};\Lambda\}\$
is Ad-invariant, hence the directional derivative of
\$\frac12\textbar{}\delta\_H\textbar{}}\{\{\rm der\};\Lambda\}\^{}2\$
along \$i{[}K,H{]}\$ vanishes blockwise and in the inductive limit.
Stationarity based on
\$\langle\delta\_H,\delta\_\{i{[}K,H{]}\}\rangle\_\{\rm der\}\$ cannot
produce a nontrivial \$G=\lambda\_\{\rm th\}H\$.

\begin{center}\rule{0.5\linewidth}{0.5pt}\end{center}

\hypertarget{b-theorem-3.2-heisenberg-dynamics-explicit-variational-calculus}{%
\subsubsection{\texorpdfstring{3.2.B --- Theorem 3.2′ (Heisenberg
dynamics; explicit variational calculus;
\$\hbar\$)}{3.2.B --- Theorem 3.2′ (Heisenberg dynamics; explicit variational calculus; \$\$)}}\label{b-theorem-3.2-heisenberg-dynamics-explicit-variational-calculus}}

\textbf{Statement.} Under the setup above, let \$U\_\cdot\^{}*\$ be an
interior maximizer of \$\mathfrak J\$ on \${[}0,T{]}\$ with
\$H\_t\textsuperscript{*=H}*(A\_t)\$ uniformly form-bounded on
\$\mathcal D\$ and
\$\textbar{[}H\_t\^{}*,A\_t{]}\textbar{}\emph{\{\rm HS\}\in L\^{}2\$.
Then for all \$A\in\mathcal A}\{\rm loc\}\$,

\[
\boxed{\ \dot A_t\ =\ \frac{i}{\hbar}\,[H^\*(A_t),A_t],\qquad
\hbar:=\lambda_{\rm th}^{-1}, }
\]

where \$H\^{}*(A)\$ is the \textbf{minimal-norm} solution of

\[
\boxed{\ \mathcal A_A\,H^\*(A)\ =\ \frac{-\,i}{\lambda_{\rm th}}\,\big[A,\,G(A)\big],\qquad H^\*(A)\ \perp\ \ker\mathcal A_A. }
\]

Equivalently,

\[
\boxed{\ H^\*(A)\ =\ \mathcal A_A^{+}\!\left(\frac{-\,i}{\lambda_{\rm th}}[A,G(A)]\right),\quad
\dot A=\frac{i}{\hbar}[H^\*(A),A]. }
\]

\textbf{Proof (primary --- pointwise/Pontryagin).} Consider the
optimal-control problem

\[
\max_{(A,H)}\int_0^T\!\Big(\mathcal S(A_t)-\tfrac{\lambda_{\rm th}}{2}\|[H_t,A_t]\|_{\rm HS}^2\Big)\,dt
\quad\text{s.t.}\quad \dot A_t=i[H_t,A_t].
\]

The Pontryagin Hamiltonian is

\[
\mathscr H(A,H,P)=\mathcal S(A)-\tfrac{\lambda_{\rm th}}{2}\|[H,A]\|_{\rm HS}^2+\langle P,\,i[H,A]\rangle_{\rm HS}.
\]

\begin{enumerate}
\def\labelenumi{(\roman{enumi})}
\tightlist
\item
  \textbf{Stationarity in \$H\$.} Using
  \$\langle {[}X,A{]},{[}Y,A{]}\rangle\_\{\rm HS\}=\langle X,\mathcal A\_A
  Y\rangle\_\{\rm HS\}\$ and
  \$\mathrm{ad}\_A\^{}\{!*\}=\mathrm{ad}\_A\$,
\end{enumerate}

\[
0=\partial_H\mathscr H=-\,\lambda_{\rm th}\,\mathcal A_A H\ +\ i\,\mathrm{ad}_A P
\quad\Rightarrow\quad
\boxed{\ \mathcal A_A H=\frac{i}{\lambda_{\rm th}}\mathrm{ad}_A P\ }.
\]

\begin{enumerate}
\def\labelenumi{(\roman{enumi})}
\setcounter{enumi}{1}
\tightlist
\item
  \textbf{Costate equation.} For orbit-tangent directions
  \$\delta A=i{[}K,A{]}\$,
\end{enumerate}

\[
\partial_A\mathscr H[i[K,A]]=\langle G(A),K\rangle_{\rm HS}
-\lambda_{\rm th}\,\langle [H,A],[H,i[K,A]]\rangle_{\rm HS}
+\langle P,\,i[H,i[K,A]]\rangle_{\rm HS}.
\]

Using HS-adjointness of \$\mathrm{ad}\_H\$ and Jacobi, this equals
\$\langle -,\mathrm{ad}\_A G(A)-i,\mathrm{ad}\_A\mathrm{ad}*H
P,~K\rangle*{\rm HS}\$. Hence

\[
\dot P=-\,\partial_A\mathscr H\ \text{ on }T_A\mathcal O
\quad\Longleftrightarrow\quad
\mathrm{ad}_A\big(G(A)+i\,\mathrm{ad}_H P\big)=0.
\]

Projecting onto \$\mathrm{Ran},\mathrm{ad}\_A\$ (HS-orthogonal to the
commutant) gives

\[
\boxed{\ \mathrm{ad}_A P\ =\ -\,i\,[A,G(A)]\ }.
\]

\begin{enumerate}
\def\labelenumi{(\roman{enumi})}
\setcounter{enumi}{2}
\tightlist
\item
  \textbf{Eliminate \$P\$.} Substitute into (i):
\end{enumerate}

\[
\mathcal A_A H=\frac{i}{\lambda_{\rm th}}\,\mathrm{ad}_A P
=\frac{1}{\lambda_{\rm th}}\big(-i[A,G(A)]\big).
\]

The minimal-norm solution (orthogonal to \$\ker\mathcal A\_A\$) is
\$H\textsuperscript{*(A)=\mathcal A\_A}\{+\}!\big(\tfrac{-,i}{\lambda\_{\rm th}}{[}A,G(A){]}\big)\$.
The state equation \$\dot A=i{[}H,A{]}\$ then yields the Heisenberg law
with \$\hbar=\lambda\_\{\rm th\}\^{}\{-1\}\$. \$\square\$

\textbf{Auxiliary calculus (explicit \$\delta\mathfrak J\$ on unitary
paths).} On a fixed block (finite \$d\$), let \$K\in C\_c\^{}1((0,T))\$
and use the identities, valid for Hermitian \$A,H,K\$:

\[
\begin{aligned}
&\text{(Adjointness)}\quad \langle[A,X],Y\rangle_{\rm HS}=\langle X,[A,Y]\rangle_{\rm HS},\\
&\text{(Bilinear)}\quad \langle[H,A],[Y,A]\rangle_{\rm HS}=\langle \mathcal A_A H,\,Y\rangle_{\rm HS},\\
&\text{(Time-derivative)}\quad \frac{d}{dt}\big(\mathcal A_{A_t}H_t\big)=i\,[H_t,\mathcal A_{A_t}H_t]+\mathcal A_{A_t}\dot H_t.
\end{aligned}
\]

The last identity follows from \$B\_t:=\mathrm{ad}\emph{\{A\_t\}\$,
\$\dot B\_t=\mathrm{ad}}\{\dot A\_t\}=i{[}\mathrm{ad}\emph{\{H\_t\},B\_t{]}\$
and \$\dot(B\_t\^{}2)=i{[}\mathrm{ad}}\{H\_t\},B\_t\^{}2{]}\$, hence
\$\frac{d}{dt}(B\_t\textsuperscript{2H\_t)=i{[}!H\_t,B\_t}2H\_t{]}+B\_t\^{}2\dot H\_t\$.

Now compute

\[
\delta\mathfrak J=\int_0^T\!\Big(\langle G(A),K\rangle_{\rm HS}-\lambda_{\rm th}\,\langle[H,A],[\dot K+i[K,H],A]\rangle_{\rm HS}\Big)\,dt.
\]

Use the bilinear identity twice and integrate by parts in time (boundary
vanishes by compact support of \$K\$):

\[
\delta\mathfrak J
=\int_0^T\!\Big(\langle G(A),K\rangle_{\rm HS}
+\lambda_{\rm th}\,\big\langle\tfrac{d}{dt}(\mathcal A_A H),\,K\big\rangle_{\rm HS}
-\lambda_{\rm th}\,i\,\langle \mathcal A_A H,\,[K,H]\rangle_{\rm HS}\Big)\,dt.
\]

Rewrite \$\langle \mathcal A\_A
H,{[}K,H{]}\rangle\_\{\rm HS\}=\langle{[}H,\mathcal A\_A
H{]},K\rangle\_\{\rm HS\}\$ to obtain

\[
\delta\mathfrak J=\int_0^T\!\big\langle\,G(A)+\lambda_{\rm th}\,\mathcal A_A\dot H\,,\,K\big\rangle_{\rm HS}\,dt.
\]

Since \$K\$ is arbitrary in \$C\_c\^{}1\$, stationarity of
\$\mathfrak J\$ \textbf{on unitary paths} enforces
\$\mathcal A\_A\dot H=-(1/\lambda\_\{\rm th\}),G(A)\$ on
\$\mathrm{Ran},\mathcal A\_A\$. This explicit computation shows why the
earlier shortcut to \$\langle G+\lambda\_\{\rm th\}\mathcal A\_A
H,K\rangle\_\{\rm HS\}\$ is invalid and justifies, for dynamics, the
pointwise/Pontryagin route used in the proof above.

\textbf{Normalization and units.} Because the \textbf{same} normalized
HS geometry defines (i) the predictive gradient and (ii) the
orbit-throughput quadratic, the physical Planck constant is fixed by

\[
\boxed{\ \hbar=\lambda_{\rm th}^{-1}\ }.
\]

Calibration can be done operationally (e.g., two-level Fubini--Study
speed) without ambiguity (§3.2.C).

\begin{center}\rule{0.5\linewidth}{0.5pt}\end{center}

\hypertarget{c-multiplier-stability-metric-uniqueness}{%
\subsubsection{3.2.C --- Multiplier stability \& metric
uniqueness}\label{c-multiplier-stability-metric-uniqueness}}

\textbf{Epi/Mosco stability across blocks.} With
\$\mathcal B\_\{\{\rm orb\};\Lambda\}(A,H):=\tfrac12\textbar{[}H,A{]}\textbar{}\emph{\{\{\rm HS\};\Lambda\}\^{}2\$
and
\$\mathcal B}\{\rm orb\}:=\sup\_\Lambda\mathcal B\_\{\{\rm orb\};\Lambda\}\$,
the value functions

\[
\Phi_\Lambda:=\sup_{U_\cdot}\int_0^T\!\Big(\mathcal S(A_t)-\lambda_{\rm th}\,\mathcal B_{{\rm orb};\Lambda}(A_t,H_t)\Big)\,dt
\]

epi-converge (Mosco) to \$\Phi\$ defined with
\$\mathcal B\_\{\rm orb\}\$. At points of differentiability, multipliers
converge: \$\lambda\_\{\{\rm th\},\Lambda\}\to\lambda\_\{\rm th\}\$.
Hence \$\hbar=\lambda\_\{\rm th\}\^{}\{-1\}\$ is \textbf{independent} of
the exhausting sequence.

\textbf{Ad-invariant metric uniqueness (up to scale).} On each finite
block, any Ad-invariant inner product on observables is a scalar
multiple \$\alpha\$ of normalized HS (Schur). Replacing
\$\langle\cdot,\cdot\rangle\_\{\rm HS\}\$ by
\$\alpha\langle\cdot,\cdot\rangle\_\{\rm HS\}\$ rescales
\$G\mapsto \alpha\^{}\{-1\}G\$ and
\$\textbar{[}H,A{]}\textbar{}\textsuperscript{2\mapsto \alpha\textbar{[}H,A{]}\textbar{}}2\$.
The product \$\alpha,\lambda\_\{\rm th\}\^{}\{-1\}\$ --- i.e., \$\hbar\$
--- is \textbf{invariant} under this rescaling. Thus \$\hbar\$ is
well-defined (after a single operational calibration).

\begin{center}\rule{0.5\linewidth}{0.5pt}\end{center}

\hypertarget{hypotheses-u-unbounded-generators-gksl-on-a-core}{%
\subsection{3.3 --- Hypotheses U (unbounded generators; GKSL on a
core)}\label{hypotheses-u-unbounded-generators-gksl-on-a-core}}

\begin{itemize}
\item
  \textbf{U1 (Core \& closability).} There exists a common invariant
  core \$\mathcal D\$ for \$H\$ and all \$L\_j\$; \$\delta\_H\$ is
  closable on \$\mathcal A\_\{\rm loc\}\$.
\item
  \textbf{U2 (Form bounds).} There exists \$N\ge0\$ (number-type) and
  constants \$a\textless1\$, \$b\textless{}\infty\$ with, for all
  \$\psi\in\mathcal D\$,

  \[
  \|H\psi\|^2+\sum_j\|L_j\psi\|^2\ \le\ a\,\|N\psi\|^2+b\,\|\psi\|^2.
  \]
\item
  \textbf{U3 (Quasi-locality).} Interactions have finite range and
  uniformly bounded overlap across \$\Lambda\$.
\item
  \textbf{U4 (Semigroup).} The GKSL closure generates a unique strongly
  continuous CPTP semigroup with Lieb--Robinson-type bounds.
\item
  \textbf{U5 (Lyapunov drift).} There exist \$c\_0,c\_1\textgreater0\$
  with \$\mathcal L\^{}*(N)\le c\_0-c\_1 N\$ on \$\mathcal D\$.
\end{itemize}

\textbf{Drift ⇒ leakage finiteness.} For \$W=(1+N)\^{}\{-s\}\$ with
\$s\textgreater1/2\$,

\[
\sum_j\omega(L_j^\dagger W L_j)<\infty\quad\Rightarrow\quad \mathcal B_{\rm leak}(\{L_j\},W)<\infty,
\]

by Cauchy--Schwarz in the \$\omega\$-HS norm and U5.

\begin{center}\rule{0.5\linewidth}{0.5pt}\end{center}

\hypertarget{zero-leakage-unitary-gksl}{%
\subsection{3.4 --- Zero-leakage ⇒ unitary
GKSL}\label{zero-leakage-unitary-gksl}}

Let \$\mathcal L=i{[}H,\cdot{]}+\sum\_j
L\_j\^{}\dagger(\cdot)L\_j-\tfrac12{L\_j^\dagger L\_j,\cdot}\$ satisfy
U1--U4, and let \$W\succ0\$.

\textbf{Proposition 3.4.1.} If
\$\mathcal B\_\{\rm leak\}(\{L\_j\},W)=\sum\_j\omega(L\_j\^{}\dagger W
L\_j)=0\$, then \$L\_j=0\$ for all \$j\$ and the semigroup is unitary
with generator \$H\$ on the core.

\emph{Proof.} Each term \$\omega(L\_j\^{}\dagger W
L\_j)=\textbar W\textsuperscript{\{1/2\}L\_j\textbar\_\{2,\omega\}}2\ge0\$.
If the sum vanishes, \$W\^{}\{1/2\}L\_j=0\$; since \$W\succ0\$,
\$L\_j=0\$. \$\square\$

Thus, in the \textbf{zero-leakage limit}, the fast sector is purely
unitary, governed by \$H\$ from §3.2.

\begin{center}\rule{0.5\linewidth}{0.5pt}\end{center}

\hypertarget{pointer-alignment-unitary-orbit-minimizer}{%
\subsection{3.5 --- Pointer alignment (unitary-orbit
minimizer)}\label{pointer-alignment-unitary-orbit-minimizer}}

Fix \$W\succ0\$. Consider a noise block with fixed singular values
(fixed ``strength spectrum''). For \$C\$ in this block define

\[
\mathcal L(C):=\operatorname{Tr}\!\big(W^{1/2}C^\dagger C\,W^{1/2}\big).
\]

\textbf{Theorem 3.5.1 (unitary-orbit rearrangement).} Over the orbit
\$\{UCV:~U,V~\text{unitary}\}\$,

\[
\boxed{\ \mathcal L(UCV)\ \text{is minimized iff}\ [C^\dagger C,\,W]=0.\ }
\]

Equivalently, minimizers \textbf{co-diagonalize} \$C\^{}\dagger C\$ and
\$W\$, selecting the pointer basis (degeneracies handled blockwise).

\emph{Proof.}
\$\mathcal L(UCV)=\operatorname{Tr}(W\textsuperscript{\{1/2\}V}\dagger C\^{}\dagger C
V W\^{}\{1/2\})\$. Let \$W=\sum\_k w\_k Q\_k\$ with
\$w\_1\ge\cdots\ge w\_d\textgreater0\$. By von Neumann/Schur--Horn, with
fixed eigenvalues \$\{\sigma\_\ell\}\$ of \$C\^{}\dagger C\$, the
minimum of \$\operatorname{Tr}(W,V\textsuperscript{\dagger C}\dagger C
V)\$ is attained when \$V\$ aligns eigenvectors so that
\$C\^{}\dagger C\$ is diagonal in the eigenbasis of \$W\$. Equality
enforces \${[}C\^{}\dagger C,,W{]}=0\$. \$\square\$

\begin{center}\rule{0.5\linewidth}{0.5pt}\end{center}

\hypertarget{summary}{%
\subsection{3.6 --- Summary}\label{summary}}

\begin{itemize}
\item
  Pricing the \textbf{orbit speed}
  \$\textbar{[}H,A{]}\textbar\_\{\rm HS\}\$ (rather than the
  Ad-invariant superoperator norm) and matching the \textbf{same}
  normalized HS geometry for gradient and quadratic yields

  \[
  \dot A=\frac{i}{\hbar}[H^\*(A),A],\qquad
  H^\*(A)=\mathcal A_A^{+}\!\left(\frac{-\,i}{\lambda_{\rm th}}[A,G(A)]\right),\quad \hbar=\lambda_{\rm th}^{-1}.
  \]

  Here \$\mathcal A\_A={[}A,{[}A,\cdot{]}{]}\ge0\$ and
  \$\mathcal A\_A\^{}+\$ projects off the commutant.
\item
  \textbf{Stability.} Epi/Mosco limits across blocks preserve
  \$\lambda\_\{\rm th\}\$; Ad-invariant metric rescalings cancel in
  \$\hbar\$. A single operational calibration fixes \$\hbar\$.
\item
  \textbf{Fast sector.} Zero leakage forces GKSL to reduce to a unitary
  group. For nonzero leakage, \textbf{pointer alignment} minimizes
  \$\operatorname{Tr}(W\textsuperscript{\{1/2\}C}\dagger C
  W\^{}\{1/2\})\$ by co-diagonalizing \$C\^{}\dagger C\$ with \$W\$.
\end{itemize}

This completes the fast-sector normalization and resolves the
stationarity/dynamics gap with explicit variational calculus and a
pointwise/Pontryagin control derivation in a C*-compatible,
mathematically airtight manner.

\hypertarget{chapter-4-scaffold-selection-ux3b3limit-einsteinhilbert-compactness}{%
\section{Chapter 4 --- Scaffold Selection: Γ‑Limit ⇒ Einstein--Hilbert
(+
compactness)}\label{chapter-4-scaffold-selection-ux3b3limit-einsteinhilbert-compactness}}

\begin{quote}
\textbf{Scope.} We state a clean cone‑preserving topology after
gauge‑fixing and prove equi‑coercivity and Γ‑compactness. This slots
into the EH Γ‑limit derivation and enforces single‑cone
microhyperbolicity.
\end{quote}

\hypertarget{ux3b3compactness-in-a-conepreserving-class-theorem-4.1}{%
\subsection{4.1 Γ‑compactness in a cone‑preserving class (Theorem
4.1′)}\label{ux3b3compactness-in-a-conepreserving-class-theorem-4.1}}

\textbf{Setup.} Bounded‑geometry manifold: uniform injectivity radius
and bounded curvature (all derivatives). Fix de Donder gauge inside the
\textbf{single‑cone} class; use harmonic coordinates on charts.

\textbf{Cone‑preserving topology.} \(g\in H^2_{\rm loc}\) and for
reference metrics \(g_\*,g^\*\) on compacts,
\(g_\*\ \le\ g\ \le\ g^\*\quad\text{a.e. in harmonic coordinates (cone inclusion).}\)

\textbf{Uniform symbol bounds.} There exist \(0<\lambda\le\Lambda\) such
that for all admissible \(g\) and \(\xi\),
\(\lambda\,|\xi|^4\ \le\ \sigma(\mathbb A_\varepsilon(g))[\xi]\ \le\ \Lambda\,|\xi|^4\qquad\text{(parameter‑ellipticity on compacts).}\)

\textbf{Gårding inequality (uniform).} There exist \(c_1,c_2,c_3>0\),
independent of \(\varepsilon\) and \(g\), with
\(\boxed{\ \mathcal F_\varepsilon(g)\ \ge\ c_1\|\nabla^2 g\|_{L^2}^2\ -\ c_2\|g\|_{H^1}^2\ -\ c_3\ }.\)

\textbf{Boundary/at‑infinity control.} If \(M\) is noncompact, either
(i) impose uniform equivalence to a reference \(g_\infty\) outside a
compact set, or (ii) include in \(V_\varepsilon\) a cutoff penalizing
deviations at infinity to control \(\|g\|_{H^1}\).

\hypertarget{theorem-4.1-equicoercivity-and-ux3b3limit}{%
\subsubsection{Theorem 4.1′ (equi‑coercivity and
Γ‑limit)}\label{theorem-4.1-equicoercivity-and-ux3b3limit}}

The family \(\{\mathcal F_\varepsilon\}\) is equi‑coercive in weak
\(H^2\); any weak \(H^2\) cluster point of minimizers is a minimizer of
\(\mathcal F_0\) (Γ‑liminf and recovery hold).

\emph{Proof sketch.} Gauge‑fixing removes diffeomorphism degeneracy;
bounded geometry yields Rellich compactness in weak \(H^2\); single‑cone
stability controls the principal symbol. Γ‑compactness follows from
equi‑coercivity and lower semicontinuity; recoveries are built by
mollification in harmonic charts and partition‑of‑unity gluing.

\hypertarget{chapter-5-coupled-laws-einsteinym-gksl-via-joint-stationarity}{%
\section{Chapter 5 --- Coupled Laws (Einstein--YM + GKSL via Joint
Stationarity)}\label{chapter-5-coupled-laws-einsteinym-gksl-via-joint-stationarity}}

\begin{quote}
\textbf{Scope.} We prove the coupled slow--fast laws at joint
stationarity of the coherence program. The argument rests on:

\begin{enumerate}
\def\labelenumi{\arabic{enumi}.}
\tightlist
\item
  a \textbf{directional envelope theorem} for worst-case pokes with
  \textbf{direction-selecting minimizers} (no illicit inf--variation
  swap);
\item
  effective sources \(T^{\rm eff},J^{\rm eff}\) defined first in the
  \textbf{Clarke} framework (existence + conservation \textbf{without}
  linearity), then \textbf{upgraded} to unique tensors under an explicit
  Gateaux/Clarke condition;
\item
  \textbf{first-variation convergence} for the slow Γ-limit via
  localized \textbf{Mosco/Attouch};
\item
  a fast-sector control law that prices the \textbf{derivation}
  \([H,\cdot]\) (the actual throughput), uses the \textbf{Moore--Penrose
  pseudoinverse} of the orbit Laplacian
  \(\mathcal A_A=\mathrm{ad}_A^{\,*}\mathrm{ad}_A\), and yields the
  \textbf{Heisenberg law} with a consistent normalization
  \(\hbar=\lambda_{\rm th}^{-1}\).
\end{enumerate}
\end{quote}

\begin{center}\rule{0.5\linewidth}{0.5pt}\end{center}

\hypertarget{admissible-variables-and-objective}{%
\subsection{5.1 --- Admissible variables and
objective}\label{admissible-variables-and-objective}}

\textbf{Slow variables.} Lorentzian metrics \(g\) of bounded geometry in
the \textbf{single-cone} class (weak \(H^2_{\rm loc}\), de Donder gauge
on compacta) and compact-group connections \(A\) with curvature
\(F\in L^2_{\rm loc}\).

\textbf{Fast variables.} A GKSL generator on the quasi-local
C\(^*\)-algebra (Ch. 3):

\[
\mathcal L(\rho)=-i[H,\rho]+\sum_j L_j\rho L_j^\dagger-\tfrac12\{L_j^\dagger L_j,\rho\},
\]

with budgets
\(\mathcal B_{\rm th},\mathcal B_{\rm cx},\mathcal B_{\rm leak}\) as in
Appendix A and Hypotheses U.

\textbf{Pokes.} \(\overline{\mathcal P}\): diamond-norm closure of the
causal, Γ-local cone (§2.2).

\textbf{Slow action (Γ-limit).} Assumption \textbf{D\(^\star\)}: a
family \(\{\mathcal F_\varepsilon(g,A)\}\) Γ-converges on the cone-class
to

\[
\mathcal F_0(g,A)=\frac{1}{16\pi G}\!\int_M (R-2\Lambda)\sqrt{|g|}\,d^dx\;+\;\frac{1}{2g_{\rm YM}^2}\!\int_M \mathrm{tr}(F\wedge\!*F),
\]

with equi-coercivity and boundary control on compacta.

\textbf{Objective.}

\[
\boxed{
\mathcal J(g,A,H,\{L_j\})\ :=\ V(g,A)\ -\ \lambda_{\rm th}\mathcal B_{\rm th}(H)\ -\ \lambda_{\rm cx}\mathcal B_{\rm cx}(\mathcal L)\ -\ \lambda_{\rm leak}\mathcal B_{\rm leak}(\{L_j\})\ -\ \mathcal F_\varepsilon(g,A)
}
\]

where
\(V(g,A):=\inf_{\Phi\in\overline{\mathcal P}}\mathrm{CL}(g,A;\Phi)\).
Slater interior holds (Appendix E).

\begin{center}\rule{0.5\linewidth}{0.5pt}\end{center}

\hypertarget{parametric-envelope-for-worst-case-pokes-directional-form}{%
\subsection{5.2 --- Parametric envelope for worst-case pokes
(directional
form)}\label{parametric-envelope-for-worst-case-pokes-directional-form}}

We work with \textbf{directional} derivatives and avoid over-claiming
Gateaux differentiability.

\hypertarget{hypotheses-e-attainment-regularity-direction-wise-selection}{%
\subsubsection{Hypotheses E′ (attainment, regularity, direction-wise
selection)}\label{hypotheses-e-attainment-regularity-direction-wise-selection}}

\begin{itemize}
\item
  \textbf{E1 (attainment/compactness).} For each \((g,A)\),
  \(\Phi\mapsto \mathrm{CL}(g,A;\Phi)\) is l.s.c. and
  \textbf{inf-compact} on \(\overline{\mathcal P}\); hence
  \(\mathsf M(g,A):=\operatorname{Argmin}_{\Phi\in\overline{\mathcal P}}\mathrm{CL}(g,A;\Phi)\)
  is non-empty and compact.
\item
  \textbf{E2 (Carathéodory).}
  \((g,A,\Phi)\mapsto \mathrm{CL}(g,A;\Phi)\) is continuous in \((g,A)\)
  for fixed \(\Phi\), and l.s.c. in \(\Phi\) for fixed \((g,A)\).
\item
  \textbf{E3 (directional differentiability).} For each \(\Phi\) and
  cone-admissible variation \(\delta=(\delta g,\delta A)\) with compact
  support,

  \[
  D\,\mathrm{CL}(g,A;\Phi;\delta):=\lim_{t\downarrow0}\frac{\mathrm{CL}(g+t\delta g,A+t\delta A;\Phi)-\mathrm{CL}(g,A;\Phi)}{t}
  \]

  exists and is positively homogeneous in \(\delta\).
\item
  \textbf{E4 (equi-differentiability on argmin graphs).} There exist
  \(M>0\) and a neighborhood \(U\ni(g,A)\) such that
  \(|D\,\mathrm{CL}(g',A';\Phi;\delta)|\le M\|\delta\|\) for all
  \((g',A')\in U\), \(\Phi\in\mathsf M(g',A')\), and admissible
  \(\delta\).
\item
  \textbf{E5 (direction-wise selection).} For each \((g,A)\) and
  direction \(\delta\), choose

  \[
  \Phi^*(g,A;\delta)\in\arg\min_{\Phi\in\mathsf M(g,A)} D\,\mathrm{CL}(g,A;\Phi;\delta)
  \]

  (measurable selection exists by Kuratowski--Ryll-Nardzewski).
\end{itemize}

\hypertarget{theorem-5.2-directional-envelope}{%
\subsubsection{Theorem 5.2′ (Directional
envelope)}\label{theorem-5.2-directional-envelope}}

Under E1--E5, for every cone-admissible \(\delta\),

\[
\boxed{\
D V(g,A;\delta)\ =\ \min_{\Phi\in\mathsf M(g,A)} D\,\mathrm{CL}(g,A;\Phi;\delta)\ =\ D\,\mathrm{CL}\big(g,A;\Phi^*(g,A;\delta);\delta\big).
\ }
\]

\hypertarget{corollary-5.2b-gateaux-differentiability-two-routes}{%
\subsubsection{Corollary 5.2b (Gateaux differentiability --- two
routes)}\label{corollary-5.2b-gateaux-differentiability-two-routes}}

At \((g,A)\), \(V\) is Gateaux differentiable (two-sided, linear in
\(\delta\)) if \textbf{either}:

\begin{enumerate}
\def\labelenumi{\arabic{enumi}.}
\tightlist
\item
  (\textbf{Unique active}). A strictly convex, cone-local tie-breaker
  (vanishing in the Γ-limit) makes \(\mathsf M(g,A)\) a singleton with
  locally Lipschitz dependence; then
  \(D V(g,A;\delta)=D\,\mathrm{CL}(g,A;\Phi^*(g,A);\delta)\).
\item
  (\textbf{Clarke-regular active set}). \(\mathsf M(g,A)\) is compact
  and \(D\,\mathrm{CL}(g,A;\Phi;\delta)\) is \textbf{constant} over
  \(\Phi\in\mathsf M(g,A)\) for each \(\delta\). Then right/left
  derivatives coincide and are linear.
\end{enumerate}

\begin{center}\rule{0.5\linewidth}{0.5pt}\end{center}

\hypertarget{effective-stress-tensor-and-current-clarke-framework-upgrade}{%
\subsection{5.3 --- Effective stress tensor and current (Clarke
framework →
upgrade)}\label{effective-stress-tensor-and-current-clarke-framework-upgrade}}

We first ensure \textbf{existence and conservation} of sources
\textbf{without} assuming linearity (Clarke framework), then
\textbf{upgrade} to unique tensors under Cor. 5.2b.

\hypertarget{assumption-l0-local-lipschitz-of-the-envelope}{%
\subsubsection{Assumption L0 (local Lipschitz of the
envelope)}\label{assumption-l0-local-lipschitz-of-the-envelope}}

Under E1--E4, the value function
\(V(g,A)=\min_{\Phi\in\mathsf M(g,A)}\mathrm{CL}(g,A;\Phi)\) is
\textbf{locally Lipschitz} on the cone-class (Danskin--Rockafellar with
the uniform bound in E4).

\textbf{Localization for Clarke calculus.} All Clarke
sub/super-differential objects are taken on bounded subdomains
(\Omega\Subset M) with compactly supported variations; conclusions pass
to (M) by exhaustion (\Omega\_n\uparrow M) and compatibility of the
cone-class cutoffs. This keeps the Banach-space hypotheses crisp and
avoids overreach beyond bounded domains.

\textbf{Clarke subdifferentials.} Let
\(\partial_g^{\mathrm C} V(g,A)\subset H^{-2}_{\rm loc}\) and
\(\partial_A^{\mathrm C} V(g,A)\subset H^{-1}_{\rm loc}\) denote the
\textbf{Clarke} subdifferentials; \(V_g^\circ(g,A;\delta g)\),
\(V_A^\circ(g,A;\delta A)\) are the Clarke directional derivatives.

\hypertarget{subgradient-sources-always-well-posed-and-conservation}{%
\subsubsection{5.3.1 Subgradient sources (always well-posed) and
conservation}\label{subgradient-sources-always-well-posed-and-conservation}}

Take any \(T^{\rm eff}\in\partial_g^{\mathrm C}V(g,A)\),
\(J^{\rm eff}\in\partial_A^{\mathrm C}V(g,A)\) so that

\[
\langle T^{\rm eff},\delta g\rangle\ \le\ V_g^\circ(g,A;\delta g),\qquad
\langle J^{\rm eff},\delta A\rangle\ \le\ V_A^\circ(g,A;\delta A).
\]

If \(\mathrm{CL}\) is diffeomorphism- and gauge-covariant on the
cone-class, then for compactly supported \(X,\chi\),

\[
V_g^\circ(g,A;\mathcal L_X g)=V_g^\circ(g,A;-\mathcal L_X g)=0,\qquad
V_A^\circ(g,A;D\chi)=V_A^\circ(g,A;-D\chi)=0,
\]

hence

\[
\langle T^{\rm eff},\mathcal L_X g\rangle=0,\qquad \langle J^{\rm eff},D\chi\rangle=0.
\]

Integrating by parts within the cone-class yields the weak conservation
laws

\[
\nabla^\mu T^{\rm eff}_{\mu\nu}=0,\qquad D^\mu J^{\rm eff}_\mu=0
\]

\textbf{for every Clarke-selection} \(T^{\rm eff},J^{\rm eff}\).

\hypertarget{upgrade-to-single-tensors-distributional-representation-riesz-on-bounded-domains-symmetry}{%
\subsubsection{5.3.2 Upgrade to single tensors (Distributional
representation (Riesz on bounded domains);
symmetry)}\label{upgrade-to-single-tensors-distributional-representation-riesz-on-bounded-domains-symmetry}}

\textbf{Assumption L (linearity at \((g,A)\)).} One of Cor. 5.2b's
conditions holds at \((g,A)\). Then \(V\) is Gateaux differentiable in
\(g,A\); the Clarke subdifferentials are singletons and coincide with
the Fréchet derivatives.

\textbf{Corollary.} Under Assumption L there exist unique distributions
\(T^{\rm eff}\in H^{-2}_{\rm loc}\), \(J^{\rm eff}\in H^{-1}_{\rm loc}\)
such that on each bounded domain \(\Omega\Subset M\)

\[
\begin{aligned}
D_g V(g,A)[\delta g] &= \tfrac12\int_\Omega T^{\rm eff}_{\mu\nu}\,\delta g^{\mu\nu}\,\sqrt{|g|}\,d^dx,\\
D_A V(g,A)[\delta A] &= \int_\Omega \langle J^{\rm eff}_\mu,\,\delta A^\mu\rangle\,\sqrt{|g|}\,d^dx,
\end{aligned}
\]

with \(T^{\rm eff}_{\mu\nu}=T^{\rm eff}_{\nu\mu}\); the representations
are compatible under exhaustion and define \(T^{\rm eff},J^{\rm eff}\)
globally on \(M\). Conservation from §5.3.1 persists.

\begin{center}\rule{0.5\linewidth}{0.5pt}\end{center}

\hypertarget{slow-sector-eulerlagrange-eh-ym-first-variation-convergence}{%
\subsection{5.4 --- Slow-sector Euler--Lagrange: EH + YM
(first-variation
convergence)}\label{slow-sector-eulerlagrange-eh-ym-first-variation-convergence}}

Γ-convergence alone does not yield convergence of first variations. We
adopt a localized Mosco/Attouch scheme.

\textbf{Assumption M (localized Mosco/Attouch).} On each bounded
\(\Omega\Subset M\), \(\mathcal F_\varepsilon|_\Omega\) are
equi-coercive, l.s.c., and admit integral representations with
Carathéodory integrands \(f_\varepsilon(x,\cdot)\) obeying uniform
growth/ellipticity and \(f_\varepsilon\to f_0\) in \(L^\infty(\Omega)\).
Then
\(\partial \mathcal F_\varepsilon|_\Omega \xrightarrow{\rm graph} \partial \mathcal F_0|_\Omega\)
(Attouch). Exhaust \(\Omega_n\uparrow M\).

\textbf{Lemma 5.4.1 (convergence of first variations).} For any
admissible \((g,A)\) and compactly supported cone-preserving
\((\delta g,\delta A)\),

\[
\lim_{\varepsilon\to0} D\,\mathcal F_\varepsilon(g,A)[\delta g,\delta A]\ =\ D\,\mathcal F_0(g,A)[\delta g,\delta A],
\]

with

\[
D_g \mathcal F_0(g,A)[\delta g]=\tfrac12\!\int \!\tfrac{1}{8\pi G}\big(G_{\mu\nu}+\Lambda g_{\mu\nu}\big)\,\delta g^{\mu\nu}\sqrt{|g|},\quad
D_A \mathcal F_0(g,A)[\delta A]=\int\!\langle D^\alpha F_{\alpha\mu},\delta A^\mu\rangle\sqrt{|g|}.
\]

\textbf{Theorem 5.4.2 (Einstein--YM with operational sources).} Let
\((g,A,H,\{L_j\})\) be a cone-class KKT point of \(\mathcal J\) and
assume \textbf{Assumption L} at \((g,A)\). Then

\[
\boxed{
G_{\mu\nu}+\Lambda g_{\mu\nu}=8\pi G\,T^{\rm eff}_{\mu\nu},\qquad
D^\alpha F_{\alpha\beta}=J^{\rm eff}_\beta,
}
\]

with \(\nabla^\mu T^{\rm eff}_{\mu\nu}=0\) and
\(D^\mu J^{\rm eff}_\mu=0\).

\textbf{Proposition 5.4.3 (constants as multipliers = Γ-calibration).}
Under Slater and strict convexity of \(\mathcal F_0\) along the
cone-class, KKT multipliers \((G^{-1},\Lambda,g_{\rm YM}^{-2})\) are
\textbf{unique}. Calibrating \(\mathcal F_\varepsilon\) on (i) Minkowski
and (ii) constant-curvature backgrounds fixes the same constants; the
Γ-calibrated constants coincide with the KKT duals.

\begin{center}\rule{0.5\linewidth}{0.5pt}\end{center}

\hypertarget{fast-sector-stationarity-and-microcausal-hygiene}{%
\subsection{5.5 --- Fast-sector stationarity and microcausal
hygiene}\label{fast-sector-stationarity-and-microcausal-hygiene}}

We price the \textbf{derivation} (the actual throughput) and solve the
first-order condition on the \textbf{unitary-orbit tangent} using the
\textbf{Moore--Penrose pseudoinverse} of the orbit Laplacian.

\hypertarget{unitary-orbit-control-with-derivation-quadratic-heisenberg-law}{%
\subsubsection{5.5.1 Unitary-orbit control with derivation-quadratic
(Heisenberg
law)}\label{unitary-orbit-control-with-derivation-quadratic-heisenberg-law}}

Work on a finite block \(\Lambda\) with normalized Hilbert--Schmidt (HS)
inner product
\(\langle X,Y\rangle_{\rm HS}:=\mathrm{Tr}_\Lambda(X^\dagger Y)/d_\Lambda\).
Let \(A_t=U_t^\dagger A_0U_t\), \(\dot U_t=-\tfrac{i}{\hbar}H_tU_t\), so

\[
\boxed{\ \dot A_t=\tfrac{i}{\hbar}[H_t,A_t]\ }.
\]

Let \(\mathsf P(A)\) be the orbit-restricted predictive score;
\(G(A):=\operatorname{grad}_{\rm HS}\mathsf P(A)\). \textbf{Key pairing
(explicit HS calculation).} Work on a finite block (\Lambda) with
normalized Hilbert--Schmidt inner product
(\langle X,Y\rangle\emph{\{\rm HS\}:=\mathrm{Tr}(X\^{}\dagger Y)/d}\Lambda).
For Hermitian (A,H), {[}

\begin{aligned}
\langle [A,X],Y\rangle_{\rm HS}
&= \frac{1}{d_\Lambda}\mathrm{Tr}\big((AX-XA)^\dagger Y\big) \\
&= \frac{1}{d_\Lambda}\mathrm{Tr}\big(X^\dagger A Y - A X^\dagger Y\big) \\
&= \frac{1}{d_\Lambda}\mathrm{Tr}\big(X^\dagger A Y - X^\dagger Y A\big) \quad(\text{cyclicity})\\
&= \langle X,[A,Y]\rangle_{\rm HS}.
\end{aligned}

{]} Thus the commutator superoperator (\mathrm{ad}\emph{A:X\mapsto[A,X])
is \textbf{HS-self-adjoint} on Hermitian (A). Along a unitary orbit
(A\_t=U\_t\^{}\dagger A\_0 U\_t) with
(\dot U\_t=-\tfrac{i}{\hbar}H\_tU\_t), {[}
\dot A\_t=\tfrac{i}{\hbar}{[}H\_t,A\_t{]}. {]} Let (\mathsf P(A)) be the
orbit-restricted predictive score and
(G(A):=\operatorname{grad}}\{\rm HS\}\mathsf P(A)). For any Hermitian
control (H), {[}
\frac{d}{dt}\Big\textbar{}\emph{\{t=0\}\mathsf P(A\_t)=\Big\langle G(A),\tfrac{i}{\hbar}{[}H,A{]}\Big\rangle}\{\rm HS\}
=\Big\langle -\tfrac{i}{\hbar}{[}G(A),A{]},,H\Big\rangle\_\{\rm HS\}.
{]} Hence the \textbf{steepest-ascent} Hamiltonian (Riesz/KKT on the
unitary manifold) is {[}
\boxed{\ H^*(A)\ \propto\ -\tfrac{i}{\hbar}[G(A),A]\ }, {]} and the
corresponding generator is (\dot A=\tfrac{i}{\hbar}{[}H\^{}*(A),A{]}).
Local Lipschitz and blockwise bounds propagate to the quasi-local
algebra via the uniform HS-block controls of Ch. 3.

\hypertarget{leakage-and-gksl}{%
\subsubsection{5.5.2 Leakage and GKSL}\label{leakage-and-gksl}}

Allow Lindblad controls \(\{L_j\}\) with leakage price
\(\lambda_{\rm leak}\sum_j\|W^{1/2}L_j\|_{\rm HS}^2\) (App. A). The
convex optimization over CP-tangent directions yields an optimal
\textbf{GKSL generator} with leakage-penalized pointer alignment as in
Ch. 3. Hypotheses U give well-posedness and Lieb--Robinson-type bounds
on the cone-class.

\hypertarget{microcausal-hygiene}{%
\subsubsection{5.5.3 Microcausal hygiene}\label{microcausal-hygiene}}

The poke cone, orbit locality, and LR-type bounds ensure commutator
growth remains inside the cone; the principal-symbol lemma (§4.1)
forbids order flips without paying the \(W_1\) gap. All variational
steps above remain legal within the cone-preserving class.

\begin{center}\rule{0.5\linewidth}{0.5pt}\end{center}

\hypertarget{summary-coupled-laws}{%
\subsection{5.6 --- Summary (coupled laws)}\label{summary-coupled-laws}}

At joint KKT stationarity of \(\mathcal J\) on the cone-preserving
class:

\begin{itemize}
\item
  \textbf{Slow (Einstein--Yang--Mills).}

  \[
  G_{\mu\nu}+\Lambda g_{\mu\nu}=8\pi G\,T^{\rm eff}_{\mu\nu},\qquad
  D^\alpha F_{\alpha\beta}=J^{\rm eff}_\beta,
  \]

  with \(T^{\rm eff},J^{\rm eff}\) \textbf{well-posed as Clarke
  subgradients} (conserved for \textbf{every} selection). Under
  \textbf{Assumption L} (unique minimizer or Clarke-regular active set
  with constant directional derivatives), they \textbf{upgrade} to
  single linear distributions with distributional
  (Riesz-on-bounded-domains) representation,
  \(T^{\rm eff}_{\mu\nu}=T^{\rm eff}_{\nu\mu}\),
  \(\nabla^\mu T^{\rm eff}_{\mu\nu}=0\), \(D^\mu J^{\rm eff}_\mu=0\).
\item
  \textbf{Fast (unitary/GKSL).}

  \[
  \dot A=\tfrac{i}{\hbar}[H^*(A),A],\quad
  H^*(A)=\mathcal A_A^{+}\!\big(-i[A,G(A)]\big),\quad
  \mathcal A_A=\mathrm{ad}_A^{\,*}\mathrm{ad}_A=\mathrm{ad}_A^2,\quad
  \hbar=\lambda_{\rm th}^{-1}.
  \]

  With leakage, the optimal generator is \textbf{GKSL} with
  leakage-penalized pointer alignment; well-posed with LR-type bounds.
\item
  \textbf{Constants.} \(\hbar,\,G,\,\Lambda,\,g_{\rm YM}\) are the
  \textbf{unique} KKT multipliers and match the Γ-calibrated constants
  (Prop. 5.4.3).
\end{itemize}

All steps are now rigorous and consistent with the coherence-budget
framework: directional envelope calculus with direction-selecting
minimizers; Clarke-sound sources and conservation (upgraded to tensors
when linearity holds); first-variation convergence for the slow Γ-limit;
and a derivation-quadratic control law on the unitary orbit (with
pseudoinverse and explicit normalization) producing the Heisenberg/GKSL
fast dynamics.

\hypertarget{chapter-6-gauge-matter-budgetsymmetry-selection-hypercharge-uniqueness}{%
\section{Chapter 6 --- Gauge \& Matter: Budget--Symmetry Selection;
Hypercharge
Uniqueness}\label{chapter-6-gauge-matter-budgetsymmetry-selection-hypercharge-uniqueness}}

\begin{quote}
\textbf{Scope.} With budgets and cone hygiene in place, we summarize the
selection for the gauge scaffold and record the hypercharge‑uniqueness
result under the standard binders (single Higgs doublet; anomaly
cancellation; minimal chiral set). Proofs and linear‑system details
mirror Appendix C.
\end{quote}

\hypertarget{budgetsymmetry-selection-sketch}{%
\subsection{6.1 Budget--symmetry selection
(sketch)}\label{budgetsymmetry-selection-sketch}}

Ad‑invariant quadratic forms on a compact simple Lie algebra are
multiples of the Killing form; additivity across factors gives
\(\sum_i C_2(G_i)\). The unique Ad‑invariant count surcharge is
\(\sum_i\dim G_i\). These, with Γ‑locality and mediator locality,
generate the mediator part of the \textbf{complexity} budget
\(B_{\rm cx}\). Minimal coupling follows from the Γ‑limit of the
connection scaffold.

\hypertarget{hypercharge-uniqueness-binder-set-yukawa-anomalies-minimal-set}{%
\subsection{6.2 Hypercharge uniqueness (binder set: Yukawa + anomalies +
minimal
set)}\label{hypercharge-uniqueness-binder-set-yukawa-anomalies-minimal-set}}

\begin{quote}
\textbf{Assumptions for hypercharge (conditional uniqueness).}\\
Uniqueness here is \textbf{conditional} on the binder set: (B\_Yuk)
renormalizable Yukawas, (B\_\{\rm anom\}) anomaly cancellation, and
(B\_\{\rm min\}) the minimal chiral content with a single Higgs doublet.
Relaxing any binder re-opens branches; see Appendix C for alternatives
and their costs.
\end{quote}

\textbf{Binders.} (i) \textbf{B\_Yuk:} only renormalizable Yukawas
\(QH d^c,\ Q\tilde H u^c,\ L H e^c\). (ii) \textbf{B\_anom:} cancel
\([SU(3)]^2U(1)_Y,\ [SU(2)]^2U(1)_Y,\ U(1)_Y^3,\ \text{grav}^2\!\!\!-\!U(1)_Y\)
anomalies. (iii) \textbf{B\_min:} one‑generation minimal chiral set
\(\{Q,u^c,d^c,L,e^c\}\); one Higgs doublet.

\textbf{Statement.} Let \((Y_Q,Y_u,Y_d,Y_L,Y_e,Y_H)\) be unknown
hypercharges. Under \textbf{B\_Yuk + B\_anom + B\_min}, the solution set
is a \textbf{one‑parameter line} (overall normalization/orientation).
Fixing the unit with the minimal‑charge binder (\(Q=T_3+Y\) and
color‑singlet integrality) yields the \textbf{Standard Model values}
\(\boxed{(Y_Q,Y_u,Y_d,Y_L,Y_e,Y_H)=\big(\tfrac16,-\tfrac23,\tfrac13,-\tfrac12,1,-\tfrac12\big)}.\)

\textbf{Proof (linear system \& rank).} Yukawa invariance gives
\(Y_Q+Y_H+Y_d=0\), \(Y_Q-Y_H+Y_u=0\), \(Y_L+Y_H+Y_e=0\). Anomalies add
\(3Y_Q+Y_L=0\) and \(6Y_Q+3Y_u+3Y_d+2Y_L+Y_e=0\). Solve: \(Y_H=-3Y_Q\),
\(Y_L=-3Y_Q\), \(Y_e=+6Y_Q\), \(Y_u=-4Y_Q\), \(Y_d=+2Y_Q\) with \(Y_Q\)
free (rank 5/6). Minimal‑charge normalization via \(Q=T_3+Y\) and
\(Q(\nu)=0\) fixes \(Y_Q=1/6\). After normalization, strict
convexity/tie‑breakers on \(B_{\rm cx}\) exclude co‑minima; the SM
assignment is unique with a positive gap.

\hypertarget{chapter-7-horizons-pointer-mechanics-area-law-hawking-flux-suppression-repaired-airtight}{%
\section{Chapter 7 --- Horizons: Pointer Mechanics ⇒ Area Law; Hawking
Flux Suppression (Repaired \&
Airtight)}\label{chapter-7-horizons-pointer-mechanics-area-law-hawking-flux-suppression-repaired-airtight}}

\begin{quote}
\textbf{Scope.} We work in a local \textbf{Killing-horizon patch} with
surface gravity \$\kappa\textgreater0\$, the \textbf{single-cone} class
and the C*-compatible budgets of Ch. 3, Γ-compact slow sector of Ch. 4,
and the coupled-laws framework of Ch. 5. We: (i) specify a
\textbf{near-horizon Unruh-diagonal GKSL model} fixed by \textbf{pointer
alignment}; (ii) take \textbf{mutual information} \$I(A:\bar A)\$ as the
primary entropy functional (with \$E\_R\le I\$), prove
\textbf{quasi-factorization} with explicit LR-dependent constants; (iii)
give a \textbf{per-tile information--budget bound} controlled by the
leakage budget; define a \textbf{pointer cutoff} \$\ell\_*\$ by KKT
thresholding; and prove an \textbf{area-law upper bound} with a
calibrated constant; (iv) prove a \textbf{sharp flux-suppression
inequality} as a linear program in the rates, in full generality and
with an exact \textbf{Hawking-weight} corollary; (v) record
\textbf{microcausality} and an \textbf{identifiability lemma} to
determine Hawking rates from pointer-basis two-point data. Proof details
and constant tracking are in \textbf{Appendix G} (area law) and
\textbf{Appendix H} (flux suppression). \textbf{KPI \& refuter.} The
near-horizon prediction is a \textbf{universal amplitude suppression} of
the outgoing flux by a budget-controlled factor at \textbf{fixed Hawking
temperature}. Any observed \textbf{temperature shift at leading order}
would falsify the budget-consistent GKSL picture; amplitude-only
suppression with the predicted frequency shaping supports it.
\end{quote}

\begin{center}\rule{0.5\linewidth}{0.5pt}\end{center}

\hypertarget{near-horizon-setup-pointer-alignment-and-quasi-factorization}{%
\subsection{7.1 Near-horizon setup, pointer alignment, and
quasi-factorization}\label{near-horizon-setup-pointer-alignment-and-quasi-factorization}}

\hypertarget{geometry-and-unruh-modes}{%
\subsubsection{7.1.1 Geometry and Unruh
modes}\label{geometry-and-unruh-modes}}

Fix a \textbf{stationary Killing horizon} with surface gravity
\$\kappa\textgreater0\$. On a bounded chart of the horizon neighborhood
we use \textbf{Rindler coordinates} to the accuracy guaranteed by Ch. 4
bounded-geometry hypotheses. Let

\[
\{b_k^{\rm out},\,b_k^{\rm in}\}_{k\in\mathcal K}
\]

denote Unruh mode operators localized to the patch (frequency
\$\omega\_k\textgreater0\$ and tangential quantum numbers). The Unruh
temperature is

\[
T_U=\frac{\kappa}{2\pi},\qquad \beta_U:=T_U^{-1}.
\]

\hypertarget{fast-sector-gksl-and-pointer-alignment}{%
\subsubsection{7.1.2 Fast-sector GKSL and pointer
alignment}\label{fast-sector-gksl-and-pointer-alignment}}

On a horizon \textbf{tile} \$T\$ (area \$\textbar T\textbar\$), the fast
sector evolves by a GKSL generator that is \textbf{diagonal in the Unruh
(pointer) basis}:

\[
\mathcal L_T(\rho)=\sum_{k\in\mathcal K_T}\!\big[\gamma_k\,\mathcal D[b_k^{\rm out}](\rho)+\tilde\gamma_k\,\mathcal D[(b_k^{\rm out})^\dagger](\rho)\big],\quad
\mathcal D[C](\rho):=C\rho C^\dagger-\tfrac12\{C^\dagger C,\rho\}.
\]

The \textbf{leakage weight} \$W\succ0\$ (Appendix A) co-diagonalizes
with the Unruh basis; write
\$w\_k:=\langle k\textbar W\textbar k\rangle\textgreater0\$. The
\textbf{per-tile leakage budget} is

\[
\mathcal B_{\rm leak}(T)=\sum_{k\in\mathcal K_T} w_k\,(\gamma_k+\tilde\gamma_k).
\]

\textbf{Hawking detailed balance} has

\[
\tilde\gamma_k^{\rm H}=\gamma_k^{\rm H}\,e^{-\beta_U\omega_k},\qquad \bar n_k=(e^{\beta_U\omega_k}-1)^{-1}.
\]

\textbf{Lemma 7.1 (Budget-monotonic pointer alignment).} Let
\$\mathcal L\_T\$ be any GKSL on \$T\$ with the same singular values of
the noise block as above. Let \$\Delta\$ be the \textbf{pinching} (full
dephasing) in the Unruh basis. Then:

\begin{enumerate}
\def\labelenumi{\arabic{enumi}.}
\tightlist
\item
  (Budget) \$\sum\_j\omega(L\_j\^{}\dagger W
  L\_j)~\ge~\sum\_j\omega!\big((\Delta L\_j)\^{}\dagger W(\Delta L\_j)\big)\$
  (Hilbert--Schmidt \textbf{pinching contraction}).
\item
  (Information) For any state \$\rho\$,
  \$I\big((\mathrm{id}\otimes\Lambda\_t)(\rho):\bar T\big)\$
  \textbf{does not increase} if one replaces
  \$\Lambda\_t:=e\^{}\{t\mathcal L\_T\}\$ by
  \$\Delta\circ\Lambda\_t\circ\Delta\$ (monotonicity of \$I\$ under
  local CPTP maps and commutation of \$\Delta\$ with the Unruh-diagonal
  semigroup).
\end{enumerate}

Hence, among channels with fixed spectral data, \textbf{Unruh-diagonal}
noise \textbf{minimizes} leakage cost and \textbf{maximizes}
\$I(T:\overline T)\$ \textbf{only} within that spectral class. We
therefore restrict to the \textbf{diagonal} form without loss for upper
bounds.

\emph{Proof.} (1) The map \$X\mapsto W\^{}\{1/2\}X\$ followed by
conditional expectation \$\Delta\$ is a contraction in the HS norm; sum
over \$j\$. (2) Mutual information is monotone under local CPTP maps;
\$\Delta\$ is local on \$T\$ and leaves the Unruh-diagonal dynamics
invariant; compose. \$\square\$

\hypertarget{entropy-functional-and-the-e_r-relation}{%
\subsubsection{7.1.3 Entropy functional and the \$E\_R\$
relation}\label{entropy-functional-and-the-e_r-relation}}

We take \textbf{mutual information} as primary:

\[
I(A:\bar A)=D\big(\rho_{A\bar A}\,\big\|\,\rho_A\otimes\rho_{\bar A}\big).
\]

We use the general inequality

\[
\boxed{\,E_R(\rho_{A\bar A})\ \le\ I(A:\bar A)\,}
\]

(no \$\tfrac12\$ factor in general). All subsequent bounds are proved
for \$I\$ and then immediately transfer to \$E\_R\$.

\hypertarget{quasi-factorization-with-lr-constants}{%
\subsubsection{7.1.4 Quasi-factorization with LR
constants}\label{quasi-factorization-with-lr-constants}}

Tile a region \$A\$ by disjoint squares \$\{T\_j\}\_\{j\in J(A)\}\$ of
side \$\ell\$, and include a \textbf{boundary collar} of width
\$O(\ell)\$, yielding \$O(\textbar{}\partial A\textbar)\$ collar tiles.
The GKSL semigroup obeys \textbf{cone-limited LR bounds} (Ch. 3 U4;
Appendix F.2) and admits an Unruh thermal \textbf{log-Sobolev} (or
spectral-gap) constant on each tile (Appendix G.1).

\textbf{Theorem 7.0 (Quasi-factorization of mutual information).} There
exist \$C\_\{\rm LR\},\xi,v\_\{\rm LR\}\textless{}\infty\$, depending
only on the LR/mixing data and the local mode density, such that for all
\$\ell\ge \ell\_0\$ and all horizon-patch regions \$A\$,

\[
\boxed{\
I(A:\bar A)\ \le\ \sum_{j\in J(A)}\! I(T_j:\overline{T_j})\ +\ C_{\rm LR}\,|\partial A|\ +\ C_{\rm LR}\,e^{-\ell/\xi}.
\ }
\]

In particular, choosing
\$\ell\ge c,\xi\log(1+\textbar{}\partial A\textbar)\$ absorbs the
remainder into the boundary term: \$I(A:\bar A)\le \sum\_j
I(T\_j:\overline{T\_j})+O(\textbar{}\partial A\textbar)\$.

\emph{Proof (outline; Appendix G.2).} Chain the mutual information by
\textbf{SSA}: \$I(A:\bar A)=\sum\_\{j\}
I\big(T\_j:\bar A,\big\textbar,T\_\{\textless j\}\big)\$ and bound each
conditional term by an LR-decaying influence from \$\overline{T\_j}\$
outside a collar of width \$\sim\ell\$. Mix to Unruh steady on the
collar using the tile log-Sobolev gap; constants track to
\$C\_\{\rm LR\},\xi\$. \$\square\$

\begin{center}\rule{0.5\linewidth}{0.5pt}\end{center}

\hypertarget{area-law-upper-bound-with-calibrated-constant}{%
\subsection{7.2 Area-law upper bound with calibrated
constant}\label{area-law-upper-bound-with-calibrated-constant}}

We now reduce the area bound to a \textbf{per-tile} estimate and
calibrate the constant.

\hypertarget{per-tile-information-vs.-leakage-budget}{%
\subsubsection{7.2.1 Per-tile information vs.~leakage
budget}\label{per-tile-information-vs.-leakage-budget}}

For a tile \$T\$, write \$\Gamma\_k:=\gamma\_k+\tilde\gamma\_k\$ and
define

\[
\tau_{\rm leak}(T):=\sum_{k\in\mathcal K_T} w_k\,\Gamma_k.
\]

\textbf{Proposition 7.1 (Linear upper bound, sharp coefficient).} There
exists a finite constant

\[
\chi_T:=\sup_{k\in\mathcal K_T}\ \frac{\partial^+ I_k}{\partial \Gamma_k}\,\frac{1}{w_k}\,,
\]

(where \$I\_k\$ is the single-mode contribution under Unruh-diagonal
dynamics and \$\partial\^{}+\$ is the right derivative at the realized
\$\Gamma\_k\$) such that

\[
\boxed{\ I(T:\overline T)\ \le\ \chi_T\,\tau_{\rm leak}(T)\ .\ }
\]

Moreover \$\chi\_T\$ depends only on the Unruh temperature, the
LR/mixing constants, and the local mode density; it is \textbf{uniform}
across tiles of the same side \$\ell\$.

\emph{Proof (Appendix G.3).} For Unruh-diagonal GKSL,
\$I(T:\overline T)=\sum\_k I\_k(\Gamma\_k)\$. Each \$I\_k\$ is
\textbf{concave}, increasing in \$\Gamma\_k\$ and differentiable a.e.
(data processing + semigroup contractivity in the normalized-HS metric
used for budgets, Appendix E.1). Lagrange's inequality then gives
\$I(T:\overline T)\le \sum\_k (\partial\^{}+
I\_k/\partial\Gamma\_k),\Gamma\_k \le \chi\_T \sum\_k w\_k \Gamma\_k.\$
Uniformity follows from bounded geometry and the common Unruh
temperature. \$\square\$

\begin{quote}
\textbf{Remark.} We carry (7.2.1) as an \textbf{upper bound}. Under the
\textbf{Hawking-weight calibration} of §7.3.2 the coefficient becomes
tile-independent (\$\chi\_T\equiv\chi\_*\$) and the bound is
\textbf{tight} (equality along the Hawking ray).
\end{quote}

\hypertarget{kkt-thresholding-and-the-pointer-cutoff-_}{%
\subsubsection{\texorpdfstring{7.2.2 KKT thresholding and the pointer
cutoff
\$\ell\_*\$}{7.2.2 KKT thresholding and the pointer cutoff \$\_*\$}}\label{kkt-thresholding-and-the-pointer-cutoff-_}}

We optimize per-tile \$I(T:\overline T)\$ subject to the leakage
allowance \$\tau\_\{\rm leak\}(T)\$. The \textbf{KKT conditions} with
multiplier \$\lambda\_\{\rm leak\}\textgreater0\$ give the
\textbf{threshold rule}

\[
\frac{\partial^+ I_k}{\partial \Gamma_k}\ \le\ \lambda_{\rm leak}\,w_k,\quad
\Gamma_k>0\ \Rightarrow\ \frac{\partial I_k}{\partial \Gamma_k}=\lambda_{\rm leak}\,w_k.
\]

Because \$w\_k\$ grows monotonically with \textbf{transverse momentum}
(pointer/energy scaling) and \$\partial I\_k/\partial\Gamma\_k\$ is
bounded and decreases with \$\textbar k\textbar\$ at fixed \$\beta\_U\$,
the optimizer \textbf{activates modes} only for
\$\textbar k\textbar{}\lesssim \ell\_*\^{}\{-1\}\$ with a \textbf{sharp
cutoff} at

\[
\boxed{\ \ell_\*=\ell_\*(\lambda_{\rm leak},W)\ \ \text{defined by}\ \ \frac{\partial I_{k}}{\partial \Gamma_{k}}\Big|_{|k|=\ell_\*^{-1}}=\lambda_{\rm leak}\,w_{k}\ .\ }
\]

(Details in Appendix G.4.) For tiles with side \$\ell=\Theta(\ell\_*)\$,
the number of active transverse modes is
\$\simeq \textbar T\textbar/\ell\_*\^{}\{,d-2\}\$ up to boundary/collar
corrections.

\hypertarget{area-law-theorem}{%
\subsubsection{7.2.3 Area-law theorem}\label{area-law-theorem}}

\textbf{Theorem 7.1 (Area-law upper bound with boundary term).} For
tiles of side \$\ell=\Theta(\ell\_*)\$,

\[
\boxed{\
I(A:\bar A)\ \le\ \chi_\*(\beta_U,{\rm LR},W)\,\frac{{\rm Area}(A)}{\ell_\*^{\,d-2}}\ +\ O(|\partial A|),\qquad
E_R(A:\bar A)\ \le\ I(A:\bar A).
\ }
\]

Here \$\chi\_*:=\sup\_T\chi\_T\$ is finite and depends only on the Unruh
temperature, LR/mixing constants and the local weight profile \$W\$
(Appendix G.5). The \$O(\textbar{}\partial A\textbar)\$ constant depends
only on LR/geometry data.

\emph{Proof.} Combine Theorem 7.0 with Proposition 7.1 and the
thresholding definition of \$\ell\_*\$; count active modes per interior
tile and absorb LR remainders into the collar. \$\square\$

\hypertarget{calibration-to-einsteinhilbert-and-the-14-coefficient}{%
\subsubsection{7.2.4 Calibration to Einstein--Hilbert and the \$1/4\$
coefficient}\label{calibration-to-einsteinhilbert-and-the-14-coefficient}}

\textbf{Proposition 7.2 (EH calibration \$\Rightarrow\$
\$\chi\_*=\tfrac14\$ in Planck units).} Under the \textbf{Γ-limit
normalization} of Appendix D (Ch. 4/D.5), the slow action equals
Einstein--Hilbert and the Unruh temperature is fixed by \$\kappa\$.
Matching the \textbf{tile-wise Clausius relation}
\$\delta Q=T\_U,\delta S\$ with the \textbf{leakage work} priced by
\$W\$ at the pointer cutoff yields

\[
\boxed{\ \chi_\*=\frac14\ ,\qquad
I(A:\bar A)\ \le\ \frac{{\rm Area}(A)}{4\,\ell_P^{\,d-2}}\ +\ O(|\partial A|). \ }
\]

\emph{Proof (Appendix G.6).} The per-tile leakage expenditure that
realizes the Unruh KMS structure at the cutoff equals the \textbf{heat}
\$\delta Q\$ through the stretched horizon; with \$T\_U\$ fixed, the
\$\delta S\$ density matches EH's area-entropy density, fixing
\$\chi\_*=1/4\$. \$\square\$

\begin{center}\rule{0.5\linewidth}{0.5pt}\end{center}

\hypertarget{hawking-flux-suppression-sharp-lp-bound-and-calibrated-equality}{%
\subsection{7.3 Hawking-flux suppression: sharp LP bound and calibrated
equality}\label{hawking-flux-suppression-sharp-lp-bound-and-calibrated-equality}}

\hypertarget{general-linear-program-bound-no-profile-assumptions}{%
\subsubsection{7.3.1 General linear-program bound (no profile
assumptions)}\label{general-linear-program-bound-no-profile-assumptions}}

Let \$c\_k\$ denote the \textbf{outgoing flux weight} per mode (number
or energy),

\[
c_k:=\begin{cases}
v_k\,\bar n_k & \text{(number flux)},\\[2pt]
\hbar\omega_k\,v_k\,\bar n_k & \text{(energy flux)},
\end{cases}
\qquad 0<v_k\le v_{\max}.
\]

For a tile \$T\$,

\[
\mathcal F_{\rm out}(T)=\sum_{k\in\mathcal K_T} c_k\,\Gamma_k,\qquad 
\tau_{\rm leak}(T)=\sum_{k\in\mathcal K_T} w_k\,\Gamma_k.
\]

\textbf{Theorem 7.2 (General LP suppression).} For every tile \$T\$,

\[
\boxed{\
\mathcal F_{\rm out}(T)\ \le\ \Big(\max_{k\in\mathcal K_T}\frac{c_k}{w_k}\Big)\ \tau_{\rm leak}(T)\ .
\ }
\]

Equality holds by concentrating the budget on a mode with maximal ratio
\$c\_k/w\_k\$. Summing tiles gives

\[
\boxed{\
\mathcal F_{\rm out}\ \le\ \Big(\max_{k}\frac{c_k}{w_k}\Big)\ \tau_{\rm leak}\ .
\ }
\]

\emph{Proof.} Linear objective over a simplex: maximize
\$\sum c\_k\Gamma\_k\$ s.t. \$\sum w\_k\Gamma\_k\le\tau\$ and
\$\Gamma\_k\ge0\$. KKT gives support on argmax of \$c\_k/w\_k\$.
\$\square\$

\begin{quote}
\textbf{Microcausality.} The LR bounds (Appendix F.2) imply
\$\textbar{[}\Phi\_t(O\_X),O\_Y{]}\textbar{}\le C,e\^{}\{-(d(X,Y)-v\_\{\rm LR\}t)/\xi\}\$;
hence no superluminal contribution to \$\mathcal F\_\{\rm out\}\$ is
admissible.
\end{quote}

\hypertarget{hawking-weight-calibration-and-exact-multiplicative-factor}{%
\subsubsection{7.3.2 Hawking-weight calibration and exact multiplicative
factor}\label{hawking-weight-calibration-and-exact-multiplicative-factor}}

Define the \textbf{Hawking-weight} \$W\_\{\rm H\}\$ by

\[
\boxed{\,w_k^{\rm H}\ \propto\ c_k\ ,\ }
\]

i.e.~choose the leakage weight to price exactly the \textbf{flux
contribution} (number or energy). This choice is canonical operationally
(Appendix J.1: weight normalization by KPI).

Let \$\Gamma\_k\^{}\{\rm H\}\$ be the \textbf{detailed-balance rates}
reproducing the semiclassical Hawking flux, and define the
\textbf{Hawking leakage cost}

\[
\tau_{\rm H}(T):=\sum_{k\in\mathcal K_T} w_k^{\rm H}\,\Gamma_k^{\rm H},\qquad
\mathcal F_{\rm Hawking}(T):=\sum_k c_k\,\Gamma_k^{\rm H}.
\]

By construction,

\[
\frac{\mathcal F_{\rm Hawking}(T)}{\tau_{\rm H}(T)}=\frac{\sum_k c_k\Gamma_k^{\rm H}}{\sum_k w_k^{\rm H}\Gamma_k^{\rm H}}=\frac{\sum_k c_k\Gamma_k^{\rm H}}{\sum_k (\alpha c_k)\Gamma_k^{\rm H}}=\alpha^{-1},
\]

a \textbf{mode-independent} constant (absorbed into the normalization of
\$W\_\{\rm H\}\$).

\textbf{Corollary 7.2′ (Budget-limited Hawking).} With
\$W=W\_\{\rm H\}\$, the LP optimizer aligns with the \textbf{Hawking
ray} and yields

\[
\boxed{\
\mathcal F_{\rm out}(T)\ =\ \min\!\Big\{1,\ \frac{\tau_{\rm leak}(T)}{\tau_{\rm H}(T)}\Big\}\ \mathcal F_{\rm Hawking}(T),
\ }
\]

with equality cases: (i) \$\tau\_\{\rm leak\}(T)\ge\tau\_\{\rm H\}(T)\$
and \$\Gamma\_k=\Gamma\_k\^{}\{\rm H\}\$ (full Hawking), (ii)
\$\tau\_\{\rm leak\}(T)\textless{}\tau\_\{\rm H\}(T)\$ and
\$\Gamma\_k=\alpha,\Gamma\_k\^{}\{\rm H\}\$ with
\$\alpha=\tau\_\{\rm leak\}(T)/\tau\_\{\rm H\}(T)\$ (budget-limited
Hawking). Summing tiles gives

\[
\boxed{\
\mathcal F_{\rm out}\ =\ \min\!\Big\{1,\ \frac{\tau_{\rm leak}}{\tau_{\rm H}(W_{\rm H})}\Big\}\ \mathcal F_{\rm Hawking}\ .
\ }
\]

\emph{Proof.} Under \$w\_k\^{}\{\rm H\}\propto c\_k\$,
\$c\_k/w\_k\^{}\{\rm H\}\$ is \textbf{constant in \$k\$}; every
Hawking-proportional allocation maximizes the LP; scale by feasibility.
\$\square\$

\begin{center}\rule{0.5\linewidth}{0.5pt}\end{center}

\hypertarget{identifiability-observables-and-explicit-f}{%
\subsection{7.4 Identifiability, observables, and explicit
\$f\$}\label{identifiability-observables-and-explicit-f}}

\hypertarget{identifiability-of-hawking-rates-in-the-pointer-model}{%
\subsubsection{7.4.1 Identifiability of Hawking rates in the pointer
model}\label{identifiability-of-hawking-rates-in-the-pointer-model}}

\textbf{Lemma 7.3 (Pointer-basis identifiability).} In the
Unruh-diagonal GKSL model, the \textbf{two-point functions} of the
outside modes,

\[
C_k(t):=\langle b_k^{\rm out}(t)\,(b_k^{\rm out})^\dagger(0)\rangle,\qquad 
\tilde C_k(t):=\langle (b_k^{\rm out})^\dagger(t)\,b_k^{\rm out}(0)\rangle,
\]

satisfy

\[
C_k(t)=\big(1+\bar n_k\big)\,e^{-\tfrac12\Gamma_k t},\qquad 
\tilde C_k(t)=\bar n_k\,e^{-\tfrac12\Gamma_k t}.
\]

Hence the \textbf{KMS ratio}
\$\tilde C\_k(t)/C\_k(t)=\bar n\_k/(1+\bar n\_k)=e\^{}\{-\beta\_U\omega\_k\}\$
fixes \$\beta\_U\$, and the \textbf{linewidth} identifies \$\Gamma\_k\$.
In particular, the Hawking rates \$\Gamma\_k\^{}\{\rm H\}\$ are
\textbf{uniquely determined} from pointer-basis two-point data within
this model class.

\emph{Proof.} Solve the Heisenberg equations under the Unruh-diagonal
Lindbladian; the amplitudes decay at rate \$\tfrac12\Gamma\_k\$ while
detailed balance fixes the ratio. \$\square\$

\hypertarget{kpi-and-computation-of-f}{%
\subsubsection{7.4.2 KPI and computation of
\$f\$}\label{kpi-and-computation-of-f}}

\textbf{Primary KPI.} The \textbf{flux ratio}

\[
R:=\frac{\mathcal F_{\rm out}}{\mathcal F_{\rm Hawking}}\ =\
\begin{cases}
\displaystyle \min\!\Big\{1,\frac{\tau_{\rm leak}}{\tau_{\rm H}(W_{\rm H})}\Big\}, & \text{for }W=W_{\rm H};\\[8pt]
\displaystyle \le\ \Big(\max_k \tfrac{c_k}{w_k}\Big)\,\frac{\tau_{\rm leak}}{\mathcal F_{\rm Hawking}}, & \text{general }W.
\end{cases}
\]

\textbf{Procedure.}

\begin{enumerate}
\def\labelenumi{\arabic{enumi}.}
\tightlist
\item
  Tomography in the \textbf{pointer basis} (Appendix J.1) gives
  \$\Gamma\_k\^{}\{\rm H\}\$ from linewidths and \$\beta\_U\$ from KMS.
\item
  Compute the \textbf{Hawking cost}
  \$\tau\_\{\rm H\}(W)=\sum w\_k,\Gamma\_k\^{}\{\rm H\}\$ for the chosen
  weight \$W\$ (or set \$W=W\_\{\rm H\}\$ to make it canonical).
\item
  Evaluate \$f\$ by the formulas above.
\end{enumerate}

\hypertarget{pointer-cutoff-and-tile-choice}{%
\subsubsection{7.4.3 Pointer cutoff and tile
choice}\label{pointer-cutoff-and-tile-choice}}

Set the tile side \$\ell\$ by \textbf{allowance matching}:

\[
\sum_{k\in\mathcal K_T} w_k\,\Gamma_k^{\rm H}\ =\ \tau_{\rm leak}(T),
\]

which solves to \$\ell=\Theta(\ell\_*)\$ defined in §7.2.2. With this
choice, the \textbf{quasi-factorization remainder} is absorbed into
\$O(\textbar{}\partial A\textbar)\$ and the area coefficient is
calibrated by Proposition 7.2.

\begin{center}\rule{0.5\linewidth}{0.5pt}\end{center}

\hypertarget{microcausality-guard-and-hygiene}{%
\subsection{7.5 Microcausality (guard) and
hygiene}\label{microcausality-guard-and-hygiene}}

The \textbf{single-cone} class (Ch. 4) and U-assumptions (Ch. 3) yield
LR bounds (Appendix F.2):

\[
\|[\Phi_t(O_X),O_Y]\|\ \le\ C\,e^{-(d(X,Y)-v_{\rm LR}t)/\xi}.
\]

Thus neither the \textbf{area-law derivation} (local tilings and collar
mixing) nor the \textbf{flux LP} can exploit superluminal influences;
all optimizers live inside the cone.

\begin{center}\rule{0.5\linewidth}{0.5pt}\end{center}

\hypertarget{summary-airtight-form}{%
\subsection{7.6 Summary (airtight form)}\label{summary-airtight-form}}

\begin{enumerate}
\def\labelenumi{\arabic{enumi}.}
\tightlist
\item
  \textbf{Pointer alignment} is \textbf{budget-monotonic} and
  information-preserving in the sense of Lemma 7.1; hence we restrict to
  \textbf{Unruh-diagonal} GKSL.
\item
  A \textbf{quasi-factorization theorem} (Theorem 7.0) holds with
  explicit LR constants, reducing the bound to \textbf{per-tile}
  contributions plus an \$O(\textbar{}\partial A\textbar)\$ boundary
  term.
\item
  \textbf{Per-tile information} obeys a \textbf{linear upper bound}
  \$I(T:\overline T)\le \chi\_T,\tau\_\{\rm leak\}(T)\$ (Proposition
  7.1), with a \textbf{KKT threshold} producing a \textbf{pointer
  cutoff} \$\ell\_*\$ (§7.2.2).
\item
  Summing tiles gives an \textbf{area law} with explicit coefficient
  \$\chi\_*/\ell\_*\^{}\{,d-2\}\$ (Theorem 7.1). Under EH
  \textbf{Γ-calibration}, \$\chi\_*=1/4\$ (Proposition 7.2), yielding
\end{enumerate}

\[
E_R(A:\bar A)\ \le\ I(A:\bar A)\ \le\ \frac{{\rm Area}(A)}{4\,\ell_P^{\,d-2}}\ +\ O(|\partial A|).
\]

\begin{enumerate}
\def\labelenumi{\arabic{enumi}.}
\setcounter{enumi}{4}
\tightlist
\item
  \textbf{Flux suppression} is a sharp \textbf{linear program} (Theorem
  7.2). With the \textbf{Hawking-weight} \$W\_\{\rm H\}\$, the optimizer
  is \textbf{budget-limited Hawking}:
\end{enumerate}

\[
\mathcal F_{\rm out}\ =\ \min\!\Big\{1,\ \frac{\tau_{\rm leak}}{\tau_{\rm H}(W_{\rm H})}\Big\}\ \mathcal F_{\rm Hawking}.
\]

\begin{enumerate}
\def\labelenumi{\arabic{enumi}.}
\setcounter{enumi}{5}
\tightlist
\item
  \textbf{Identifiability} (Lemma 7.3) pins \$\Gamma\_k\^{}\{\rm H\}\$
  from pointer-basis two-point data; \textbf{microcausality} is enforced
  by LR bounds.
\end{enumerate}

All constants and intermediate inequalities are tracked in
\textbf{Appendix G} (area) and \textbf{Appendix H} (flux), ensuring the
chapter's statements are \textbf{fully rigorous} within the stated
hypotheses.

\hypertarget{chapter-8-coherence-rg-scale-flow-fixed-points}{%
\section{Chapter 8 --- Coherence RG: Scale Flow \& Fixed
Points}\label{chapter-8-coherence-rg-scale-flow-fixed-points}}

\begin{quote}
\textbf{Scope.} Define a coherence‑preserving coarse‑graining and derive
an RG flow on budgets and predictive envelopes. Fixed points correspond
to scale‑invariant scaffolds; relevant directions match active budgets.
\end{quote}

\hypertarget{coarsegraining-monotonicity}{%
\subsection{8.1 Coarse‑graining \&
monotonicity}\label{coarsegraining-monotonicity}}

Local coarse‑grainings \(\mathcal C_\ell\) at scale \(\ell\) respect the
poke cone (causal, Γ‑local). Budgets obey \textbf{monotone} inequalities
\(B_\bullet(\mathcal C_\ell A)\ \le\ c_\bullet(\ell)\, B_\bullet(A),\quad \bullet\in\{\rm th,cx,leak\}.\)

\hypertarget{flow-equations-envelope-picture}{%
\subsection{8.2 Flow equations (envelope
picture)}\label{flow-equations-envelope-picture}}

Scale‑\(\ell\) objective
\(\mathcal V_\ell(A)= \inf_{\Phi\in\overline{\mathcal P}}\mathrm{CL}(\mathcal C_\ell A,\Phi) - \sum_\bullet \lambda_\bullet B_\bullet(\mathcal C_\ell A)\).
The \textbf{coherence RG} is
\(\partial_{\ln\ell}\, \mathcal V_\ell\ =\ \mathcal D\,\mathcal V_\ell\ -\sum_\bullet (\partial_{\ln\ell}\ln c_\bullet)\cdot \lambda_\bullet B_\bullet\ +\ \text{(irrelevant corrections)}.\)

\hypertarget{fixed-points-and-stability}{%
\subsection{8.3 Fixed points and
stability}\label{fixed-points-and-stability}}

A \textbf{fixed point} satisfies stationarity of \(\mathcal V_\ell\) up
to rescaling; budgets transform covariantly. Linearization gives scaling
exponents for (th,cx,leak) channels and binders. The RG‑envelope yields
closure rules for multipliers by matching to observed invariants.

\hypertarget{chapter-9-quantum-geometry-completion-constraint-algebra-path-measure}{%
\section{Chapter 9 --- Quantum Geometry Completion: Constraint Algebra
\& Path
Measure}\label{chapter-9-quantum-geometry-completion-constraint-algebra-path-measure}}

\begin{quote}
\textbf{Scope.} We (i) compute the hypersurface-deformation
(ADM/Henneaux--Teitelboim) algebra \emph{inside the cone-preserving
class}, verifying \textbf{closure without anomalies} and persistence at
the Γ-limit, and (ii) construct a \textbf{cone-limited projective path
measure} with explicit finite-dimensional marginals, proving
\textbf{cylinder consistency}, \textbf{FKG/positive association} under
an attractive discretization, and \textbf{microcausality via LR-type
bounds} at the measure level. All hypotheses and objects align with the
budgets/topologies fixed earlier (bounded geometry, de Donder/harmonic
gauge on charts, single-cone hygiene, Γ-locality). \textbf{BRST/BV
hygiene note (anomaly absence at the Γ-limit).} Within the
cone-preserving, gauge-fixed class and bounded-geometry hypotheses, the
hypersurface-deformation algebra closes \textbf{without central
extensions} (Appendix D/F bounds). The corresponding BRST charge is
nilpotent, and a unitary BV/BV measure exists for the projective path
construction; microcausality (LR-type) guards ensure cylinder
consistency.
\end{quote}

\begin{center}\rule{0.5\linewidth}{0.5pt}\end{center}

\hypertarget{constraint-algebra-closure-admhenneauxteitelboim-inside-the-cone-class}{%
\subsection{9.1 Constraint algebra closure (ADM/Henneaux--Teitelboim
inside the cone
class)}\label{constraint-algebra-closure-admhenneauxteitelboim-inside-the-cone-class}}

\hypertarget{phase-space-constraints-and-smearings}{%
\subsubsection{Phase space, constraints, and
smearings}\label{phase-space-constraints-and-smearings}}

Let \$\Sigma\$ be a Cauchy slice of a globally hyperbolic spacetime with
bounded geometry. The canonical variables are the Riemannian metric
\$q\_\{ab\}\$ on \$\Sigma\$ and its conjugate momentum
\$\pi\textsuperscript{\{ab\}=\sqrt{q},(K}\{ab\}-K q\^{}\{ab\})\$
(indices raised/lowered with \$q\$). The (Poisson) symplectic form is

\[
\{F,G\}=\int_\Sigma\!\mathrm d^3x\ \Big(\frac{\delta F}{\delta q_{ab}}\frac{\delta G}{\delta \pi^{ab}}-\frac{\delta G}{\delta q_{ab}}\frac{\delta F}{\delta \pi^{ab}}\Big).
\]

The \textbf{scalar (Hamiltonian)} and \textbf{vector (momentum)}
constraints (pure gravity with \$\Lambda\$; minimal coupling to
gauge/matter adds standard pieces without modifying the algebraic
structure functions) are

\[
\begin{aligned}
\mathcal H_\perp &= \frac{1}{\sqrt{q}}\big(\pi^{ab}\pi_{ab}-\tfrac12\pi^2\big)-\sqrt{q}\,(R-2\Lambda)\ +\ \mathcal H_\perp^{\rm matter},\\
\mathcal H_a &= -2\,q_{ac}\,D_b\pi^{bc}\ +\ \mathcal H_a^{\rm matter},
\end{aligned}
\]

with \$D\$ the Levi-Civita connection of \$q\$, \$R\$ its scalar
curvature, and \$\pi:=q\_\{ab\}\pi\^{}\{ab\}\$. For smearings
\$N\in C\_c\^{}\infty(\Sigma)\$ and \$N\^{}a\in\mathfrak X\_c(\Sigma)\$
(compact support or appropriate falloff), define

\[
H[N]:=\int_\Sigma\! N\,\mathcal H_\perp,\qquad D[\vec N]:=\int_\Sigma\! N^a\,\mathcal H_a.
\]

All fields are restricted to the \textbf{single-cone} class (principal
symbol bounds) and we work in harmonic/de Donder gauge on charts when
needed (for well-posedness and elliptic control of gauge).

\hypertarget{hypotheses-for-this-block}{%
\subsubsection{Hypotheses for this
block}\label{hypotheses-for-this-block}}

\begin{itemize}
\tightlist
\item
  \textbf{(C1) Cone hygiene \& bounded geometry.} As in Ch. 4, uniform
  injectivity radius and curvature bounds; single-cone principal-symbol
  bounds hold.
\item
  \textbf{(C2) Boundary/falloff.} Either compact \$\Sigma\$ without
  boundary or standard ADM falloffs guaranteeing boundary terms vanish
  in the bracket computations (or are absorbed in the ADM surface
  charges if present; here we take vanishing boundaries for brevity).
\item
  \textbf{(C3) Γ-limit stability.} The slow action is the Γ-limit of
  second-order local forms (Ch. 4/Appendix D), so all variational
  derivatives converge in the sense needed below.
\end{itemize}

\hypertarget{the-hypersurface-deformation-algebra-hda-statements}{%
\subsubsection{The hypersurface-deformation algebra (HDA):
statements}\label{the-hypersurface-deformation-algebra-hda-statements}}

\[
\boxed{
\begin{aligned}
\{D[\vec N],D[\vec M]\} &= D\!\big[\,\mathcal L_{\vec N}\vec M\,\big],\\[2pt]
\{D[\vec N],H[M]\} &= H\!\big[\,\mathcal L_{\vec N} M\,\big],\\[2pt]
\{H[N],H[M]\} &= D\!\big[\,q^{ab}(N\partial_b M - M\partial_b N)\,\big],
\end{aligned}}
\tag{9.1}
\]

i.e., \textbf{closure with structure functions \$q\^{}\{ab\}\$ and no
central extensions.}

We prove (9.1) \emph{within} the cone-preserving class and verify
persistence at the Γ-limit.

\begin{center}\rule{0.5\linewidth}{0.5pt}\end{center}

\hypertarget{proof-9.1a-dndmdl_nm}{%
\subsubsection{\texorpdfstring{Proof (9.1a) ---
\$\{D{[}\vec N{]},D{[}\vec M{]}\}=D{[}\mathcal L\_\{\vec N\}\vec M{]}\$}{Proof (9.1a) --- \$\{D{[}N{]},D{[}M{]}\}=D{[}L\_\{N\}M{]}\$}}\label{proof-9.1a-dndmdl_nm}}

\$D{[}\vec N{]}\$ generates spatial diffeomorphisms:

\[
\{q_{ab},D[\vec N]\}=\mathcal L_{\vec N}q_{ab},\qquad \{\pi^{ab},D[\vec N]\}=\mathcal L_{\vec N}\pi^{ab}.
\]

Therefore, for any functional \$F\$,
\$\{F,D{[}\vec N{]}\}=\mathcal L\_\{\vec N\}F\$. Apply to
\$F=D{[}\vec M{]}\$; since \$D\$ is a spatial vector density of weight
one,

\[
\{D[\vec M],D[\vec N]\}=\mathcal L_{\vec N}D[\vec M]=D[\mathcal L_{\vec N}\vec M].
\]

No boundary term survives by (C2). \$\square\$

\hypertarget{proof-9.1b-dnhmhl_nm}{%
\subsubsection{\texorpdfstring{Proof (9.1b) ---
\$\{D{[}\vec N{]},H{[}M{]}\}=H{[}\mathcal L\_\{\vec N\}M{]}\$}{Proof (9.1b) --- \$\{D{[}N{]},H{[}M{]}\}=H{[}L\_\{N\}M{]}\$}}\label{proof-9.1b-dnhmhl_nm}}

\$H{[}M{]}\$ is a scalar density of weight one. Using the same generator
property,

\[
\{H[M],D[\vec N]\}=\mathcal L_{\vec N}H[M]=H[\mathcal L_{\vec N}M].
\]

Again, boundary terms vanish by (C2). \$\square\$

\hypertarget{proof-9.1c-hnhmdqabn_bm-m_bn}{%
\subsubsection{\texorpdfstring{Proof (9.1c) ---
\$\{H{[}N{]},H{[}M{]}\}=D{[}q\^{}\{ab\}(N\partial\_bM-M\partial\_bN){]}\$}{Proof (9.1c) --- \$\{H{[}N{]},H{[}M{]}\}=D{[}q\^{}\{ab\}(N\_bM-M\_bN){]}\$}}\label{proof-9.1c-hnhmdqabn_bm-m_bn}}

This is the nontrivial bracket. Write \$H{[}N{]}=T{[}N{]}+V{[}N{]}\$
with

\[
T[N]=\int N\ \frac{1}{\sqrt q}\Big(\pi^{ab}\pi_{ab}-\tfrac12\pi^2\Big),\qquad
V[N]=-\int N\ \sqrt q\,(R-2\Lambda)+\int N\,\mathcal H_\perp^{\rm matter}.
\]

Compute functional derivatives:

\[
\frac{\delta T[N]}{\delta \pi^{ab}}=\frac{2N}{\sqrt q}\Big(\pi_{ab}-\tfrac12\pi\,q_{ab}\Big),\qquad
\frac{\delta T[N]}{\delta q_{ab}}=-\frac{N}{2\sqrt q}\Big(\pi^{cd}\pi_{cd}-\tfrac12\pi^2\Big)q^{ab}+\frac{N}{\sqrt q}\big(2\pi^{ac}\pi^b_{\ c}-\pi\,\pi^{ab}\big).
\]

For \$V{[}N{]}\$, use
\$\delta(\sqrt q,R)=\sqrt q,(G\textsuperscript{\{ab\}\delta q\_\{ab\}+D\_c\Theta}c)\$
with \$G\^{}\{ab\}\$ the Einstein tensor of \$q\$ and \$\Theta\^{}c\$ a
boundary term; thus

\[
\frac{\delta V[N]}{\delta q_{ab}}= -N\sqrt q\,(G^{ab}+\Lambda q^{ab}) + \text{(total divergences)},\qquad \frac{\delta V[N]}{\delta \pi^{ab}}=0.
\]

Insert in the Poisson bracket and integrate by parts. Divergences vanish
by (C2). Curvature terms combine with derivatives of \$N,M\$ to yield
the shift vector

\[
\beta^a[q;N,M]:=q^{ab}(N\partial_b M-M\partial_b N).
\]

A direct (standard) but lengthy cancellation gives

\[
\{H[N],H[M]\}=D[\vec\beta],\qquad \vec\beta=\beta^a\partial_a.
\]

Matter contributions are covariant and assemble into the same form (the
momentum constraint is the Noether charge for spatial diffeomorphisms),
hence do not alter the structure functions. \textbf{No central term}
appears: any putative c-number must be a boundary integral antisymmetric
in \$(N,M)\$, which vanishes under (C2). \$\square\$

\begin{center}\rule{0.5\linewidth}{0.5pt}\end{center}

\hypertarget{gauge-fixing-dirac-brackets-and-anomaly-exclusion}{%
\subsubsection{Gauge-fixing, Dirac brackets, and anomaly
exclusion}\label{gauge-fixing-dirac-brackets-and-anomaly-exclusion}}

Let \$\chi\^{}\mu=0\$ be a (cone-preserving) gauge,
e.g.~\textbf{harmonic/de Donder} on charts. The set
\$\{\mathcal H\_\perp,\mathcal H\_a,\chi\^{}\mu\}\$ is second-class with
invertible bracket matrix on the single-cone domain. The \textbf{Dirac
bracket} \$\{\cdot,\cdot\}\_D\$ equals the Poisson bracket on
\textbf{gauge-invariant} functionals. Since \$H{[}N{]},D{[}\vec N{]}\$
generate diffeomorphisms, their algebra (9.1) remains valid with
\$\{\cdot,\cdot\}\_D\$ when acting on gauge-invariant observables.

\textbf{Absence of anomalies at the Γ-limit.} The slow action is a
Γ-limit of second-order local functionals with \textbf{uniform symbol
bounds} (Ch. 4). Variational derivatives of the discretized constraints
converge (in the Mosco sense) to those of EH+YM; the symplectic form is
fixed. Hence the \textbf{structure functions converge to
\$q\^{}\{ab\}\$}, and the brackets converge to (9.1). A central
extension would require either (i) a cone flip (excluded by the
\textbf{\$W\_1\$ gap}; App. F.1) or (ii) higher-derivative remnants
violating H1/H5; both are ruled out by our hypotheses.

\[
\boxed{\text{The constraint algebra closes (no central terms) in the cone-preserving class and persists at the Γ-limit.}}
\tag{9.2}
\]

\begin{center}\rule{0.5\linewidth}{0.5pt}\end{center}

\hypertarget{path-measure-formulation-projective-construction-fkg-microcausality}{%
\subsection{9.2 Path-measure formulation (projective construction, FKG,
microcausality)}\label{path-measure-formulation-projective-construction-fkg-microcausality}}

We build a \textbf{cone-limited} probability measure on histories of the
slow fields (metric \$g\$ and gauge fields \$A\$) compatible with the
budgets and microcausality. The construction is by \textbf{projective
limits of cylinder measures} with explicit finite-dimensional marginals.

\hypertarget{indexing-of-cylinders-and-state-spaces}{%
\subsubsection{Indexing of cylinders and state
spaces}\label{indexing-of-cylinders-and-state-spaces}}

Let \$\mathscr P\$ be the directed set of finite \textbf{space--time
partitions} \$P\$: harmonic-chart coverings of compact slabs
\$K\times{[}t\_0,t\_1{]}\$ with mesh \$(\ell,\Delta t)\$ respecting the
single-cone bounds (principal symbol within fixed
\${[}\lambda,\Lambda{]}\$). For \$P\in\mathscr P\$, define the finite
space

\[
\mathcal X_P:=\{(g_v,A_v)_{v\in V(P)}:\ \text{component values in harmonic coordinates, obeying cone and gauge bounds}\},
\]

equipped with the product Borel \$\sigma\$-algebra and a reference
product measure \$\nu\_P\$ (Gaussian blocks reflecting the quadratic
part of the gauge-fixed action; see below).

\hypertarget{local-weights-and-finite-dimensional-marginals}{%
\subsubsection{Local weights and finite-dimensional
marginals}\label{local-weights-and-finite-dimensional-marginals}}

On each \$P\$, define the \emph{discretized action} (EH+YM Γ-compatible;
Ch. 4/Appendix D) with gauge-fixing and cone penalty,

\[
S_P(g,A)=\sum_{c\in C(P)} \Big[\,\tfrac12\,\langle (g,A),\mathbb A_c (g,A)\rangle - \langle J_c,(g,A)\rangle + U_c(g,A)\,\Big]\ +\ V_P(g,A),
\]

where:

\begin{itemize}
\tightlist
\item
  \$\mathbb A\_c\$ are \textbf{uniformly parameter-elliptic} local
  operators (principal symbol bounds inherited from the single-cone
  class);
\item
  \$U\_c\$ collects \textbf{lower-order} Γ-local terms;
\item
  \$V\_P\$ aggregates \textbf{budget terms} that are Γ-local and
  \textbf{attractive} in the discretization (see FKG below): throughput
  and complexity contributions enter as convex quadratics; leakage
  enters as a convex weight aligned with the pointer basis (Ch. 3/7).
\end{itemize}

Define the \textbf{finite-dimensional probability} on
\$(\mathcal X\_P,\mathcal B\_P)\$ by

\[
\mu_P(\mathrm d x):=\frac{1}{Z_P}\,\exp\!\big(\!-S_P(x)\big)\,\mathbf 1_{\text{single-cone}}(x)\ \nu_P(\mathrm d x),\qquad Z_P:=\int e^{-S_P}\mathbf 1_{\text{cone}}\ \mathrm d\nu_P.
\tag{9.3}
\]

\hypertarget{cylinder-consistency-projective-system}{%
\subsubsection{Cylinder consistency (projective
system)}\label{cylinder-consistency-projective-system}}

If \$P\preceq P'\$ (refinement), write
\$\pi\_\{P'!,P\}:\mathcal X\_\{P'\}!\to\mathcal X\_P\$ for
restriction/averaging on cells. We define \$S\_P\$ by \emph{backward
induction} from fine to coarse scales:

\[
e^{-S_P(x)}\ \propto\ \int_{\pi_{P'\!,P}^{-1}(x)}\!\exp\!\big(\!-S_{P'}(x')\big)\ \nu_{P'}(\mathrm d x'\mid \pi_{P'\!,P}=x).
\tag{9.4}
\]

This is a \textbf{local RG/coarse-graining} defining the coarse
potential as a log-partition over refined variables. By construction,

\[
(\pi_{P'\!,P})_\#\,\mu_{P'}=\mu_P\qquad\text{for all }P\preceq P'.
\tag{9.5}
\]

Hence \$\{\mu\_P\}\_\{P\in\mathscr P\}\$ is a \textbf{projective
family}. Tightness follows from uniform Gårding/coercivity (Ch. 4),
giving \textbf{Kolmogorov extension}:

\[
\boxed{\ \exists!\ \mu\ \text{ on }\ \mathcal X:=\varprojlim \mathcal X_P\ \text{ with }(\pi_P)_\#\mu=\mu_P\ \forall P.\ }
\tag{9.6}
\]

\hypertarget{fkgpositive-association-under-attractive-discretization}{%
\subsubsection{FKG/positive association under attractive
discretization}\label{fkgpositive-association-under-attractive-discretization}}

On each \$P\$, the potential is a sum of \textbf{convex on-site} terms
and \textbf{pairwise attractive} interactions (mixed second derivatives
\$\le 0\$ in the partial order induced by componentwise increase within
the cone window). This can be arranged by:

\begin{itemize}
\tightlist
\item
  harmonic gauge (quadratic principal part convex);
\item
  YM energy densities as convex quadratics locally;
\item
  budget add-ons chosen \textbf{convex} in the pointer-aligned
  coordinates (leakage: quadratic in \$W\^{}\{1/2\}\$-weighted
  amplitudes; complexity/throughput: convex quadratics).
\end{itemize}

By \textbf{Holley's criterion}, \$\mu\_P\$ satisfies \textbf{FKG}.
Projective limits preserve association on cylinder events; thus for
increasing cylinder functions \$f,g\$,

\[
\mathbb E_\mu[f\,g]\ \ge\ \mathbb E_\mu[f]\ \mathbb E_\mu[g].
\tag{9.7}
\]

\hypertarget{microcausality-lr-type-mixing-bound-at-the-measure-level}{%
\subsubsection{Microcausality (LR-type mixing bound at the measure
level)}\label{microcausality-lr-type-mixing-bound-at-the-measure-level}}

Time-slice the partition \$P\$ into levels \$\{t\_k\}\$. The
coarse-graining (9.4) can be implemented via \textbf{Markov transfer
kernels} \$K\_\{k\to k+1\}(x\_\{t\_k\},\mathrm d x\_\{t\_\{k+1\}\})\$
induced by the local quadratic principal symbol, with
\textbf{finite-speed} propagation constant \$v\_\ast\$ determined by the
single-cone bounds. The corresponding Dobrushin influence coefficients
\$\alpha(A\to B)\$ between spatial blocks \$A\$ at \$t\_k\$ and \$B\$ at
\$t\_\{k+1\}\$ obey

\[
\alpha(A\to B)\ \le\ C\,\exp\!\Big(-\gamma\,[\,\mathrm{dist}(A,B)-v_\ast\Delta t\,]_+\Big),
\tag{9.8}
\]

uniformly in \$P\$, for some \$C,\gamma\textgreater0\$ (cone-limited
Lieb--Robinson-type estimate at the \textbf{kernel} level; cf.~U.LR and
the principal-symbol lemma). Standard Dobrushin/cluster-mixing then
yields, for cylinder observables \$F\_A,G\_B\$ localized in
spacelike-separated regions \$A,B\$,

\[
\big|\mathrm{Cov}_\mu(F_A,G_B)\big|\ \le\ C'\,\|F_A\|_{\rm Lip}\,\|G_B\|_{\rm Lip}\ \exp\!\Big(-\gamma'\,[\,\mathrm{dist}(A,B)-v_\ast\,\Delta t\,]_+\Big).
\tag{9.9}
\]

Thus \textbf{microcausality} (no superluminal statistical influence)
holds at the measure level.

\hypertarget{mixed-fastslow-characteristic-functionals}{%
\subsubsection{Mixed fast--slow characteristic
functionals}\label{mixed-fastslow-characteristic-functionals}}

Let \$\mathcal G\$ be the fast-sector quasi-local algebra. For a
cylinder field \$(g,A)\mapsto J(g,A)\$ coupled to a stationary GKSL fast
state with \textbf{LR bounds} (U4/U.LR), define the mixed characteristic
functional

\[
\Xi[\theta]:=\int_{\mathcal X}\!\exp\!\Big(i\,\theta\cdot J(g,A)\Big)\ \mu(\mathrm d g\,\mathrm d A).
\]

The LR-type bound (9.9) and U.LR imply the \textbf{same cone-limited
decay} for mixed covariances; variational differentiation of \$\Xi\$
recovers the coupled Euler--Lagrange equations (Ch. 5) by standard
Gibbs/DLR calculus (Appendix E), with budgets entering through \$V\_P\$.

\begin{center}\rule{0.5\linewidth}{0.5pt}\end{center}

\hypertarget{summary-of-chapter-9}{%
\subsubsection{Summary of Chapter 9}\label{summary-of-chapter-9}}

\begin{enumerate}
\def\labelenumi{\arabic{enumi}.}
\tightlist
\item
  The \textbf{constraint algebra closes} in the cone-preserving class
  with the standard structure functions \$q\^{}\{ab\}\$ and \textbf{no
  central extensions}; the result \textbf{persists at the Γ-limit} by
  stability of variational derivatives and symbol bounds.
\item
  The \textbf{path measure} for the slow fields exists as a
  \textbf{projective limit} of explicit finite-dimensional marginals; it
  satisfies \textbf{FKG} under an attractive discretization and obeys
  \textbf{microcausal LR-type bounds} uniform across scales.
\item
  The construction is \textbf{budget-compatible} (convex, Γ-local
  add-ons) and integrates consistently with the fast sector (GKSL with
  LR), completing the quantum-geometry layer of the selection framework.
\end{enumerate}

\hypertarget{chapter-10-coherence-nucleation-cosmology-rigorous-edition}{%
\section{Chapter 10 --- Coherence Nucleation ⇒ Cosmology (Rigorous
Edition)}\label{chapter-10-coherence-nucleation-cosmology-rigorous-edition}}

\begin{quote}
\textbf{Scope.} We construct FRW cosmology from \textbf{coherence-first}
principles: a \textbf{seed ensemble} produces a \textbf{supercritical
coherence cascade} when a computable \textbf{reproduction number}
exceeds unity. Inside the cascade, the slow Γ‑limit reduces to FRW with
quantitative smoothing/flatness bounds. We fix the \textbf{information
framework} to a class of divergences, show \textbf{RG‑stability} in the
tiling scale, formalize the \textbf{risk‑sensitive} envelope and
\textbf{commuting limits}, and prove \textbf{conservation} of the
effective stress tensor in the slow sector. Bounces arise precisely when
leakage drives \(w<-1/3\) under pointer freezing.
\end{quote}

\begin{center}\rule{0.5\linewidth}{0.5pt}\end{center}

\hypertarget{hypotheses-information-class-notation-dimensionuniform}{%
\subsection{10.0 Hypotheses, information class, notation
(dimension‑uniform)}\label{hypotheses-information-class-notation-dimensionuniform}}

We adopt the quasi‑local C* and fast GKSL framework of Chs. 3--5. Space
is tiled by hypercubes (``tiles'') of side \(\ell>0\); write
\(\mathbb T_\ell\) for the tile set; \(\partial A\) the boundary
\textbf{area} of a union \(A\subset\mathbb T_\ell\). The spatial
dimension is \(d\ge2\) (observationally \(d=3\)); all proofs below are
dimension‑uniform, with \(d\) only in packing constants.

\textbf{Information‑measure class \(\mathsf{IM}(\tau)\).} A divergence
\(D\) lies in \(\mathsf{IM}(\tau)\) if: (i) \textbf{data processing}
holds for GKSL channels and the pinching map \(\mathcal P^W\); (ii)
\textbf{LR quasi‑factorization} holds with constant \(C_q(\tau)\); (iii)
a \textbf{Dobrushin} coefficient satisfies
\(\delta_D(\mathcal P^W\!\circ\!\Phi_t)\le e^{-\gamma_W t}\) at the KKT
point; (iv) on calibrated sublevels there exist constants
\(m(\tau),M(\tau)\) with \(m D_2\le D\le M D_2\) and similarly vs.~KL.
(Sandwiched Rényi \(1\le\alpha\le2\) and KL are in
\(\mathsf{IM}(\tau)\).)

\textbf{(H10.1) Fast dynamics \& LR.} On each finite union \(A\), the
fast sector evolves by a GKSL semigroup \(\Phi_t^A=e^{t\mathcal L^A}\)
obeying Lieb--Robinson locality with velocity \(v_{\rm LR}\) and profile
\(f_{\rm LR}\), and respecting the three budget quadratics (Ch. 3).

\textbf{(H10.2) Pointer alignment \& Dirichlet linearity.} The leakage
quadratic is \textbf{left‑Dirichlet} in the pointer weight \(W\) and
\textbf{orthogonally additive} in its spectral decomposition;
\(\mathcal P^W\) is a \(D\)-contraction for any \(D\in\mathsf{IM}\).

\textbf{(H10.3) Throughput normalization.}
\(\hbar=\lambda_{\rm th}^{-1}\) (Ch. 3), so the HS metric prices motion
along unitary orbits.

\textbf{(H10.4) Slow Γ‑limit.} Under localized Mosco/Attouch, first
variations commute with \(\varepsilon\to0\) and the slow sector
satisfies
\(G_{\mu\nu}+\Lambda g_{\mu\nu}=8\pi G\,T^{\rm eff}_{\mu\nu},\qquad \nabla^\mu T^{\rm eff}_{\\mu\nu}=0,\)
with calibrated multipliers \((G,\Lambda)\).

\textbf{Light‑cone time.} \(\tau_{\rm LR}(\ell)=\ell/v_{\rm LR}\). The
adjacency graph \(\mathcal G_\ell\) connects face‑sharing tiles.

\begin{center}\rule{0.5\linewidth}{0.5pt}\end{center}

\hypertarget{seed-ensemble-poke-process}{%
\subsection{10.0.1 Seed ensemble (poke
process)}\label{seed-ensemble-poke-process}}

Pokes arrive as a \textbf{Poisson random measure} on spacetime with rate
\(\lambda_0\), amplitudes with \textbf{sub‑exponential tails} (parameter
\(\theta>0\)), and LR‑compatible support (no super‑cone events). A seed
tile is \textbf{aligned} if its post‑poke state is
\(\mathcal P^W\)-close (in any \(D\in\mathsf{IM}\)) to a pointer block.
For \(\gamma_W>0\) and finite \(\alpha_{\rm leak}\), aligned seeds occur
with positive probability.

\begin{center}\rule{0.5\linewidth}{0.5pt}\end{center}

\hypertarget{tile-balance-and-the-reproduction-number}{%
\subsection{10.1 Tile balance and the reproduction
number}\label{tile-balance-and-the-reproduction-number}}

For a tile union \(A\), let \(\mathsf C_t(A)\) be the \textbf{coherence
content} at time \(t\) measured by any \(D\in\mathsf{IM}\). Over
\(\Delta t\le \tau_{\rm LR}(\ell)\), three effects act on a seed tile
\(T\):

\begin{enumerate}
\def\labelenumi{\arabic{enumi}.}
\item
  \textbf{Leakage load}:
  \(L(\ell,\Delta t)\le C_L\,\alpha_{\rm leak}\,\tfrac{\Delta t}{\ell}.\)
\item
  \textbf{Pointer alignment}:
  \$\eta\_\{\rm align\}(\Delta t)=1-\delta\emph{D(\mathcal P\^{}W!\circ!\Phi}\{\Delta t\})\in(0,1{]}
  \$ with \(1-\eta_{\rm align}\le e^{-\gamma_W\Delta t}.\)
\item
  \textbf{Neighbor recruitment}:
  \(M(\ell,\Delta t)\le C_M\,(\tfrac{v_{\rm LR}\Delta t}{\ell})^d\)
  (packing; App. I.1).
\end{enumerate}

\textbf{Definition 10.1 (Reproduction number).}
\(\mathcal R_{\rm coh}(\ell,\Delta t)=\eta_{\rm align}(\Delta t)\,(1-L(\ell,\Delta t))\,M(\ell,\Delta t).\)

\textbf{Proposition 10.2 (Optimized one‑step threshold).} At
\(\Delta t=\ell/v_{\rm LR}\),
\(\mathcal R_{\rm coh}\ \ge\ \underbrace{(1-e^{-\gamma_W\ell/v_{\rm LR}})}_{\eta}\,\underbrace{(1-C_L\alpha_{\rm leak}/v_{\rm LR})}_{1-L}\,\underbrace{C_M}_{M}.\)
Thus \(\mathcal R_{\rm coh}>1\) if
\((1-e^{-\gamma_W\ell/v_{\rm LR}})C_M>(1-C_L\alpha_{\rm leak}/v_{\rm LR})^{-1}.\)

\begin{center}\rule{0.5\linewidth}{0.5pt}\end{center}

\hypertarget{nucleation-and-nonrecurrence}{%
\subsection{10.2 Nucleation and
non‑recurrence}\label{nucleation-and-nonrecurrence}}

\textbf{Theorem 10.3 (Nucleation ⇒ supercritical cascade).} If
\(\mathcal R_{\rm coh}(\ell,\Delta t)>1\) for some \((\ell,\Delta t)\),
an aligned seed produces (in the worst‑case envelope) a \textbf{growing
cluster} of aligned tiles whose boundary advances at positive asymptotic
speed. The occupied set stochastically dominates a \textbf{supercritical
Galton--Watson / first‑passage} process on \(\mathcal G_\ell\) with mean
offspring \(>1\). \emph{Proof idea:} LR quasi‑factorization and
orthogonal additivity in \(W\) yield FKG/Harris positive association;
dominate by a branching process of mean \(\mathcal R_{\rm coh}>1\) (App.
I.3).

\textbf{Theorem 10.4 (Non‑recurrence in aligned media).} In an already
pointer‑aligned bath, \(\mathcal R_{\rm coh}^{\rm ambient}\le1\) for all
\(\Delta t\le\tau_{\rm LR}\); seeds decohere or are absorbed.
\emph{Reason:} DPI for \(\mathcal P^{W_{\rm env}}\circ\Phi_{\Delta t}\)
gives \(\eta_{\rm align}^{\rm env}\le1\) with equality only on the
ambient block; leakage is at least that to the full bath.

\begin{center}\rule{0.5\linewidth}{0.5pt}\end{center}

\hypertarget{rgstability-in-the-tiling-scale-ell}{%
\subsection{\texorpdfstring{10.2′ RG‑stability in the tiling scale
\(\ell\)}{10.2′ RG‑stability in the tiling scale \textbackslash ell}}\label{rgstability-in-the-tiling-scale-ell}}

Rescale \(\ell\mapsto\lambda\ell\) with \(\lambda\in(\tfrac12,2)\).
Define the \textbf{budget reparametrization}
\(\alpha_{\rm leak}\mapsto \alpha_{\rm leak}^{(\lambda)}:=\lambda^{-1}\alpha_{\rm leak},\quad \alpha_{\rm th}\mapsto\alpha_{\rm th}^{(\lambda)}:=\alpha_{\rm th},\quad \alpha_{\rm cx}\mapsto\alpha_{\rm cx}^{(\lambda)}:=\alpha_{\rm cx},\)
so that \(\rho_{\rm leak}\propto \alpha_{\rm leak}\ell^{-1}\) is
invariant and the other budgets are unchanged to leading order. Then to
first order in \(|\lambda-1|\), \textbf{observable} quantities
\(H(a),\,w(a),\,\Omega_k\) are invariant up to controlled
\(\mathcal O(|\lambda-1|)\) shifts absorbed in calibration constants.
(App. J.4 gives a beta‑function view.)

\textbf{Operational Planck tile.} If \(v_{\rm LR}\le c\), adopt
\(\tilde\ell_P:=\ell_P (c/v_{\rm LR})^{1/2}\) as the Planck‑window
coarse‑graining scale; all Planck‑window statements hold with
\(\ell\approx\tilde\ell_P\).

\begin{center}\rule{0.5\linewidth}{0.5pt}\end{center}

\hypertarget{budget-quadratics-equations-of-state-with-robustness}{%
\subsection{10.3 Budget quadratics ⇒ equations of state (with
robustness)}\label{budget-quadratics-equations-of-state-with-robustness}}

Let the calibrated quadratics be: throughput
\(Q_{\rm th}(A)=\langle[H,A],[H,A]\rangle_{\rm HS}\); complexity
\(Q_{\rm cx}(A)=\langle\nabla_\xi A,\nabla_\xi A\rangle\) (Ad‑invariant
Hilbertian with correlation length \(\xi\)); leakage
\(Q_{\rm leak}(A)=\|(I-\mathcal P^W)A\|^2\) (Dirichlet in \(W\)). Denote
multipliers \(\alpha_{\rm th},\alpha_{\rm cx},\alpha_{\rm leak}\).

\textbf{Lemma 10.5 (Throughput).} With \(\hbar=\lambda_{\rm th}^{-1}\):
\(\rho_{\rm th}=c_{\rm th}\,\alpha_{\rm th}\,\Omega^2\) and
\(p_{\rm th}=\tfrac13\rho_{\rm th}\) (ultra‑relativistic sector).

\textbf{Lemma 10.6 (Complexity).} With tile‑uniform LSI/Poincaré at
scale \(\xi\): \(\rho_{\rm cx}=c_{\rm cx}\,\alpha_{\rm cx}\,\xi^{-2}\),
and \(p_{\rm cx}=-\partial\rho_{\rm cx}/\partial (3\log a)\).

\textbf{Lemma 10.7 (Leakage).} With left‑Dirichlet \(Q_{\rm leak}\) and
pinching gap \(\gamma_W\):
\(\rho_{\rm leak}=c_{\rm leak}\,\alpha_{\rm leak}\,A/V\sim c_{\rm leak}\,\alpha_{\rm leak}\,\ell^{-1}\),
and \(w_{\rm leak}\in[-1,1/3]\). Endpoint \(w\to-1\) holds when the
leakage block is pointer‑frozen (log‑Sobolev gap \(\ge\gamma_0>0\));
\(w\to1/3\) holds if it thermalizes within the LR cone.

\textbf{Stability under representatives.} If \(Q\) and \(Q'\) are
norm‑equivalent on the admissible class (constants \(m,M\)), the
scalings and \(w\) persist; only \(c_\bullet\) rescale within \([m,M]\).

\begin{center}\rule{0.5\linewidth}{0.5pt}\end{center}

\hypertarget{frw-emergence-and-quantitative-nearflatness}{%
\subsection{10.4 FRW emergence and quantitative
near‑flatness}\label{frw-emergence-and-quantitative-nearflatness}}

Inside the supercritical domain \(\Omega_{\rm nuc}\), LR locality and
\(D\)-contraction imply decay of anisotropic stress and vorticity over
large boxes \(B_R\):
\(\|\pi_{ij}\|+\|\omega_i\|\ \le\ C\,\frac{|\partial B_R|}{|B_R|}\,\Phi\!\left(\tfrac{\alpha_{\rm leak}}{\alpha_{\rm th}},\tfrac{\alpha_{\rm leak}}{\alpha_{\rm cx}}\right)\xrightarrow{R\to\infty}0.\)
Thus the slow Γ‑limit reduces to FRW with
\(p_{\rm eff}=\tfrac13\rho_{\rm eff}+o(1)\) initially and standard
Friedmann--Raychaudhuri holds.

\textbf{Proposition 10.8 (Near‑flatness inequality).} For any comoving
exhaustion \(B_R\) and horizon time \(t_H(R)\),
\(\sup_{t\le t_H(R)}\frac{|K|}{a^2}\ \le\ C_{\rm flat}\,\frac{\alpha_{\rm leak}\,A(B_R)}{V(B_R)\,H^2}+o_R(1).\)
Equivalently,
\(|\Omega_k|\le C_{\rm flat}(\alpha_{\rm leak}/\alpha_{\rm th})(A/V)(\Omega_r/H_0^2)+o_R(1).\)

\begin{center}\rule{0.5\linewidth}{0.5pt}\end{center}

\hypertarget{risksensitive-envelope-epiconvergence-and-commuting-limits}{%
\subsection{10.5 Risk‑sensitive envelope; epi‑convergence and commuting
limits}\label{risksensitive-envelope-epiconvergence-and-commuting-limits}}

Let \((\mathcal P,\mathcal F,\mu_\beta)\) be the poke space with
exponentially tight tails. Define
\(\mathrm{CL}(A,\Phi)=\inf_{p\in\mathcal P}L(A,\Phi;p)\), l.s.c. in
\((A,\Phi)\).

\textbf{Lemma 10.9 (Envelope epi‑convergence).} As \(\beta\to\infty\)
with \(\mu_\beta\Rightarrow\mu_\infty\) supported on the admissible
cone,
\(\lim_{\beta\to\infty}\sup_A \mathrm{CL}_\beta(A,\Phi)=\sup_A\operatorname*{ess\,inf}_{p\sim\mu_\infty}L(A,\Phi;p),\)
and maximizers exist on calibrated sublevels.

\textbf{Lemma 10.10 (Commuting limits).} On finite boxes, taking
\(\beta\to\infty\) then \(\varepsilon\to0\) (Γ‑limit) yields the same
envelope as any interleaving respecting LR locality.

\begin{center}\rule{0.5\linewidth}{0.5pt}\end{center}

\hypertarget{conservation-in-the-slow-sector-noether-ux3b3limit}{%
\subsection{10.6 Conservation in the slow sector (Noether +
Γ‑limit)}\label{conservation-in-the-slow-sector-noether-ux3b3limit}}

Let the slow functional be \(\mathcal S[g;\alpha_\bullet]\) with budget
constraints enforced by multipliers. Variations \(\delta g\) obeying
compact support and cone‑preserving gauge produce
\(\delta\mathcal S=\tfrac12\int (T^{\rm eff}_{\mu\nu})\,\delta g^{\mu\nu}\,\sqrt{-g}\,d^4x,\qquad T^{\rm eff}=T^{\rm vis}+T^{\rm fast}.\)
Cone‑preserving diffeomorphisms give \(\nabla\!\cdot T^{\rm eff}=0\).
Leakage only exchanges budget \textbf{within} \(T^{\rm fast}\); the
\textbf{sum} is conserved in the slow sector.

\begin{center}\rule{0.5\linewidth}{0.5pt}\end{center}

\hypertarget{bigbang-vs.-coherence-bounce-energy-conditions}{%
\subsection{10.7 Big‑Bang vs.~coherence bounce; energy
conditions}\label{bigbang-vs.-coherence-bounce-energy-conditions}}

Let \(\rho=\rho_{\rm th}+\rho_{\rm cx}+\rho_{\rm leak}\) with
\(w_{\rm th}\approx\tfrac13\), \(w_{\rm cx}\in[0,1]\),
\(w_{\rm leak}\in[-1,1/3]\).

\textbf{Theorem 10.11 (Bounce criterion).} If
\(\rho_{\rm leak}+3p_{\rm leak}<-(\rho_{\rm th}+\rho_{\rm cx}+3p_{\rm th}+3p_{\rm cx})\)
over an interval, then \(\ddot a>0\) and a \textbf{minimum scale factor}
\(a_{\min}>0\) occurs. (Raychaudhuri.)

\textbf{Lemma 10.12 (SEC/ANEC status).} The effective fluid need not
satisfy SEC; pointer‑frozen leakage (\(w\to-1\)) violates SEC while LR
microcausality holds. Hence singularity theorems' hypotheses fail in the
bounce branch.

\textbf{Sequestered windows; anisotropy seeds.} Near horizons/interiors,
leakage throttling (\(L\downarrow\)) with \(\gamma_W\uparrow\) can allow
\(\mathcal R_{\rm coh}>1\) \textbf{inside} a causal window, producing
daughter domains that are \textbf{sequestered} (domain‑wall
containment). Front roughness induces a two‑point function
\(\langle\delta\rho(\mathbf x)\delta\rho(\mathbf y)\rangle\sim r^{-(d-1)}\)
inside the LR horizon, giving a near scale‑invariant spectrum in \(d=3\)
after horizon crossing (cf.~Sim‑S2).

\begin{center}\rule{0.5\linewidth}{0.5pt}\end{center}

\hypertarget{calibration-protocol-constants-falsifiers}{%
\subsection{10.8 Calibration protocol; constants;
falsifiers}\label{calibration-protocol-constants-falsifiers}}

Constants \(C_L,\gamma_W,v_{\rm LR},C_M,c_\bullet\) are fixed by App.
J.2; predictions are stable under norm‑equivalent representatives (App.
J.1). \textbf{Falsifiers:} (i) LR violation; (ii) failure of Lemma 10.9
(non‑attainment); (iii) robust violation of Prop. 10.8 across multiple
\(B_R\) at fixed calibration; (iv) in‑medium supercritical cascades
(violates Thm 10.4); (v) need for a \textbf{fourth} budget.

\begin{center}\rule{0.5\linewidth}{0.5pt}\end{center}

\hypertarget{chapter-11-planck-window-budgets-area-bounds-and-operational-scales}{%
\section{Chapter 11 --- Planck Window: Budgets, Area Bounds, and
Operational
Scales}\label{chapter-11-planck-window-budgets-area-bounds-and-operational-scales}}

\begin{quote}
\textbf{Scope.} We fix Planck‑window hypotheses, prove
\textbf{short‑time} and \textbf{global‑in‑time} tile‑area bounds for
cross‑boundary information in any \(D\in\mathsf{IM}(\tau)\), and relate
the tiling scale to the operational Planck length \(\tilde\ell_P\) when
\(v_{\rm LR}\le c\).
\end{quote}

\begin{center}\rule{0.5\linewidth}{0.5pt}\end{center}

\hypertarget{planckwindow-hypotheses}{%
\subsection{11.1 Planck‑window
hypotheses}\label{planckwindow-hypotheses}}

\begin{itemize}
\tightlist
\item
  \textbf{(P1) Local dimension control.} On any tile \(T\) of side
  \(\ell\), \(d_{\rm loc}(\ell)\lesssim (\ell/\ell_*)^{\kappa}\) for
  microscopic \(\ell_*\), \(\kappa>0\).
\item
  \textbf{(P2) LR microcausality.} GKSL obeys LR bounds with velocity
  \(v_{\rm LR}(\ell)\) uniformly finite across tiles.
\item
  \textbf{(P3) Dirichlet linearity.} Leakage is left‑Dirichlet in \(W\)
  and orthogonally additive in its spectral decomposition.
\item
  \textbf{(P4) Throughput normalization.}
  \(\hbar=\lambda_{\rm th}^{-1}\) (local blocks and inductive limit).
\item
  \textbf{(P5) Information class.} Use \(D\in\mathsf{IM}(\tau)\)
  throughout.
\end{itemize}

\textbf{Operational Planck tile.} If \(v_{\rm LR}\le c\), the
Planck‑window tile is \(\tilde\ell_P=\ell_P(c/v_{\rm LR})^{1/2}\); use
\(\ell\approx\tilde\ell_P\) in area bounds.

\begin{center}\rule{0.5\linewidth}{0.5pt}\end{center}

\hypertarget{shorttime-tilearea-bound-single-cone}{%
\subsection{11.2 Short‑time tile‑area bound (single
cone)}\label{shorttime-tilearea-bound-single-cone}}

For a union \(A\) of tiles and times
\(t\le c_{\rm time}\,\ell/v_{\rm LR}\), the GKSL evolution at a KKT
point satisfies, for any \(D\in\mathsf{IM}(\tau)\),

\[
\boxed{\ I_D(A{:}\bar A; t)\ \le\ C_{\rm info}(D)\,\frac{|\partial A|}{\ell}\,\Phi\!\left(\frac{\alpha_{\rm leak}}{\alpha_{\rm th}},\frac{\alpha_{\rm leak}}{\alpha_{\rm cx}}\right)\ },
\]

with
\(C_{\rm info}(D)=C_0(D)\,(1-e^{-\gamma_W t})/(1-e^{-\gamma_* t})\).
Here \(\gamma_*\) is the slowest mixing rate in \(A\);
\(c_{\rm time},C_0(D)\) depend only on LR and tile LSI.

\emph{Proof sketch:} LR quasi‑factorization + pinching contraction +
orthogonal additivity in \(W\). Constants are \textbf{calibrated}, not
universal.

\begin{center}\rule{0.5\linewidth}{0.5pt}\end{center}

\hypertarget{globalized-area-bound-all-times}{%
\subsection{11.2′ Globalized area bound (all
times)}\label{globalized-area-bound-all-times}}

For any \(t\ge0\),

\[
I_D(A{:}\bar A;t)\ \le\ \frac{|\partial A|}{\ell}\Big[C_1(D)\big(1-e^{-\gamma_W t}\big)+C_2(D)\!\int_0^t e^{-\gamma_* (t-s)}\,\Xi(s)\,ds\Big],
\]

where \(\Xi\) depends only on LR profile and tile LSI constants.
(Iterated cone decomposition + Grönwall on \(I_D\).)

\textbf{Consequence.} At \(\ell\approx\tilde\ell_P\), cross‑boundary
predictive content is area‑limited with coefficients controlled by
budget ratios and \((\gamma_W,v_{\rm LR},\text{LSI})\) fixed at
calibration; stability holds under norm‑equivalent representatives (App.
J.1).

\begin{center}\rule{0.5\linewidth}{0.5pt}\end{center}

\hypertarget{inflationary-embedding-optional}{%
\subsection{11.3 Inflationary embedding
(optional)}\label{inflationary-embedding-optional}}

If an inflaton exists, coherence supplies \textbf{dissipative slow‑roll}
corrections: leakage shifts friction by
\(\Delta\Gamma\propto\alpha_{\rm leak}\), throughput enforces a kinetic
penalty \(\propto\alpha_{\rm th}\); LR locality is preserved. Pure
coherence‑smoothing (§10.4) remains viable with \(T^{\rm vis}\) absent.

\begin{center}\rule{0.5\linewidth}{0.5pt}\end{center}

\hypertarget{chapter-12-dark-sector-from-coherence-minimal-candidates-inequalities}{%
\section{Chapter 12 --- Dark Sector from Coherence (Minimal Candidates +
Inequalities)}\label{chapter-12-dark-sector-from-coherence-minimal-candidates-inequalities}}

\begin{quote}
\textbf{Scope.} We present three \textbf{minimal} dark‑matter candidates
as coherence‑selected patterns. Each couples gravitationally via the
slow sector, is budget‑suppressed in the fast/pointer sector, and admits
\textbf{quantitative inequalities} mapping viability directly to
calibrated \textbf{\(\alpha\)-ratios} and constants.
\end{quote}

\begin{center}\rule{0.5\linewidth}{0.5pt}\end{center}

\hypertarget{hiddenpointer-hp-sector-cdmlike}{%
\subsection{12.1 Hidden‑pointer (HP) sector ---
CDM‑like}\label{hiddenpointer-hp-sector-cdmlike}}

\textbf{Construction.} Let \(W=\sum_a w_a P_a\); a hidden block \(P_D\)
with \(w_D\gg w_{\rm vis}\) suppresses leakage for operators on \(P_D\).
Consider a HP U(1) sector with field strength \(F^D\) contributing only
gravitationally at the Γ‑limit.

\textbf{Coldness inequality.} The effective sound speed obeys
\(c_s^2\ \le\ \varepsilon_{\rm cold}:=C_{\rm gap}\Big(\frac{T}{\Delta_{\rm HP}}\Big)^2,\)
with pointer gap \(\Delta_{\rm HP}\). Require \(c_s^2\lesssim10^{-6}\)
by matter--radiation equality.

\textbf{Discriminants.} (i) \textbf{Growth‑index shift} \(\Delta\gamma\)
from residual \(c_s^2\neq0\); (ii) \textbf{absence of isocurvature} by
pointer isolation. Failure of either falsifies HP under fixed
calibration.

\begin{center}\rule{0.5\linewidth}{0.5pt}\end{center}

\hypertarget{phasemode-axion-px-misalignment-coldwarm}{%
\subsection{12.2 Phase‑mode axion (PX) --- misalignment
cold/warm}\label{phasemode-axion-px-misalignment-coldwarm}}

\textbf{Construction.} A global \(U(1)\) phase, protected by pointer
alignment, yields a light \textbf{phase mode} \(\theta\) with
\(V(\theta)=\Lambda_{\rm cx}^4\,(1-\cos\theta),\qquad \Lambda_{\rm cx}^4\propto\alpha_{\rm cx},\)
kinetic term fixed by \(\hbar=\lambda_{\rm th}^{-1}\). Misalignment
abundance
\(\rho_{\rm PX}(a_0)=\tfrac12 f_a^2 m_a^2\,\theta_0^2\,\mathcal T\big(m_a/H\big),\)
with \((m_a,f_a)\) constrained by \((\alpha_{\rm cx},\alpha_{\rm th})\)
calibration.

\textbf{Warmness bound.} Ly‑\(\alpha\) and LSS require free‑streaming
below the observed cutoff, which translates to a \textbf{lower} bound on
\(m_a\) (given \(f_a\)) and excludes parts of
\((\alpha_{\rm cx},\alpha_{\rm th})\) space.

\textbf{Discriminant.} A \textbf{correlated late‑time viscosity} from
\(Q_{\rm cx}\) must align with the allowed \((m_a,f_a)\); mismatch
falsifies PX.

\begin{center}\rule{0.5\linewidth}{0.5pt}\end{center}

\hypertarget{sterilemixing-neutrino-sn-warm}{%
\subsection{12.3 Sterile‑mixing neutrino (SN) ---
warm}\label{sterilemixing-neutrino-sn-warm}}

\textbf{Construction.} A minimal seesaw‑like block with
derivation‑pricing + leakage alignment suppresses visible mixing by
\(\sin^2(2\theta)\ \sim\ C_{\rm mix}\Big(\frac{\alpha_{\rm th}}{\alpha_{\rm leak}}\Big)^2.\)
Freeze‑in from pointer‑misaligned dissipators sets the relic abundance.

\textbf{Free‑streaming inequality.} Require
\(\lambda_{\rm FS}\ \simeq\ \int \frac{v(a)}{a^2 H(a)}\,da\ \le\ 0.1\ \text{Mpc},\)
which imposes a \textbf{lower} bound on mixing mapped to
\(\alpha_{\rm th}/\alpha_{\rm leak}\); combine with X‑ray line limits to
confine the viable band.

\begin{center}\rule{0.5\linewidth}{0.5pt}\end{center}

\hypertarget{shared-falsifiers-calibration-stability-baryogenesis-routes}{%
\subsection{12.4 Shared falsifiers; calibration stability; baryogenesis
routes}\label{shared-falsifiers-calibration-stability-baryogenesis-routes}}

All candidates share a single calibration
\((\alpha_{\rm th},\alpha_{\rm cx},\alpha_{\rm leak})\). A persistent
need for a \textbf{fourth} budget, or lab evidence of
\textbf{basis‑invariant} decoherence contradicting pointer alignment,
invalidates the constructions.

\textbf{Proposition 12.1 (Calibration stability).} If
\(\langle\cdot,\cdot\rangle\) and \(\langle\cdot,\cdot\rangle'\) are
norm‑equivalent on the admissible class with
\(m\|X\|^2\le\|X\|'^{2}\le M\|X\|^2\), then multipliers satisfy
\(m\,\alpha_\bullet\le\alpha'_\bullet\le M\,\alpha_\bullet\)
component‑wise; predictions depending on \textbf{ratios}
\(\alpha_i/\alpha_j\) are invariant up to \([m,M]\).

\textbf{Baryogenesis.} Baseline: \textbf{standard leptogenesis} via the
SN sector. Optional: \textbf{leakage‑phase baryogenesis}, where a
pointer‑dependent phase in the Dirichlet form biases sphalerons
(predicts CP‑odd leakage observables in lab GKSL analogs).

\begin{center}\rule{0.5\linewidth}{0.5pt}\end{center}

\hypertarget{chapter-13-predictions-calibration-tests}{%
\section{Chapter 13 --- Predictions, Calibration \&
Tests}\label{chapter-13-predictions-calibration-tests}}

\begin{quote}
\textbf{Scope.} Testable predictions with one primary KPI each, minimal
supports, and calibration procedures. Constants are linked to
multipliers via envelope identities. \textbf{Global KPI carried
through.} We fit the \textbf{coherence number}
(\chi=\tau\emph{\{\rm dec\}/\tau}\{\rm mess\}) in each setting and
extract multipliers via the envelope identities (App. E.4). Each
subsection names \textbf{one predictive bit} (primary KPI) and at most
two supports; ``ablation'' notes indicate how relaxing assumptions would
alter the fit.
\end{quote}

\hypertarget{toy-world-validation-of-ca-and-qubit}{%
\subsubsection{\texorpdfstring{13.0 Toy-world validation of
(\mathrm{CL}) (CA and
qubit)}{13.0 Toy-world validation of () (CA and qubit)}}\label{toy-world-validation-of-ca-and-qubit}}

\textbf{Protocol.} We run the CA toy ((\mathrm{CL}\emph{\{\rm CA\})) on
(n!\times!n) grids with a fixed menu (\mathscr T}\{\rm CA\}) of
thresholds and weights, and the qubit toy ((\mathrm{CL}\_\{\rm Q\}))
with ((\rho\emph{0,\rho\emph{1)) aligned/misaligned with (W). For each,
we estimate the global KPI (\chi=\tau}\{\rm dec\}/\tau}\{\rm mess\}) and
fit multipliers via the envelope identities.

\textbf{Predicted invariants.} (i) \textbf{Monotone equivalence:}
rankings of (A) by (\mathrm{CL}\emph{\{\rm CA\}) and
(\mathrm{CL}}\{\rm Q\}) agree up to an increasing transform; (ii)
\textbf{Pointer alignment:} leakage-minimizing generators align with
(W); (iii) \textbf{Ablation:} relaxing Ad-invariance reintroduces a
fourth quadratic direction and breaks the equivalence (Section 2.2
note).

\textbf{Measurement.} Report (\mathrm{CL})-scores with (95\%) binomial
CIs, estimated (\chi), and inferred multipliers. Each run uses finite,
auditable protocols with reproducible seeds.

\hypertarget{gravitationalwave-phasing-residual-binary-inspirals}{%
\subsection{13.1 Gravitational‑wave phasing residual (binary
inspirals)}\label{gravitationalwave-phasing-residual-binary-inspirals}}

\textbf{Prediction.} Phase residual \(\Delta p\) relative to GR
templates scales as
\(\Delta p\approx \varepsilon\,\eta\,(\pi G M_c f/c^3)^{\!\eta}\) with
\(\varepsilon\sim0.05,\eta\approx1\). \textbf{KPI:} bias‑robust
estimator of \(\eta\) across events. \textbf{Supports:}
residual‑vs‑frequency slope; cross‑event scaling with \(M_c\).

\hypertarget{horizon-flux-suppression}{%
\subsection{13.2 Horizon flux
suppression}\label{horizon-flux-suppression}}

\textbf{Prediction.} Outgoing flux \textbf{below} Hawking prediction by
factor \(f(\lambda_{\rm leak},W)\le1\). \textbf{KPI:}
\(\mathcal F_{\rm out}/\mathcal F_{\rm Hawking}\). \textbf{Supports:}
two‑point distortions; pointer‑basis stability.

\hypertarget{pointer-basis-in-noisy-interferometers}{%
\subsection{13.3 Pointer basis in noisy
interferometers}\label{pointer-basis-in-noisy-interferometers}}

\textbf{Prediction.} Environmental weight \(W\) selects a unique pointer
basis minimizing \(\sum_j\omega(L_j^\dagger W L_j)\). \textbf{KPI:}
basis‑aligned decoherence rates vs misaligned configs.

\hypertarget{calibration-of-hbar}{%
\subsection{\texorpdfstring{13.4 Calibration of
\(\hbar\)}{13.4 Calibration of \textbackslash hbar}}\label{calibration-of-hbar}}

With \(\dot A=i\hbar^{-1}[H,A]\), use a two‑level system at frequency
\(\Omega\); measure Fubini--Study speed \(v\); set \(v=2\Delta H/\hbar\)
to calibrate \(\hbar=\lambda_{\rm th}^{-1}\).

\hypertarget{envelope-identities}{%
\subsection{13.5 Envelope identities}\label{envelope-identities}}

Optimal value \(\Phi(\tau)\) ⇒
\(\lambda_\bullet=\partial\Phi/\partial\tau_\bullet\) a.e.; defines
\(\hbar,G,\Lambda,\alpha_i,\ldots\). \textbf{Protocol:} fit \(\lambda\)s
by matching observed invariants under RG constraints.

\hypertarget{measurement-hygiene}{%
\subsection{13.6 Measurement hygiene}\label{measurement-hygiene}}

Collect only predictive signals; cone‑limited timing windows; publish
priors/instruments/code; pre‑register stop rules.

\hypertarget{appendix-a-h0-spaces-norms-budgets-ccompatible}{%
\section{Appendix A --- H0′: Spaces, Norms \& Budgets
(C*‑compatible)}\label{appendix-a-h0-spaces-norms-budgets-ccompatible}}

\begin{itemize}
\tightlist
\item
  \textbf{Algebra.} Inductive‑limit C*‑algebra
  \(\mathcal A=\overline{\bigcup_\Lambda \mathcal A_\Lambda}\).
\item
  \textbf{State/GNS.} Faithful normal state \(\omega\), GNS triple
  \((\pi_\omega,\mathcal H_\omega,\Omega_\omega)\); identify
  \(\mathcal A\) with
  \(\pi_\omega(\mathcal A)\subset\mathcal B(\mathcal H_\omega)\).
\item
  \textbf{Core.} Dense invariant \(\mathcal D\) common to all
  generators.
\end{itemize}

\textbf{Throughput quadratic.} On each finite \(\Lambda\), using
normalized HS inner product from \(\mathrm{tr}_\Lambda/d_\Lambda\):

\$\$
\mathcal B\_\{\rm th\}\textsuperscript{\Lambda:=\tfrac12\textbar{}\delta\emph{H\textbar{}}\{\{\rm der\};\Lambda\}}2,\quad
\boxed{\mathcal B_{\rm th}:=\sup_\Lambda \mathcal B_{\rm th}^\Lambda}.{]}

\textbf{Leakage budget.} {[} \boxed{\,\mathcal B_{\rm leak}(\{L_j\},W)
= \sum_j \| W^{1/2} L_j \|_{2,\omega}^2
= \sum_j \omega(L_j^\dagger W L_j)\,} \$\$

\textbf{Complexity budget.}

\[
\boxed{\,\mathcal B_{\rm cx}(\mathcal L)
= \inf\Big\{\sum_\alpha \kappa_\alpha \|\mathcal J_\alpha\|_{cb}^2\;:\; 
\mathcal L=\sum_\alpha \mathcal J_\alpha\ \text{ on }\ \mathcal A_{\rm loc}\Big\}\,}
\]

\textbf{Collected properties.}
\(\mathcal B_{\rm th},\mathcal B_{\rm leak},\mathcal B_{\rm cx}\) are
convex, l.s.c.; sublevels equi‑coercive (Ch. 4). Only three independent
budgets up to nulls (Thm I.1).

\emph{Notation.} We use \textbf{\(B_{\rm cx}\)} (``complexity'')
interchangeably with the local‑decomposition \textbf{compression}
quadratic; \(B_{\rm cx}\equiv \mathcal B_{\rm cx}\).

\hypertarget{appendix-b-largedeviation-machinery-risksensitive-worstcase}{%
\section{Appendix B --- Large‑Deviation Machinery (Risk‑Sensitive ⇒
Worst‑Case)}\label{appendix-b-largedeviation-machinery-risksensitive-worstcase}}

\begin{itemize}
\tightlist
\item
  Varadhan's lemma for \(\log\mathbb E e^{\beta\ell}\) ⇒
  \(\beta\to\infty\) yields worst‑case inner min.
\item
  Laplace principle on \(\mathcal A\) under coercivity and u.s.c.
\item
  Epi‑convergence in \(\beta\):
  \(\mathrm{CL}_\beta\searrow \inf_\Phi \mathrm{CL}\).
\end{itemize}

\hypertarget{appendix-c-gaugematter-derivations-yukawas-anomalies-hypercharge}{%
\section{Appendix C --- Gauge/Matter Derivations (Yukawas, Anomalies,
Hypercharge)}\label{appendix-c-gaugematter-derivations-yukawas-anomalies-hypercharge}}

Binder set, linear system, and solution giving the SM hypercharges
(details as in Ch. 6).

\hypertarget{appendix-d-ux3b3-limit-to-einsteinhilbert-airtight-revision}{%
\section{Appendix D --- Γ-Limit to Einstein--Hilbert (airtight
revision)}\label{appendix-d-ux3b3-limit-to-einsteinhilbert-airtight-revision}}

\begin{quote}
\textbf{Goal.} Construct an explicit, gauge-fixed family
\(\{\mathcal F_\varepsilon\}_{\varepsilon\downarrow 0}\) whose Γ-limit
on the admissible class is precisely the Einstein--Hilbert (EH)
functional with cosmological constant. We (i) correct the order mismatch
in the approximant, (ii) unify measures, (iii) prove equi-coercivity in
the right topology, (iv) secure Γ-liminf and recovery with coefficient
stability via a smoothing functor, (v) identify the principal symbol,
and (vi) fix \(G\) and \(\Lambda\) by two independent normalizations.
\end{quote}

\begin{center}\rule{0.5\linewidth}{0.5pt}\end{center}

\hypertarget{d.1-setting-gauge-admissible-class-as-previously-accepted}{%
\subsection{D.1 Setting, gauge, admissible class (as previously
accepted)}\label{d.1-setting-gauge-admissible-class-as-previously-accepted}}

\begin{itemize}
\item
  \textbf{Manifold \& charts.} \(M\) is a smooth 4-manifold of bounded
  geometry (uniform injectivity radius; all curvature derivatives
  bounded) with a countable harmonic-chart atlas.
\item
  \textbf{Cone class.} Admissible metrics \(g\) lie in the
  \textbf{single-cone}, time-oriented class \(\mathfrak G\): there exist
  Lorentzian \(g_\*,g^\*\) with \(g_\*\le g\le g^\*\) a.e. in harmonic
  coordinates, uniformly on compacts.
\item
  \textbf{Gauge.} We work \textbf{on the de Donder slice}
  \(\mathcal F_\mu(g)=0\) where

  \[
  \mathcal F_\mu(g):=g_{\mu\nu}\Gamma^\nu(g),\qquad
  \Gamma^\nu(g):=g^{\alpha\beta}\Gamma^\nu_{\alpha\beta}(g).
  \]

  The gauge holds distributionally in each harmonic chart.
\item
  \textbf{Topology.} Unless otherwise stated, convergence is
  \textbf{weak \(H^1_{\rm loc}\)} in harmonic charts (see D.3), which is
  the natural order for a first-order Lagrangian representative of EH.
\end{itemize}

\begin{center}\rule{0.5\linewidth}{0.5pt}\end{center}

\hypertarget{d.2-approximating-family-mathcal-f_varepsilon-with-explicit-first-order-coefficients-and-a-single-measure}{%
\subsection{\texorpdfstring{D.2 Approximating family
\(\mathcal F_\varepsilon\) with \textbf{explicit first-order}
coefficients and a \textbf{single
measure}}{D.2 Approximating family \textbackslash mathcal F\_\textbackslash varepsilon with explicit first-order coefficients and a single measure}}\label{d.2-approximating-family-mathcal-f_varepsilon-with-explicit-first-order-coefficients-and-a-single-measure}}

\hypertarget{d.2.1-the-first-order-eh-representative-ux3b3ux3b3-form}{%
\subsubsection{D.2.1 The first-order EH representative (Γ·Γ
form)}\label{d.2.1-the-first-order-eh-representative-ux3b3ux3b3-form}}

Recall the classical identity (valid in any coordinates):

\[
\sqrt{|g|}\,R(g)
=\sqrt{|g|}\,g^{\alpha\beta}\!\left(\Gamma^\mu_{\alpha\nu}\Gamma^\nu_{\beta\mu}-\Gamma^\mu_{\alpha\beta}\Gamma^\nu_{\mu\nu}\right)
\ +\ \partial_\alpha\!\big(\sqrt{|g|}\,V^\alpha(g)\big),
\tag{D.1}
\]

where
\(V^\alpha(g)=g^{\mu\nu}\Gamma^\alpha_{\mu\nu}-g^{\alpha\mu}\Gamma^\nu_{\mu\nu}\).
Using

\[
\Gamma^\mu_{\alpha\beta}=\tfrac12 g^{\mu\lambda}\!\left(\partial_\alpha g_{\beta\lambda}+\partial_\beta g_{\alpha\lambda}-\partial_\lambda g_{\alpha\beta}\right),
\tag{D.2}
\]

the \textbf{non-divergence} part of \(\sqrt{|g|}R(g)\) is a
\textbf{purely first-order quadratic} in \(\partial g\). Expanding
(D.1)--(D.2) yields the canonical ``Γ·Γ'' quadratic density

\[
\boxed{\;
\mathcal Q(g,\partial g)
:=\sqrt{|g|}\,g^{\alpha\beta}
\Big(\Gamma^\mu_{\alpha\nu}\Gamma^\nu_{\beta\mu}-\Gamma^\mu_{\alpha\beta}\Gamma^\nu_{\mu\nu}\Big)
\;=\; \mathsf C^{\alpha\beta\,\mu\nu\rho\sigma}(g)\,(\partial_\alpha g_{\mu\nu})(\partial_\beta g_{\rho\sigma}),
\;}
\tag{D.3}
\]

with \textbf{explicit} coefficient tensor

\[
\mathsf C^{\alpha\beta\,\mu\nu\rho\sigma}(g)
=\tfrac14\sqrt{|g|}\,g^{\alpha\beta}\Big(
g^{\mu\rho}g^{\nu\sigma}+g^{\mu\sigma}g^{\nu\rho}-2\,g^{\mu\nu}g^{\rho\sigma}
\Big)
+\tfrac14\sqrt{|g|}\Big(
g^{\alpha\mu}g^{\beta\rho}g^{\nu\sigma}+g^{\alpha\mu}g^{\beta\sigma}g^{\nu\rho}
-g^{\alpha\mu}g^{\beta\nu}g^{\rho\sigma}-g^{\alpha\rho}g^{\beta\sigma}g^{\mu\nu}
\Big),
\tag{D.4}
\]

which is obtained by substituting (D.2) into (D.1) and collecting the
\((\partial g)(\partial g)\) terms. (Any equivalent explicit expansion
is acceptable; (D.4) fixes a concrete choice.) Thus,

\[
\sqrt{|g|}\,R(g)=\mathcal Q(g,\partial g)+\partial_\alpha\!\big(\sqrt{|g|}\,V^\alpha(g)\big).
\tag{D.5}
\]

\begin{quote}
\textbf{Order fix.} There is \textbf{no zeroth-order surrogate} for the
Γ·Γ piece; all non-divergence contributions are first order in
\(\partial g\). We therefore \textbf{remove} the erroneous term
\(\langle B_0(g)g,g\rangle\) and work with the explicit first-order
tensor \(\mathsf C(g)\) in (D.4).
\end{quote}

\hypertarget{d.2.2-unified-measure-and-the-approximating-family}{%
\subsubsection{D.2.2 Unified measure and the approximating
family}\label{d.2.2-unified-measure-and-the-approximating-family}}

Fix a smooth \textbf{reference volume}
\(d{\rm vol}_\sharp:=\sqrt{|g_\sharp|}\,d^4x\) from a bounded-geometry
Riemannian metric \(g_\sharp\) (used only for analytic control). Let

\[
J(g):=\frac{\sqrt{|g|}}{\sqrt{|g_\sharp|}}\quad\text{(Jacobian density)}.
\tag{D.6}
\]

Define, for \(\varepsilon\in(0,1]\),

\[
\boxed{\;
\begin{aligned}
\mathcal F_\varepsilon(g)\;:=\;\frac{1}{16\pi\,\mathsf G_\varepsilon}\!
\int_M\!\Big[
J(g)^{-1}\,\mathcal Q(g,\partial g)\;+\;\varepsilon\,\mathsf a\,\langle\nabla g,\nabla g\rangle_{g_\sharp}
\;+\;\varepsilon^2\,\mathsf b\,\langle\nabla^2 g,\nabla^2 g\rangle_{g_\sharp}
\Big]\ d{\rm vol}_\sharp \\
-\;\frac{\Lambda_\varepsilon}{8\pi\,\mathsf G_\varepsilon}\int_M\! J(g)\ d{\rm vol}_\sharp\;+\;\mathcal B_{\rm bdry}[g;\varepsilon],
\end{aligned}
\;}
\tag{D.7}
\]

with fixed constants \(\mathsf a,\mathsf b>0\). The boundary/shell term
\(\mathcal B_{\rm bdry}\) is the \textbf{chartwise integral of the
divergence} in (D.5) written against \(d{\rm vol}_\sharp\) (or zero if
boundaryless/decay suffices). Thus the \textbf{entire integrand uses one
measure} \(d{\rm vol}_\sharp\); the EH density appears as
\(J(g)^{-1}\mathcal Q\) + divergence (cf.~(D.5)--(D.6)).

\emph{Notes.} (i) The first-order stabilizer
\(\varepsilon\,\langle\nabla g,\nabla g\rangle_{g_\sharp}\) and the
second-order
\(\varepsilon^2\langle\nabla^2 g,\nabla^2 g\rangle_{g_\sharp}\) are
analytic \textbf{regularizers} (vanishing in the Γ-limit) that ensure
tightness in \(H^1\) and control of oscillations. (ii) Gauge is imposed
at the level of the \textbf{admissible class} (D.1), so no gauge penalty
term is needed; if one works off-slice, add
\(\varepsilon\!\int |{\rm div}_g g|^2\) contracted with
\(g_\sharp^{-1}\) without changing the limit on \(\mathfrak G\).

\begin{center}\rule{0.5\linewidth}{0.5pt}\end{center}

\hypertarget{d.3-equi-coercivity-in-weak-h1_rm-loc}{%
\subsection{\texorpdfstring{D.3 Equi-coercivity in \textbf{weak
\(H^1_{\rm loc}\)}}{D.3 Equi-coercivity in weak H\^{}1\_\{\textbackslash rm loc\}}}\label{d.3-equi-coercivity-in-weak-h1_rm-loc}}

We carry Γ-convergence in the \textbf{\(H^1_{\rm loc}\)} topology, the
natural order for the first-order representative of EH.

\hypertarget{d.3.1-local-guxe5rding-type-lower-bound}{%
\subsubsection{D.3.1 Local Gårding-type lower
bound}\label{d.3.1-local-guxe5rding-type-lower-bound}}

On any harmonic chart \(U\Subset M\), cone bounds give uniform
ellipticity/comparability between \(g\) and \(g_\sharp\). Freezing
coefficients and using (D.3)--(D.4) one obtains, for some \(c_U,C_U>0\)
(depending only on the cone and bounded-geometry constants):

\[
\int_U J(g)^{-1}\,\mathcal Q(g,\partial g)\ d{\rm vol}_\sharp
\ \ge\
-c_U\int_U |\partial g|^2_{g_\sharp}\ d{\rm vol}_\sharp
\ -\ C_U.
\tag{D.8}
\]

(The possible negative part reflects Lorentzian signature; it is
\textbf{uniformly controlled} on \(\mathfrak G\).)

\hypertarget{d.3.2-equi-coercivity-from-the-vanishing-regularizer}{%
\subsubsection{D.3.2 Equi-coercivity from the vanishing
regularizer}\label{d.3.2-equi-coercivity-from-the-vanishing-regularizer}}

Combining (D.8) with the explicit regularizers in (D.7) gives, on each
chart,

\[
\mathcal F_\varepsilon(g)\ \ge\ \frac{\varepsilon\,\mathsf a}{16\pi}\int_U |\partial g|^2_{g_\sharp}\ d{\rm vol}_\sharp\ -\ C_U,
\tag{D.9}
\]

and summing over a finite cover of a compact exhaustion of \(M\) yields
\textbf{tightness of sublevel sets in \(H^1_{\rm loc}\)} uniformly in
\(\varepsilon\). (The \(\varepsilon\)-weighted first-order stabilizer is
precisely what prevents high-frequency escape while still vanishing in
the limit; see Γ-liminf/D.4.)

\begin{quote}
\textbf{Conclusion (D.3).} The family \(\{\mathcal F_\varepsilon\}\) is
\textbf{equi-coercive} in the weak \(H^1_{\rm loc}\) topology on
\(\mathfrak G\).
\end{quote}

\begin{center}\rule{0.5\linewidth}{0.5pt}\end{center}

\hypertarget{d.4-ux3b3-liminf-and-recovery-with-stable-coefficients}{%
\subsection{D.4 Γ-liminf and recovery (with stable
coefficients)}\label{d.4-ux3b3-liminf-and-recovery-with-stable-coefficients}}

Two technical issues are addressed: \textbf{(i)} coefficient stability
along weakly convergent sequences and \textbf{(ii)} gauge-preserving
recovery with an \textbf{elliptic} operator.

\hypertarget{d.4.1-coefficients-via-a-smoothing-functor-s}{%
\subsubsection{\texorpdfstring{D.4.1 Coefficients via a smoothing
functor
\(S\)}{D.4.1 Coefficients via a smoothing functor S}}\label{d.4.1-coefficients-via-a-smoothing-functor-s}}

Let \(S\) be a fixed, bounded-geometry \textbf{smoothing operator}
defined chartwise by heat-kernel regularization with respect to
\(g_\sharp\) at scale \(\tau(\varepsilon)\downarrow 0\), patched by a
partition of unity; then

\[
S:\ H^1_{\rm loc}\to C^{0,\alpha}_{\rm loc}\cap L^\infty_{\rm loc},\qquad
S(g_\varepsilon)\to S(g)\ \text{uniformly on compacts if }g_\varepsilon\rightharpoonup g\text{ in }H^1_{\rm loc}.
\tag{D.10}
\]

We \textbf{define} the coefficient tensor in the approximants by

\[
\mathsf C_\varepsilon(\cdot):=\mathsf C\big(S(\cdot)\big),\quad
J_\varepsilon(\cdot):=J\big(S(\cdot)\big),
\tag{D.11}
\]

i.e., \textbf{Carathéodory structure}: measurable in \(x\), continuous
in the (smoothed) field. Then along any
\(g_\varepsilon\rightharpoonup g\),

\[
\mathsf C_\varepsilon(g_\varepsilon)\to \mathsf C\big(S(g)\big),\qquad
J_\varepsilon(g_\varepsilon)\to J\big(S(g)\big)
\quad\text{in }L^\infty_{\rm loc},
\tag{D.12}
\]

and since \(S(g)\to g\) in \(L^2_{\rm loc}\) and a.e., replacing
\((\mathsf C,J)\) by \((\mathsf C(Sg),J(Sg))\) does \textbf{not change
the limit functional} (standard stability argument; see Γ-limsup below).
For readability we suppress \(S\) in what follows; the proof implicitly
uses (D.11)--(D.12).

\hypertarget{d.4.2-ux3b3-liminf}{%
\subsubsection{D.4.2 Γ-liminf}\label{d.4.2-ux3b3-liminf}}

Let \(g_\varepsilon\rightharpoonup g\) in \(H^1_{\rm loc}\), with
\(g_\varepsilon,g\in\mathfrak G\). By (D.12) and weak lower
semicontinuity of convex quadratic forms in \(\partial g\),

\[
\liminf_{\varepsilon\downarrow 0}\!\int J(g_\varepsilon)^{-1}\,\mathcal Q(g_\varepsilon,\partial g_\varepsilon)\,d{\rm vol}_\sharp
\ \ge\
\int J(g)^{-1}\,\mathcal Q(g,\partial g)\,d{\rm vol}_\sharp.
\tag{D.13}
\]

The stabilizers are nonnegative, so they only \textbf{increase} the
liminf. Using (D.5) and the definition of \(\mathcal B_{\rm bdry}\),

\[
\int J(g)^{-1}\,\mathcal Q(g,\partial g)\,d{\rm vol}_\sharp
=\int \sqrt{|g|}\,R(g)\,d^4x\ -\ \int \partial_\alpha\!\big(\sqrt{|g|}\,V^\alpha(g)\big)\,d^4x,
\tag{D.14}
\]

and the last term cancels with \(\mathcal B_{\rm bdry}\) (or vanishes
under the boundary/decay conditions built into
\(\mathcal B_{\rm bdry}\)). Therefore

\[
\boxed{\;
\Gamma\text{-}\liminf_{\varepsilon\downarrow 0}\ \mathcal F_\varepsilon(g_\varepsilon)
\ \ge\
\mathcal F_0(g)
:=\frac{1}{16\pi\,\mathsf G}\int \sqrt{|g|}\,R(g)\,d^4x\ -\ \frac{\Lambda}{8\pi\,\mathsf G}\int \sqrt{|g|}\,d^4x.
\;}
\tag{D.15}
\]

\hypertarget{d.4.3-recovery-sequences-gauge-preserving-elliptic-repair}{%
\subsubsection{D.4.3 Recovery sequences (gauge-preserving, elliptic
repair)}\label{d.4.3-recovery-sequences-gauge-preserving-elliptic-repair}}

Fix \(g\in\mathfrak G\). Choose a compact exhaustion and mollify
chartwise by \(S\) to get smooth \(g^{(k)}:=S_{\tau_k}(g)\to g\) in
\(H^1_{\rm loc}\) and a.e., preserving the cone bounds for all large
\(k\). The mollification may violate de Donder gauge \textbf{slightly};
repair gauge using a \textbf{fixed elliptic} reference operator:

\[
\Delta_{g_\sharp}X^{(k)}=\mathcal F\!\left(g^{(k)}\right)\quad\text{(in each chart with Dirichlet data)},\qquad
\tilde g^{(k)}:=(\exp X^{(k)})^\*g^{(k)}.
\tag{D.16}
\]

Standard elliptic estimates on bounded-geometry charts give
\(\|X^{(k)}\|_{H^2}\lesssim \|\mathcal F(g^{(k)})\|_{L^2}\to 0\), hence
\(\tilde g^{(k)}\to g\) in \(H^1_{\rm loc}\) and
\(\mathcal F(\tilde g^{(k)})=0\). Set \(\varepsilon_k\downarrow 0\) with
\(\tau_k\downarrow 0\) and define \(g_{\varepsilon_k}:=\tilde g^{(k)}\).
Using (D.12) and dominated convergence,

\[
\lim_{k\to\infty}\ \mathcal F_{\varepsilon_k}(g_{\varepsilon_k})\
=\ \mathcal F_0(g).
\tag{D.17}
\]

Thus \textbf{recovery sequences exist for every \(g\in\mathfrak G\)}.

\begin{center}\rule{0.5\linewidth}{0.5pt}\end{center}

\hypertarget{d.5-principal-symbol-identification-lichnerowicz-in-de-donder}{%
\subsection{D.5 Principal-symbol identification (Lichnerowicz in de
Donder)}\label{d.5-principal-symbol-identification-lichnerowicz-in-de-donder}}

Linearize at a bounded-geometry background \(\bar g\in\mathfrak G\):
write \(g=\bar g+h\) with \(h\in C_0^\infty(S^2T^*M)\) satisfying the
linearized de Donder condition
\(\bar\nabla^\mu (h_{\mu\nu}-\tfrac12\bar g_{\mu\nu}h)=0\). The second
variation of the first-order part (D.3)--(D.4) produces the
\textbf{Lichnerowicz operator}; its principal symbol is

\[
\sigma_{\rm pr}(\mathcal E_0)(x,\xi)[h]
=-\tfrac12\big(\bar g^{\alpha\beta}\xi_\alpha\xi_\beta\big)\Big(h_{\mu\nu}-\tfrac12\bar g_{\mu\nu}h\Big),
\tag{D.18}
\]

i.e.~the gauge-fixed Einstein symbol. The \(\varepsilon\)-regularizers
contribute \(O(\varepsilon|\xi|^2)\) and \(O(\varepsilon^2|\xi|^4)\)
symbols, which \textbf{vanish} in the Γ-limit and serve only for
tightness.

\begin{center}\rule{0.5\linewidth}{0.5pt}\end{center}

\hypertarget{d.6-fixing-g-and-lambda-by-two-independent-normalizations}{%
\subsection{\texorpdfstring{D.6 Fixing \(G\) and \(\Lambda\) by two
\textbf{independent}
normalizations}{D.6 Fixing G and \textbackslash Lambda by two independent normalizations}}\label{d.6-fixing-g-and-lambda-by-two-independent-normalizations}}

\begin{itemize}
\item
  \textbf{Weak-field (Poisson) limit → \(G\).} Around Minkowski \(\eta\)
  in global harmonic coordinates, for static weak fields
  \(g_{00}=-(1+2\phi)\), \(g_{ij}=(1-2\phi)\delta_{ij}\), coupling to
  matter through the operational \(T^{\rm eff}\) (Chapter 5) yields

  \[
  \Delta\phi=4\pi\,\mathsf G\,\rho.
  \]

  Matching Newtonian gravity fixes \(\boxed{\ \mathsf G=G\ }\).
\item
  \textbf{Constant-curvature normalization → \(\Lambda\).} For
  \(\mathrm{Ric}(g_\kappa)=3\kappa\,g_\kappa\) (de Sitter/anti-de Sitter
  class), stationarity gives
  \(G_{\mu\nu}(g_\kappa)+\Lambda\,g_{\mu\nu}=0 \iff \Lambda=3\kappa\).
  Hence \(\boxed{\ \Lambda=3\kappa\ }\) fixed independently of the
  weak-field fit.
\end{itemize}

\begin{center}\rule{0.5\linewidth}{0.5pt}\end{center}

\hypertarget{d.7-ux3b3-limit-theorem-final}{%
\subsection{D.7 Γ-limit theorem
(final)}\label{d.7-ux3b3-limit-theorem-final}}

\begin{quote}
\textbf{Theorem D (Γ-limit to Einstein--Hilbert).} On the
cone-preserving, de Donder-gauge class \(\mathfrak G\) with the
\textbf{weak \(H^1_{\rm loc}\)} topology, the functionals
\(\{\mathcal F_\varepsilon\}\) of (D.7) are equi-coercive and
\textbf{Γ-converge} to

\[
\boxed{\;
\mathcal F_0(g)=\frac{1}{16\pi G}\int_M \sqrt{|g|}\,\big(R(g)-2\Lambda\big)\,d^4x.
\;}
\]

Moreover: (i) the \textbf{principal symbols} of the Euler--Lagrange
operators converge to the Lichnerowicz symbol in de Donder gauge (D.18);
(ii) \textbf{recovery sequences} exist for every \(g\in\mathfrak G\)
(D.17); (iii) the constants \(G\) and \(\Lambda\) are fixed by the
\textbf{two normalizations} in D.6 (and not by a single calibration).
\end{quote}

\emph{Proof sketch.} Equi-coercivity: D.3. Γ-liminf: D.4.2 with
(D.12)--(D.15). Γ-limsup: D.4.3. Symbol: D.5. Constants: D.6. The
divergence term is neutralized by \(\mathcal B_{\rm bdry}\) (or decay),
so the limit functional equals EH exactly, not modulo a boundary
integral.

\begin{center}\rule{0.5\linewidth}{0.5pt}\end{center}

\hypertarget{remarks-on-robustness}{%
\subsubsection{Remarks on robustness}\label{remarks-on-robustness}}

\begin{enumerate}
\def\labelenumi{\arabic{enumi}.}
\tightlist
\item
  \textbf{Order correctness.} The non-divergence EH part is
  \textbf{purely first-order}; no zeroth-order surrogate appears. This
  is enforced by (D.3)--(D.5).
\item
  \textbf{Unified measure.} All integrals are against
  \(d{\rm vol}_\sharp\); the EH density is represented as
  \(J(g)^{-1}\mathcal Q\) (plus a divergence accounted for in
  \(\mathcal B_{\rm bdry}\)).
\item
  \textbf{Coefficient stability.} The \textbf{Carathéodory+smooth}
  design (D.10)--(D.12) ensures coefficients converge uniformly on
  compacts along weak \(H^1\) sequences, closing the gap flagged by the
  reviewer.
\item
  \textbf{Elliptic gauge repair.} The gauge-restoration step uses
  \(\Delta_{g_\sharp}\), not a Lorentzian wave operator, eliminating any
  ambiguity and preserving the cone.
\item
  \textbf{No reliance on a single calibration.} \(G\) and \(\Lambda\)
  are fixed by \textbf{two independent} backgrounds (Newtonian and
  constant-curvature), completing the identification.
\end{enumerate}

\emph{This completes the airtight Appendix D in line with the requested
fixes.}

\hypertarget{appendix-e-kkt-riesz-and-unbounded-operator-hygiene}{%
\section{Appendix E --- KKT, Riesz, and Unbounded-Operator
Hygiene}\label{appendix-e-kkt-riesz-and-unbounded-operator-hygiene}}

\hypertarget{e.0-standing-setting-finite-blocks-inductive-limit}{%
\subsection{E.0 Standing setting (finite blocks → inductive
limit)}\label{e.0-standing-setting-finite-blocks-inductive-limit}}

On a finite local block \$\Lambda\$ with matrix algebra
\$\mathcal A\_\Lambda\simeq M\_\{d\_\Lambda\}\$, endow the space of
linear maps \$X:\mathcal A\_\Lambda!\to!\mathcal A\_\Lambda\$ with the
\textbf{normalized Hilbert--Schmidt} inner product

\[
\langle X,Y\rangle_{{\rm der};\Lambda}
:=\frac{1}{d_\Lambda}\,\operatorname{Tr}_{\rm HS}\!\big(X^\dagger Y\big),
\qquad 
\|X\|_{{\rm der};\Lambda}^2=\langle X,X\rangle_{{\rm der};\Lambda}.
\]

For a (symmetric) Hamiltonian \$H\in\mathcal A\_\Lambda\$, the
\textbf{inner derivation}

\[
\delta_H(A):=i[H,A]
\]

is skew-adjoint with respect to
\$\langle\cdot,\cdot\rangle\_\{\{\rm der\};\Lambda\}\$, and every
derivation on \$M\_\{d\_\Lambda\}\$ is inner. The throughput budget on
\$\Lambda\$ is

\[
\mathcal B_{\rm th}^\Lambda(H):=\tfrac12\|\delta_H\|_{{\rm der};\Lambda}^{2},\qquad 
\mathcal B_{\rm th}:=\sup_\Lambda \mathcal B_{\rm th}^\Lambda.
\]

\hypertarget{e.1-metric-matching-gradient-quadratic}{%
\subsection{E.1 Metric matching (gradient ↔
quadratic)}\label{e.1-metric-matching-gradient-quadratic}}

All \textbf{Gateaux/Fréchet derivatives} of the predictive term are
taken with respect to the same normalized HS inner product that defines
\$\mathcal B\_\{\rm th\}\$. Equivalently, whenever a linear functional
on velocities \$X\$ appears (e.g., \$X\mapsto D\Phi\_A{[}X{]}\$), its
\textbf{Riesz representer} is computed in
\$\langle\cdot,\cdot\rangle\_\{\{\rm der\};\Lambda\}\$. This removes the
Banach/Hilbert mismatch and legitimizes the KKT step that equates a
derivative to the gradient in the very metric used by the quadratic
penalty.

\hypertarget{e.2-dynamics-from-a-time-parametrized-variational-principle}{%
\subsection{E.2 Dynamics from a time-parametrized variational
principle}\label{e.2-dynamics-from-a-time-parametrized-variational-principle}}

We now provide the \textbf{missing bridge} from stationarity to
dynamics. Work first on a fixed block \$\Lambda\$.

\hypertarget{e.2.1-riesz-representer-on-the-derivation-cone}{%
\subsubsection{E.2.1 Riesz representer on the derivation
cone}\label{e.2.1-riesz-representer-on-the-derivation-cone}}

Let the predictive term at \$A\in\mathcal A\_\Lambda\$ have a Gateaux
derivative \$D\Phi\_A{[}\cdot{]}\$ continuous on the space of
velocities. Restrict it to the \textbf{admissible velocity space}

\[
\mathsf{Der}_\Lambda:=\{\delta_K:\ K=K^\dagger\in\mathcal A_\Lambda\}\subset \mathcal B(\mathcal A_\Lambda),
\]

which is a finite-dimensional Hilbert space under
\$\langle\cdot,\cdot\rangle\_\{\{\rm der\};\Lambda\}\$. By Riesz, there
exists a \textbf{unique} (up to addition of a multiple of the identity
inside the commutator) \$K\_\Phi(A)=K\_\Phi(A)\^{}\dagger\$ such that

\[
\boxed{\quad D\Phi_A[\delta_J]\;=\;\big\langle \delta_{K_\Phi(A)}\,,\,\delta_J\big\rangle_{{\rm der};\Lambda}
\quad\text{for all }J=J^\dagger.\quad}
\]

We call \$\delta\_\{K\_\Phi(A)\}\$ the \textbf{predictive Riesz
representer} at \$A\$ (restricted to derivations).

\hypertarget{e.2.2-instantaneous-ascent-optimization-and-eulerlagrange-velocity}{%
\subsubsection{E.2.2 Instantaneous ascent optimization and
Euler--Lagrange
velocity}\label{e.2.2-instantaneous-ascent-optimization-and-eulerlagrange-velocity}}

Fix \$A\$ and \$H\$ on \$\Lambda\$, and consider the
\textbf{instantaneous} (small-time) optimization over admissible
velocities \$X\in\mathsf{Der}\_\Lambda\$:

\[
\mathcal L_A(X;H):=D\Phi_A[X]\;-\;\frac{\lambda_{\rm th}}{2}\,\|X\|_{{\rm der};\Lambda}^{2}.
\]

This is a strictly concave quadratic in \$X\$ with unique maximizer

\[
\boxed{\quad X^*_A=\lambda_{\rm th}^{-1}\,\Pi_{\mathsf{Der}_\Lambda}\big(\nabla\Phi_A\big)
\;=\;\lambda_{\rm th}^{-1}\,\delta_{K_\Phi(A)}.\quad}
\]

Hence the \textbf{Euler--Lagrange velocity field} on \$\Lambda\$ is the
predictive Riesz representer scaled by
\$\lambda\_\{\rm th\}\^{}\{-1\}\$.

\hypertarget{e.2.3-kkt-in-h-alignment-of-riesz-and-throughput-directions}{%
\subsubsection{E.2.3 KKT in \$H\$: alignment of Riesz and throughput
directions}\label{e.2.3-kkt-in-h-alignment-of-riesz-and-throughput-directions}}

The blockwise Lagrangian in \$(A,H)\$ reads

\[
\mathsf S_\Lambda(A,H)=\Phi(A)\;-\;\lambda_{\rm th}\,\mathcal B_{\rm th}^\Lambda(H)\;+\;\cdots,
\qquad \mathcal B_{\rm th}^\Lambda(H)=\tfrac12\|\delta_H\|_{{\rm der};\Lambda}^2.
\]

Stationarity in \$H\$ (Slater interior and metric matching) yields the
\textbf{KKT identity on \$\Lambda\$}

\[
\boxed{\quad D\Phi_A[\delta_H]\;=\;\lambda_{\rm th}\,\langle \delta_H,\delta_H\rangle_{{\rm der};\Lambda}
\quad\Longleftrightarrow\quad \delta_{K_\Phi(A)}=\lambda_{\rm th}\,\delta_H.\quad}
\]

(The forward equivalence is precisely the Riesz characterization in
§E.2.1.) Substituting this into the maximizer of §E.2.2 gives

\[
X^*_A=\lambda_{\rm th}^{-1}\,\delta_{K_\Phi(A)}=\delta_H.
\]

\hypertarget{e.2.4-emergent-unitary-dynamics-and}{%
\subsubsection{\texorpdfstring{E.2.4 Emergent unitary dynamics and
\$\hbar\$}{E.2.4 Emergent unitary dynamics and \$\$}}\label{e.2.4-emergent-unitary-dynamics-and}}

Let \$A(t)\$ be an absolutely continuous path with
\$\dot A(t)=X\^{}\emph{\_\{A(t)\}\$ on each block. From
\$X\^{}}\_A=\delta\_H\$ we obtain

\[
\boxed{\quad \dot A(t)=\delta_H\big(A(t)\big)=i[H,A(t)]\quad\text{on every finite block } \Lambda.\quad}
\]

Introducing \textbf{physical time} via
\$t\_\{\rm phys\}:=\lambda\_\{\rm th\},t\$ (units fixed by experiment),
the evolution becomes

\[
\boxed{\quad \frac{d}{dt_{\rm phys}}A(t_{\rm phys})=i\,\hbar^{-1}[H,A(t_{\rm phys})],\qquad \hbar:=\lambda_{\rm th}^{-1}.\quad}
\]

Thus \textbf{\$\hbar\$ is the inverse throughput multiplier} fixed by
the chosen C*-compatible quadratic and operational calibration (e.g.,
Fubini--Study speed).

\hypertarget{e.2.5-inductive-limit-transfer}{%
\subsubsection{E.2.5 Inductive-limit
transfer}\label{e.2.5-inductive-limit-transfer}}

Let \$\{\Lambda\_n\}\$ exhaust the system. If
\$\{\lambda\_\{\rm th\}\^{}\{(\Lambda\_n)\}\}\$ is tight and the KKT
identities hold on each \$\Lambda\_n\$, then by \textbf{graph-closure}
of the derivation and strong convergence of the blockwise flows, the
limit dynamics on the common local core \$\mathcal A\_\{\rm loc\}\$
satisfies \$\dot A=i\hbar\^{}\{-1\}{[}H,A{]}\$.

\hypertarget{e.3-unbounded-gksl-core-closability-drift}{%
\subsection{E.3 Unbounded GKSL: core, closability,
drift}\label{e.3-unbounded-gksl-core-closability-drift}}

Let \$\mathcal D\$ be a common invariant core for \$H\$ and
\$\{L\_j\}\$.

\begin{itemize}
\item
  \textbf{(U1) Closability on the core.} \$\delta\_H\$ is closable on
  \$\mathcal A\_\{\rm loc\}\$; the GKSL form is well-defined on
  \$\mathcal D\$.
\item
  \textbf{(U2) Form bounds.} There exist \$N\ge0\$, \$a\textless1\$,
  \$b\textless{}\infty\$ with

  \[
  \|H\psi\|^2+\sum_j\|L_j\psi\|^2\ \le\ a\,\|N\psi\|^2+b\,\|\psi\|^2,\qquad \psi\in\mathcal D.
  \]
\item
  \textbf{(U3) Quasi-locality.} Finite interaction radius and bounded
  overlap uniformly in volume.
\item
  \textbf{(U4) Semigroup.} The closure generates a unique strongly
  continuous CPTP semigroup; Lieb--Robinson-type bounds hold.
\item
  \textbf{(U5) Lyapunov drift.} For some \$c\_0,c\_1\textgreater0\$,
  \$\mathcal L\^{}*(N)\le c\_0-c\_1N\$ on \$\mathcal D\$.
\end{itemize}

With \$W=(1+N)\^{}\{-s\}\$, \$s\textgreater{}\tfrac12\$, one has
\$\sum\_j\omega(L\_j\^{}\dagger W L\_j)\textless{}\infty\$, so
\$\mathcal B\_\{\rm leak\}\textless{}\infty\$ and the \textbf{leakage
budget is controlled}.

\hypertarget{e.4-envelope-identities-constants-as-multipliers}{%
\subsection{E.4 Envelope identities (constants as
multipliers)}\label{e.4-envelope-identities-constants-as-multipliers}}

Let \$\Phi(\tau)\$ be the optimal value under allowances \$\tau\$. Under
convexity and Slater interior,

\[
\lambda_\bullet(\tau)=\frac{\partial\Phi}{\partial\tau_\bullet}\quad\text{for a.e. }\tau,
\]

pinning \textbf{\$\hbar=\lambda\_\{\rm th\}\^{}\{-1\}\$} and, in the
slow sector, \$G,\Lambda,\ldots\$ as multipliers associated to their
respective allowances.

\textbf{Outcome.} The time-parametrized ascent principle, together with
the KKT alignment of the predictive Riesz representer and the throughput
derivation, yields \$\dot A=i\hbar\^{}\{-1\}{[}H,A{]}\$ with
\$\hbar=\lambda\_\{\rm th\}\^{}\{-1\}\$, and the unbounded-operator
hygiene ensures this extends from blocks to the inductive limit. 

\hypertarget{appendix-f-technical-lemmas-quantitative-gaps-w1-microcausality}{%
\section{Appendix F --- Technical Lemmas \& Quantitative Gaps (W1;
microcausality)}\label{appendix-f-technical-lemmas-quantitative-gaps-w1-microcausality}}

\hypertarget{f.1-orderflip-w_1-gap}{%
\subsection{\texorpdfstring{F.1 Order‑flip ⇒ \(W_1\)
gap}{F.1 Order‑flip ⇒ W\_1 gap}}\label{f.1-orderflip-w_1-gap}}

For a flip tube with mass parameter \(\theta\) and gradient bound \(L\),
define a 1‑Lipschitz test via a clipped arrival‑time difference.
Kantorovich--Rubinstein duality gives
\(W_1(P_g,P_{\tilde g})\ \ge\ \theta\,\frac{\Delta v_*}{L}\,\ell_{\rm poke}.\)

\hypertarget{f.2-microcausality-guards}{%
\subsection{F.2 Microcausality guards}\label{f.2-microcausality-guards}}

Cone‑limited Lieb--Robinson‑type bounds for GKSL commutators;
principal‑symbol lemma enforcing single‑cone hyperbolicity.

\hypertarget{appendix-g-horizon-pointer-mechanics-area-law-details}{%
\section{Appendix G --- Horizon Pointer Mechanics: Area Law
Details}\label{appendix-g-horizon-pointer-mechanics-area-law-details}}

Block decomposition near horizon; additivity and boundary terms; minimal
tilings and constants. Links to Ch. 7 area‑law statement.

\hypertarget{appendix-h-hawking-flux-suppression-microcausality-asymptotics}{%
\section{Appendix H --- Hawking Flux Suppression \& Microcausality
(Asymptotics)}\label{appendix-h-hawking-flux-suppression-microcausality-asymptotics}}

Transfer‑envelope asymptotics; LR guards; flux comparison (strict
inequality unless \(\lambda_{\rm leak}=0\)).

\hypertarget{appendix-i-notation-functionalanalysis-lemmas}{%
\section{Appendix I --- Notation \& Functional‑Analysis
Lemmas}\label{appendix-i-notation-functionalanalysis-lemmas}}

Topologies; Riesz on blocks; l.s.c.; Γ‑convergence; \(W_1\) duality;
normalized HS inner product; cb‑norm basics.

\hypertarget{appendix-j-estimation-data-kits-operational-measurability}{%
\section{Appendix J --- Estimation \& Data Kits (Operational
Measurability)}\label{appendix-j-estimation-data-kits-operational-measurability}}

\hypertarget{j.1-budget-estimation}{%
\subsection{J.1 Budget estimation}\label{j.1-budget-estimation}}

\begin{itemize}
\tightlist
\item
  \textbf{Throughput:} calibrate \(\hbar=\lambda_{\rm th}^{-1}\) via
  two‑level Fubini--Study speed.
\item
  \textbf{Leakage:} estimate \(\sum_j\omega(L_j^\dagger W L_j)\) by
  tomography in the pointer basis; use s.n.f. weight normalization.
\item
  \textbf{Complexity:} bound \(\mathcal B_{\rm cx}\) via local CP
  decompositions with measured cb‑norm surrogates.
\end{itemize}

\hypertarget{j.2-coherence-functional}{%
\subsection{J.2 Coherence functional}\label{j.2-coherence-functional}}

Design poke ensembles covering a basis of the closure hull; use
diamond‑norm continuity to control errors. Publish kernels, fits, and
code.

\hypertarget{j.3-stop-rules-and-leakage-hygiene}{%
\subsection{J.3 Stop rules and leakage
hygiene}\label{j.3-stop-rules-and-leakage-hygiene}}

Define explicit stop conditions based on primary KPI plateaus; cap
leakage channels per ethics/privacy constraints.

\textbf{Outcome.} Ready‑to‑run templates for experiments that feed the
selection program with auditable inputs.

\hypertarget{appendix-k-prediction-worksheets-fitting-protocols}{%
\section{Appendix K --- Prediction Worksheets \& Fitting
Protocols}\label{appendix-k-prediction-worksheets-fitting-protocols}}

GW phasing residual; horizon flux suppression; interferometer pointer
selection; publishing kit and calibration worksheets aligned with Ch.
10.

\hypertarget{appendix-l}{%
\section{Appendix L }\label{appendix-l}}

\begin{center}\rule{0.5\linewidth}{0.5pt}\end{center}

\hypertarget{l.1-ux3b3limit-promotion-of-budget-tiles-on-frw}{%
\subsection{L.1 Γ‑limit promotion of budget tiles on
FRW}\label{l.1-ux3b3limit-promotion-of-budget-tiles-on-frw}}

Finite‑window proofs (LR quasi‑factorization, pinching contraction,
Dirichlet linearity) extend to FRW via exhaustion by comoving boxes and
cone‑preserving gauge. First variations commute with the
\(\varepsilon\to0\) limit by the localized Mosco/Attouch framework used
in Ch. 5. In particular:

\begin{itemize}
\tightlist
\item
  \textbf{Locality → global envelope.} Tilewise GKSL estimates with
  leakage/throughput/complexity budgets remain stable under FRW
  coarse‑graining; their per‑tile constants are uniform on bounded
  curvature charts.
\item
  \textbf{Continuity of multipliers.} Calibration multipliers
  (\(\alpha_{\rm th},\alpha_{\rm cx},\alpha_{\rm leak}\)) obtained on
  finite windows converge along the exhaustion; the slow‑sector Γ‑limit
  preserves the Euler--Lagrange form with \(T^{\rm eff}\) defined by the
  limit of the fast KKT tensors.
\end{itemize}

\begin{center}\rule{0.5\linewidth}{0.5pt}\end{center}

\hypertarget{l.2-planck-window-constants-and-the-area-coefficient}{%
\subsection{L.2 Planck window constants and the area
coefficient}\label{l.2-planck-window-constants-and-the-area-coefficient}}

The constant \(C\) in the Planck‑window mutual‑information bound is
fixed by:

\begin{enumerate}
\def\labelenumi{\arabic{enumi}.}
\tightlist
\item
  \textbf{Tile spectral gap / log‑Sobolev constants} (pointer‑aligned
  GKSL),
\item
  \textbf{Lieb--Robinson velocity} and mixing length for the fast
  sector, and
\item
  \textbf{Normalized HS metric calibration} that identifies
  \(\hbar=\lambda_{\rm th}^{-1}\).
\end{enumerate}

Under these inputs, the coefficient of the area law is \textbf{unique}
and stable under norm‑equivalent budget representatives; replacing any
quadratic by a norm‑equivalent surrogate leaves the coefficient
invariant. Cross‑boundary predictive content in the Planck window is
area‑limited independently of bare UV counts; the leakage budget acts as
a \textbf{soft UV regulator}.

\begin{center}\rule{0.5\linewidth}{0.5pt}\end{center}

\hypertarget{l.3-coherence-smoothing-vs.-inflation-compatibility-note}{%
\subsection{L.3 Coherence smoothing vs.~inflation (compatibility
note)}\label{l.3-coherence-smoothing-vs.-inflation-compatibility-note}}

The ``coherence‑driven smoothing'' mechanism in Ch. 10.3 relies only on
LR microcausality and Dirichlet linearity and does \textbf{not} require
an inflaton. If an inflaton sector is present in \(T^{\rm vis}\),
pointer‑aligned leakage contributes an effective dissipative term
(friction shift \(\Delta\Gamma\propto\alpha_{\rm leak}\)) while
derivation‑pricing enforces a kinetic penalty
\(\propto\alpha_{\rm th}\). This yields a controlled channel to
warm‑inflation‑like dynamics without violating LR bounds.

\begin{center}\rule{0.5\linewidth}{0.5pt}\end{center}

\hypertarget{l.4-placement-in-the-manuscript}{%
\subsection{L.4 Placement in the
manuscript}\label{l.4-placement-in-the-manuscript}}

Insert \textbf{Appendix L} after \textbf{Appendix K --- Prediction
Worksheets \& Fitting Protocols} and \textbf{before} \textbf{Appendix A1
--- Concrete Coherence Functionals}, preserving the alphabetical order
of lettered appendices while keeping \textbf{A1} as a special technical
appendix.

\begin{center}\rule{0.5\linewidth}{0.5pt}\end{center}

\hypertarget{l.5-refuter-ledger-cosmology-dark-sector}{%
\subsection{L.5 Refuter ledger (cosmology \& dark
sector)}\label{l.5-refuter-ledger-cosmology-dark-sector}}

\begin{enumerate}
\def\labelenumi{\arabic{enumi}.}
\tightlist
\item
  Observation of \textbf{super‑cone signaling} or long‑range poke
  effects in early‑time cosmological data (violates LR microcausality).
\item
  Empirical need for a \textbf{fourth independent quadratic budget} that
  cannot be represented as a norm‑equivalent quadratic under the same
  symmetries/calibration.
\item
  Laboratory observation of \textbf{basis‑invariant} decoherence
  contradicting pointer alignment.
\item
  Cosmological data requiring a time‑varying \(\hbar\) or multipliers
  inconsistent with \textbf{Γ‑calibration}.
\end{enumerate}

\hypertarget{appendix-a1-concrete-coherence-functionals-properties-robustness}{%
\section{Appendix A1 --- Concrete Coherence Functionals: Properties \&
Robustness}\label{appendix-a1-concrete-coherence-functionals-properties-robustness}}

We prove l.s.c., concavity under mixing, and risk-sensitive convergence
for the two concrete CLs of §1.2, and then show robustness within a
broader admissible family.

\hypertarget{a1.1-setup-and-notation}{%
\subsection{A1.1 Setup and notation}\label{a1.1-setup-and-notation}}

Let (\mathscr P) be the admissible poke cone (closed under
composition/mixing), and let (\mathscr T) be a finite family of
protocols (finite observables and post-processings). A \textbf{mixed
pattern} is a probability measure (\mu) on a finite set of base patterns
(\{A\_j\}); its dynamics are mixtures of the base dynamics.

\hypertarget{a1.2-ca-toy-_-is-l.s.c.-and-concave-under-mixing}{%
\subsection{\texorpdfstring{A1.2 CA toy: (\mathrm{CL}\_\{\rm CA\}) is
l.s.c. and concave (under
mixing)}{A1.2 CA toy: (\_\{\}) is l.s.c. and concave (under mixing)}}\label{a1.2-ca-toy-_-is-l.s.c.-and-concave-under-mixing}}

\textbf{Lemma A1.1 (l.s.c.).} On a fixed (n,T), the maps (
(A,\Phi)\mapsto S\^{}\{T\_\{\rm CA\}\}\emph{\{A,\Phi\},,M\^{}\{T}\{\rm CA\}\}\emph{\{A,\Phi\},,L\^{}\{T}\{\rm CA\}\}\emph{\{A,\Phi\}
) are continuous in the product of discrete/trace topologies; hence any
finite affine combination is continuous, and
(\mathrm{CL}}\{\rm CA\}=\max\emph{\{T}\{\rm CA\},u\}F\_\{T\_\{\rm CA\},u\})
is l.s.c.

\emph{Proof.} Each is a cylinder-event probability or bounded
expectation on a finite Markov chain parameterized by finitely many
transition probabilities; continuity follows from finite-dimensionality.
Max over a finite set preserves l.s.c. ∎

\textbf{Lemma A1.1′ (concavity under mixing).} If (\mu) is a
distribution over base patterns (\{A\_j\}), then
(A\mapsto \mathbb E\_\mu[F_{T_{\rm CA},u}(A,\Phi)]) is affine in (\mu);
the pointwise maximum over ((T\_\{\rm CA\},u)) of affine maps is
\textbf{concave}. Hence (A\mapsto \mathrm{CL}\_\{\rm CA\}(A,\Phi)) is
concave in (\mu).

\emph{Proof.} Linearity in (\mu) is immediate (law of total
expectation). Pointwise max of affine maps is concave. ∎

\hypertarget{a1.3-quantum-toy-_-is-l.s.c.-and-concave-under-mixing}{%
\subsection{\texorpdfstring{A1.3 Quantum toy: (\mathrm{CL}\_\{\rm Q\})
is l.s.c. and concave (under
mixing)}{A1.3 Quantum toy: (\_\{\}) is l.s.c. and concave (under mixing)}}\label{a1.3-quantum-toy-_-is-l.s.c.-and-concave-under-mixing}}

\textbf{Lemma A1.2 (l.s.c.).} The map
(\mathcal N\mapsto \textbar{}\mathcal N(\Delta)\textbar{}\emph{1) is
continuous in the diamond norm; taking
(\inf}\{\Phi\in\overline{\mathscr P}\}) yields an l.s.c. functional
(\mathrm{CL}\_\{\rm Q\}) on ((A,\Phi)).

\emph{Proof.} For fixed (\Delta),
(\textbar{}\cdot(\Delta)\textbar{}\emph{1\le \textbar{}\cdot\textbar{}}\diamond\textbar{}\Delta\textbar{}\emph{1).
Continuity in (\textbar{}\cdot\textbar{}}\diamond) and the infimum over
a compact poke class give l.s.c. ∎

\textbf{Lemma A1.2′ (concavity under mixing).} If (A\sim \mu) is a
mixture of base channels (\{\mathcal N\_\{A\_j,\Phi\}\}), then
(\textbar{}\mathbb E\_\mu[\mathcal N_{A,\Phi}(\Delta)]\textbar{}\emph{1\le \mathbb E}\mu[\|\mathcal N_{A,\Phi}(\Delta)\|_1]).
Thus (A\mapsto F\_\{\rm Q\}(A,\Phi)) is concave in (\mu), and so is
(A\mapsto \mathrm{CL}\emph{\{\rm Q\}(A,\Phi)) after (\inf}\Phi).

\emph{Proof.} Triangle inequality and Jensen for the norm; the affine
dependence on (\mu) gives concavity. ∎

\hypertarget{a1.4-risk-sensitive-worst-case-limit}{%
\subsection{A1.4 Risk-sensitive worst-case
limit}\label{a1.4-risk-sensitive-worst-case-limit}}

\textbf{Proposition A1.2 (risk-sensitive limit).} For either concrete CL
and any poke law (\Pi), define
(\mathrm{CL}\emph{\beta(A):=\beta\^{}\{-1\}\log\mathbb E}\{\Phi\sim\Pi\}\exp(\beta,\mathrm{CL}(A,\Phi))).
Then
(\mathrm{CL}\emph{\beta\searrow \inf}\{\Phi\in\overline{\mathscr P}\}\mathrm{CL}(A,\Phi))
pointwise and epi-converges as (\beta\to\infty).

\emph{Proof.} Standard log-moment generating function convergence
(Varadhan/Cramér type) on bounded functionals; epi-convergence follows
from monotone convergence properties. ∎

\hypertarget{a1.5-robustness-no-fine-tuning}{%
\subsection{A1.5 Robustness: no
fine-tuning}\label{a1.5-robustness-no-fine-tuning}}

\textbf{Proposition A1.3 (equivalence class of CLs).} Let (\mathcal F)
be the family {[}
\mathcal F:=\Big\{\mathrm{CL}\emph{s(A,\Phi):=\sup}\{T\in\mathscr T\}\mathbb E\big[s_T(Z_{A,\Phi})\big]~\Big\textbar~s\_T:\mathcal Z\_T\to\mathbb R~\text{bounded, concave, strictly increasing in success features}\Big\}.
{]} On any bounded window and admissible poke cone, there exist
constants (0\textless c\_1\le c\_2\textless{}\infty) and increasing
functions (h\_1,h\_2) such that for any (s,s') in the family, {[}
h\_1!\big(\mathrm{CL}\emph{s(A,\Phi)\big)~\le~\mathrm{CL}}\{s'\}(A,\Phi)~\le~h\_2!\big(\mathrm{CL}\_s(A,\Phi)\big),
{]} uniformly on compact sets. Consequently, maximizing
(\mathrm{CL}\_s-\sum\lambda\emph{i B\_i) or
(\mathrm{CL}}\{s'\}-\sum\lambda\emph{i B\_i) has the \textbf{same
maximizers} in the (\Gamma)-limit, and the KKT multipliers
(e.g.~(\hbar=\lambda}\{\rm th\}\^{}\{-1\})) agree.

\emph{Proof sketch.} Bounded concave scores on finite observables are
bi-Lipschitz equivalent up to increasing transforms on compact ranges;
the sup over a finite (\mathscr T) preserves equivalence. Stability of
maximizers and multipliers follows from (\Gamma)-equivalence and
envelope regularity established in Ch. 4--5. ∎

\textbf{Corollary A1.4 (pointer robustness).} For (s) chosen as the
quantum reliability score (F\_\{\rm Q\}) or any strictly increasing
concave transform thereof, leakage-budget minimizers align with the
(W)-eigenbasis (pointer basis), independent of the particular (s).

\end{document}
